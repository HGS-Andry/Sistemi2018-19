\documentclass[a4paper, 12pt, italian]{book}
\usepackage{mathptmx}
\usepackage[T1]{fontenc}
\usepackage[utf8]{inputenc}
\usepackage[italian]{babel}
\usepackage{graphicx}
\usepackage{float}
\usepackage{amsmath}
\usepackage{mathtools}
\usepackage{amsfonts}
\usepackage{bbm}
\usepackage{enumerate}
\usepackage[margin=2cm]{geometry}
\usepackage{amsthm}
\usepackage{mathrsfs}
\usepackage{cancel}
\usepackage{tikz}
\usepackage{booktabs}
\usepackage{siunitx}

%%%%%%%%%%%%%%%%%%%%%%%%%%%%%%%%%%%%%
%% Definizione caption
%%%%%%%%%%%%%%%%%%%%%%%%%%%%%%%%%%%%%
\usepackage{caption}
\captionsetup{tableposition=top,figureposition=bottom,font=small}
\captionsetup{format=hang,labelfont={sf,bf}}

%%%%%%%%%%%%%%%%%%%%%%%%%%%%%%%%%%%%%
%% Definizione struttura
%%%%%%%%%%%%%%%%%%%%%%%%%%%%%%%%%%%%%
\newenvironment{theorem}{Teorema: }{}	% Teorema
\newenvironment{exercise}{Esercizio:}{}	% esercizio					  
%\newenvironment{definition}{Definizione: }{}
%\newenvironment{oss}{Osservazione: }{}
\theoremstyle{remark}
\newtheorem*{example}{Es}				% esempio non numerato
\newtheorem{nexample}{Es}{\itshape}		% esempio numerato secondo il capitolo e sezione
\numberwithin{nexample}{section}
\theoremstyle{definition}				% definizione
\newtheorem*{definizione}{Def}
\theoremstyle{remark}					% osservazione
\newtheorem*{osservazione}{Oss}	
%\newenvironment{example}{Esempio:}{}		
%\newenvironment{proof}{Dim:}{}	

%%%%%%%%%%%%%%%%%%%%%%%%%%%%%%%%%%%%%
%% Preamboli Ivan
%%%%%%%%%%%%%%%%%%%%%%%%%%%%%%%%%%%%%
\newcommand{\AdC}{ascisse di convergenza }
\newcommand{\RdC}{regione di convergenza }
\newcommand{\NB}{I nota bene della zia wilma}

%%%%%%%%%%%%%%%%%%%%%%%%%%%%%%%%%%%%%
%% Funzioni frequenti
%%%%%%%%%%%%%%%%%%%%%%%%%%%%%%%%%%%%%

\renewcommand{\l}{\ell}										% simbolo elle come indice
\renewcommand{\d}{\, d}										% simbolo per fiire la derivata, aggiunge automaticamente uno spazio
\newcommand{\N}{\mathbb{N}} 								% insieme numeri naturali
\newcommand{\Z}{\mathbb{Z}}									% insieme numeri interi
\newcommand{\R}{\mathbb{R}} 								% insieme numeri reali
\newcommand{\C}{\mathbb{C}} 								% insieme numeri complessi
\renewcommand{\Re}{Re} 										% parte reale di un numero immaginario
\renewcommand{\Im}{Im} 										% parte immaginaria di un numero immaginario
\newcommand{\LA}{\mathscr{L}} 								% trasformata di Laplace
\newcommand{\TdL}{\mathscr{L}} 								% trasformata di Laplace
\newcommand{\FO}{\mathcal{F}} 								% trasformata di Fourier
\DeclarePairedDelimiter{\abs}{\lvert}{\rvert}			 	% assoluto normale
\DeclarePairedDelimiter{\Abs}{\bigg\lvert}{\bigg\rvert} 	% assoluto grande
\newcommand{\intZeroInfinity}{\int_{0^-}^{+\infty}}			% integrale da 0- a +infinito
\newcommand{\gradino}{\delta_{-1}}							% gradino
\newcommand{\derN}[2]{\frac{d^{#2}#1}{dt^{#2}}}				% derivata n
\newcommand{\der}[1]{\frac{d#1}{dt}}						% derivata
\newcommand{\dbdec}{\, \si{dB/dec}} 						% unità di misura dB/dec
\newcommand{\sng}{sng} 										% funzione segno

% voglio compilare sono cap1_Intro.tex
%\includeonly{./capitoli/cap4_TrasfDiLaplace, ./capitoli/cap5_TrasfDiFourier}
% NB \includeonly va prima di \begin{document}

%%%%%%%%%%%%%%%%%%%%%%%%%%%%%%%%%%%%%
%%Inizio documento
%%%%%%%%%%%%%%%%%%%%%%%%%%%%%%%%%%%%%

\begin{document}

\begin{titlepage}

\begin{center}
\LARGE{Università degli Studi di Verona}\\
\line(1,0){450}\\
\vspace{10em}
\Huge{\textbf{Appunti di Sistemi}}\\
\vspace{14em}
\Large{di Wilma Valentino, Ivan Piazza, Laura Maule, Andrea Dall'Alba}\\
\line(1,0){450}\\
\LARGE{2019}\\
\end{center}

\end{titlepage}

\tableofcontents

\chapter{Introduzione}
\section{ Premessa }

\textbf{Def - Sistemi dinamici:}  sono un'evoluzione temporale di un ingresso e un'uscita, vengono modellati dalle \textbf{equazioni differenziali}.\\
Vedremo sono funzioni monodimensionali (cioè dipendenti solo dal tempo) ma in realtà possiamo benissimo avere più dimensioni.\\
I modelli ci danno una stima del comportamento di una sistema, in particolare il modello delle equazioni differenziali si può svolgere: nel tempo, nel dominio della trasformata di Laplace (utile con i segnali esponenziali e sinusoidali) e nel dominio della trasformata di Fourier (è un restrizione di Laplace, è utile nei segnali sinusoidali).\\
Studieremo le proprietà dei sistemi, di queste la più importante è la stabilità.\\
\textbf{Modelliamo un sistema} come fosse una scatola nera, avrà un segnale come input e output (es. suono 1D, immagine 2D).\\
I sistemi ci servono per vari motivi: possono facilitare la trasmissione di un segnale, migliorarlo accentuando alcune informazioni o eliminandone altre (filtraggio).\\
Vogliamo vedere i sistemi come \textbf{funzioni matematiche} (pag. 69 libro "Structure and Interpretation of Signals and Systems")\\

\begin{figure}
	\centering
	\includegraphics[width=0.7\linewidth]{immagini/sistema}
	\caption{ modello di un sistema come scatola nera}
	\label{fig:sistema}
\end{figure}



$ \forall s \in D' $ allora $ v(s)=(S(u))(s)\in R'  $ \\

\pagebreak

Abbiamo tre tipi di sistemi:\\
- Continui: operano su segnali continui\\
- Discreti: operano su segnali discreti\\
- Ibridi fra continui e discreti (non li vedremo)\\

\begin{figure}[h]
	\centering
	\includegraphics[width=0.7\linewidth]{immagini/seno}
	\caption{Segnale continuo}
	\label{fig:seno}
\end{figure}


\begin{figure}[h]
	\centering
	\includegraphics[width=0.7\linewidth]{immagini/tempo_discreto}
	\caption{ Segnale discreto}
	\label{fig:tempodiscreto}
\end{figure}

\textbf{Quantizzazione e campionamento}\\
Per andare da analogico a digitale (A->D) ho bisogno di campionamento e quantizzazione.\\

Campionamento: trasformiamo il dominio. Non ho sempre perdita di dati.\\

\begin{figure}[h]
	\centering
	\includegraphics[width=0.7\linewidth]{immagini/tempo_discreto}
	\caption{ Segnale campionato nel tempo}
	\label{fig:tempodiscreto}
\end{figure}

\pagebreak
Quantizzazione: trasformiamo il codominio. Ha sempre perdita di informazione.\\

\begin{figure}[h]
	\centering
	\includegraphics[width=0.7\linewidth]{immagini/quantizzato}
	\caption{ Segnale sinusoidale quantizzato nelle ampiezze}
	\label{fig:quantizzato}
\end{figure}

\textbf{Sistemi LTI}\\
Studiamo un tipo particolare di sistemi, che hanno due proprietà: linearità e tempo invarianza (L=lineari, TI=tempo invarianti).\\
- Linearità:\\
Se $ v_{1}=S(u_{1}) $ e $ v_{2}=S(u_{2})  $ 
allora $ S(\alpha u_{1} + \beta u_{2}) 
= \alpha S(u_{1}) + \beta S(u_{2})
= \alpha v_{1} + \beta v_{2} $ con $ \alpha , \beta \in C^{*} $ \\
Oppure grazie al \textbf{principio di sovrapposizione degli effetti} ( stabilisce che per un sistema dinamico lineare l'effetto di una somma di perturbazioni in ingresso è uguale alla somma degli effetti prodotti da ogni singola perturbazione): \\
$ S( \sum_{i=0}^n \alpha_i u_i ) 
=  \sum_{i=0}^n \alpha_i S( u_i )
$ \\
La linearità è dovuta alle equazioni differenziali, all'interno hanno la derivata prima che è essa stessa lineare.\\
- Tempo invariante:\\
 Significa che l'uscita non dipende esplicitamente dal tempo, cioè se un ingresso x(t) produce l'uscita y(t) allora per ogni ingresso traslato $x(t+ \delta )$ si ha un'uscita traslata dello stesso fattore $y(t+ \delta )$.\\
 
 $u(t+ t_0 ) \rightarrow v(t+ t_0 )$
 
\begin{figure}[h]
	\centering
	\includegraphics[width=0.7\linewidth]{immagini/sistema2}
	\caption{ Sistema generico }
	\label{fig:sistema2}
\end{figure}

\begin{figure}[h]
	\centering
	\includegraphics[width=0.7\linewidth]{immagini/esponenziale}
	\caption{ Esempio di tempo invariante }
	\label{fig:esponenziale}
\end{figure}

\pagebreak

I sistemi LTI che studieremo sono definiti da equazioni differenziali a coefficienti costanti.

Per i sistemi continui useremo il modello:
\begin{equation*}
\sum\limits_{i=0}^n a_{i} \frac{ d^{i} v }{ dt^{i}  } = \sum\limits_{i=0}^m b_{i} \frac{ d^{i} u }{ dt^{i}  }
\tag{1}\label{equation 1}
\end{equation*}
con $u$ e $v$ funzioni con dominio uguale a $\mathbb{R}$. In generale abbiamo $n\geq m$.


Per i sistemi discreti useremo il modello:
\begin{equation*}
\sum\limits_{i=0}^n a_{i} v(k-i) = \sum\limits_{i=0}^m b_{i} u(k-i)
\end{equation*}
con $u$ e $v$ funzioni con dominio uguale a $\mathbb{Z}$ e $ k \in \mathbb{Z} $.

\subsection*{Sistemi SISO}
Per semplicità considereremo solo i sistemi SISO, cioè single input single output. Ma in generale i sistemi sono MIMO, cioè multiple input multiple output (questi non li vedremo).

\subsection*{Esempio semplice di un sistema}

Nel dominio del tempo rappresentiamo un sistema $S$ continuo come: \\
\begin{equation*}
S: [\mathbb{R} \rightarrow \mathbb{R}] \rightarrow [\mathbb{R} \rightarrow \mathbb{R}]
\end{equation*}

Abbiamo anche visto che modelliamo i sistemi con le derivate (che ehhh non ci piacciono molto) e vorremmo quindi qualcosa di più semplice. Usiamo Laplace (salvatore del popolo) che trasforma le derivate in moltiplicazioni ("Trasformò acqua in vino" semicit.). Le moltiplicazioni sono semplici e belle (insomma tanto love per le moltiplicazioni e per Laplace $\heartsuit$).\\

Nel dominio complesso o delle frequenze rappresentiamo quindi il sistema $S$ continuo come: \\
\begin{equation*}
S: [\mathbb{C} \rightarrow \mathbb{C}] \rightarrow [\mathbb{C} \rightarrow \mathbb{C}]
\end{equation*}

\begin{figure}[h]
	\centering
	\includegraphics[width=0.7\linewidth]{immagini/LaplaceLove}
	\caption{ Come Laplace trasforma dal dominio del tempo a quello complesso }
	\label{fig: Sistema Laplace}
\end{figure}

\subsection*{Esempio di sistema discreto}
Consideriamo il sistema: \\
\begin{equation*}
	\begin{split}
	S: [\mathbb{Z} \rightarrow \mathbb{R}] \rightarrow [\mathbb{Z} \rightarrow \mathbb{R}] \\
	u \mapsto S(u) = v
	\end{split}
\end{equation*}
dove $v(n) = \dfrac{u(n)+u(n-1)}{2} \in \mathbb{R} $. Cioè l'output è $\forall n $ la media dell'input attuale e dell'input precedente (sistema moving average).\\
$u(n)$ può essere una qualsiasi funzione discreta. Prendiamo due esempi diversi di input.\\


Per primo esempio prendo come input:
\begin{equation*}
u(n)=
\begin{cases} 
	1, & \mbox{se }n\geq 0 \\ 
	0, & \mbox{altrimenti}
\end{cases} 
\end{equation*}

\begin{figure}[h]
	\centering
	\includegraphics[scale=0.5]{immagini/gradino}
	\caption{ La funzione gradino (unit step) come input }
	\label{fig: Unit step}
\end{figure}

\pagebreak

Avrò come output:
\begin{equation*}
v(n)=
\begin{cases} 
0, & \mbox{se }n < 0 \\ 
1/2, & \mbox{se }n = 0 \\ 
1, & \mbox{se }n > 0
\end{cases} 
\end{equation*}

\begin{figure}[h]
	\centering
	\includegraphics[scale=0.5]{immagini/gradinoMedia}
	\caption{ L'output del sistema moving average con in input la funzione gradino}
	\label{fig: Output sistema moving average gradino}
\end{figure}
Per secondo esempio prendo come input:
\begin{equation*}
u(n)= \cos(2\pi f n)
\end{equation*}
dove $n \in \mathbb{Z} $ e $f \in \mathbb{R}_{+}^{*}$\\
Avrò come output:
\begin{equation*}
v(n)= S(u)(n) = \frac{1}{2} (\cos(2\pi f n) + \cos(2\pi f (n-1))) = R\cos( 2\pi f n + \theta) \in \mathbb{R}^{*}
\end{equation*}
dove $\theta = \frac{1}{2} \arctan(\frac{ \sin (-2\pi f)    }{1+\cos(2\pi f)})$ e $ R = \sqrt{ 2+2\cos(2\pi f)} \in \mathbb{R}^{*}$\\
NB: Amplifico l'input sinudoidale con R ma il periodo e la frequenza rimangono invariate. In output quindi avrò ancora un segnale sinusoidale (questo fenomeno lo studieremo nella frequence responce).

\pagebreak

\section{ Ripasso sui numeri complessi }

\subsection*{Rappresentazione cartesiana di un numero complesso}
Definiamo $s$ un numero complesso, $ s = x+jy$ con $j=\sqrt{-1}$ o $j^{2} = -1$.\\
$x$ è la parte reale di $s$ e viene scritta come $Re(s)$ \\
$y$ è la parte immaginaria di $s$ e viene scritta come $Im(s)$.\\
Un numero viene definito \textbf{ immaginario puro} se ha $ Re(s) = 0 $ e $ Im(s) = y $, cioè ha la parte reale nulla.

\subsection*{Il piano complesso}

\begin{figure}[h]
	\centering
	\includegraphics[scale=1.2]{immagini/pianoComplesso}
	\caption{ Piano complesso }
	\label{fig: piano complesso}
\end{figure}

\subsection*{Il coniugato complesso}
Il coniugato complesso di $s$ è $ \bar{s} = x -jy $.\\
Da notare come $s\bar{s} = (x+jy)(x-jy) = x^{2}+y^{2}$.\\

\subsection*{L'insieme dei numeri complessi}
$\mathbb{C} =$ \{ $s / s = x+jy, x,y \in \mathbb{R} $\} 

\subsection*{Rappresentazione polare di un numero complesso}

$s$ si può scrivere anche in forma:
\begin{equation*}
 s = \rho (cos\theta + j sen\theta)
\end{equation*}
con $\theta \in \mathbb{R}$ .\\
Definiamo rho come il raggio di $s$, cioè: \\
$\rho := |s| = \sqrt{x^{2}+y^{2}} = \sqrt{ Re^{2}(s)+Im^{2}(s)} = \sqrt{s\bar{s}} = |\bar{s}|$.\\
Definiamo theta come "l'angolo" di $s$, cioè è l'argomento di $s$: \\
$ \theta = Arg(s)$ \\
OSS: $\theta$ non è univoco, devo imporre un limite, che può essere:
$\theta = [0,2\pi)$ oppure $\theta = (-\pi, \pi] $ \\
Da notare come: $Arg(s) = Arg(\bar{s})$


\subsection*{Operazioni con i numeri complessi}
Consideriamo due numeri complessi $ s_{0} = x_{0}+jy_{0} $ e $ s_{1} = x_{1}+jy_{1} $ \\

\textbf{Somma:} \\
$ s_{0} + s_{1} = ( x_{0}+x_{1} ) + j (y_{0} + y_{1}) $ \\

\textbf{Prodotto:} \\
$ s_{0} s_{1} = ( x_{0} + j y_{0})( x_{1} + j y_{1}) $ \\
$ = x_{0} x_{1} - y_{0} y_{1} + j (x_{0} y_{1} + x_{1} y_{0} ) $ questa purtroppo non ci è molto utile.\\
Proviamo quindi ad usare la rappresentazione polare, troviamo che: \\
$ = \rho_{0} (cos\theta_{0} + j sin\theta_{0})\rho_{1} (cos\theta_{1} + j sin\theta_{1}) $ \\
$ = \rho_{0} \rho_{1} ( \cos\theta_{0} \cos\theta_{1} + j \cos\theta_{0} \sin\theta_{1} + j \sin\theta_{0} \cos\theta_{1}-\sin\theta_{0} \sin\theta_{1}) $ \\
$ = \rho_{0} \rho_{1} ( \cos (\theta_{0} + \theta_{1}) + j \sin ( \theta_{0} + \theta_{1} )) $ \\
L'ultima riga è molto più facile da ricordare e da utilizzare.
NB: Dobbiamo sempre rimanere con $\theta \in (-\pi , \pi]$

\begin{figure}[h]
	\centering
	\includegraphics[scale=0.5]{immagini/moltiplicazioneComplessi}
	\caption{ Moltiplicazione fra due complessi }
	\label{fig: Moltiplicazione fra due complessi}
\end{figure}

NB: $ |s_{0} s_{1}| = |s_{0}||s_{1}|$ \\
$ Arg( s_{0} s_{1} ) = Arg(s_{0} ) + Arg( s_{1} )$ sempre $ \in (-\pi , \pi]$\\
$ \bar{s_{0} s_{1}} = \bar{s_{0} } \bar{s_{1}}$ \\

\textbf{Inverso:} \\
Se $ s \neq 0 $ dove $ 0 = 0 + j0 $,\\
definiamo l'inverso di un complesso come $ \frac{ \bar{s} }{ |s|^{2}} = \frac{x-jy}{x^{2} + y^{2}} $ \\
\begin{proof}[Dim]
	\begin{equation*}
	\begin{split}
	s \frac{1}{s} = s \frac{ \bar{s} }{s \bar{s}} = 1 \\
	\end{split}
	\end{equation*}
	$s \frac{1}{s}$ dà proprio 1 quindi è l'inverso.
\end{proof}

\textbf{Quoziente:} \\
Definiamo $\frac{s_{0}}{s_{1}} := s_{0} \frac{ 1 }{ s_{1}} $ \\
$= s_{0} \frac{ \bar{s_{1} }}{x_{1}^{2} + y_{1}^{2}} = \frac{ x_{0} x_{1} + y_{0} y_{1} }{ x_{1}^{2} + y_{1}^{2} } - j \frac{ x_{0} y_{1} + x_{1} y_{0} }{ x_{1}^{2} + y_{1}^{2} } $\\

\subsection*{Numeri complessi unitari}
L'insieme dei numeri complessi unitari è $\mathbb{U} =$ \{ $s \in \mathbb{C} : |s|=1 $\} , cioè l'insieme dei complessi con modulo uguale a 1.\\
Se $ u \in \mathbb{U} $
allora $ \rho = |u| = \sqrt{ x^{2} + y^{2}} = 1 $

\begin{figure}[h]
	\centering
	\includegraphics[scale=0.5]{immagini/pianoComplessoUnitari}
	\caption{ Piano complesso con i numeri unitari }
	\label{fig: piano complesso unitari}
\end{figure}

\textbf{Moltiplicazione tra due numeri complessi unitari:} \\
Prendiamo due numeri complessi $ u = cos\theta +j sin\theta $ e $ s = \rho ( cos( Arg(s) ) + j sin ( Arg(s) ) ) $ \\
allora $ s u = \rho ( \cos ( Arg(s) + \theta) + j \sin (Arg(s) + \theta) ) $

Graficamente abbiamo:
\begin{figure}[h]
	\centering
	\includegraphics[scale=0.5]{immagini/ComplessiUnitariMoltiplicazioni}
	\caption{ Somma degli argomenti di due numeri complessi unitari }
	\label{fig: complesso unitari moltiplicazioni}
\end{figure}

\subsection*{Potenze dei numeri complessi}
Prendiamo un numero complesso $ s = \rho (cos\theta + j sen\theta) $ \\
la sua potenza ennesima sarà $ s^{n} = \rho^{n} (cos(n\theta) + j sen(n\theta ) ) $ sempre ricordando che $n\theta \in (-\pi , \pi ]$.

\subsection*{Radice nei complessi}

Dato $ s = \rho (cos\theta + j sen\theta) $ vogliamo trovare $ w = r (cos\alpha + j sen \alpha )$ tale che $ w^{n} = s$ , cioè w è la radice n-sima di un numero complesso s. \\
s ha n radici complesse, prendiamo $ w_{k} = r (cos\alpha_{k} + j sen \alpha_{k} ) $ con $r = \sqrt{\rho}$ , $ \alpha_{k} = \frac{\theta}{ n} + k \frac{2 \pi}{n} $ e $ k = 0,1,...,n-1$ \\
allora abbiamo che $ w_{k} = s$. \\

Per $ k = 0$ avremo che $ |w_{0}| = \sqrt[n]{ \rho}  $ e $ Arg(w_{0}) = \frac{1}{n} Arg(s)$.\\

\subsection*{Radici complesse dell'unità}

Definiamo l'unità come $s=1$, è comunque un numero complesso unitario ($ |s| = 1$) e quindi anche lui avrà n radici.\\

\begin{figure}[h]
	\centering
	\includegraphics[scale=0.5]{immagini/pianoComplessoUnitari-s}
	\caption{ s=1, numero complesso unitario }
	\label{fig: numero complesso unitario}
\end{figure}


Posso vedere $s=1$ come: $ cos0 + j sen0 $. \\
Qual è quel numero $ w_{k} =  cos\alpha_{k} + j sen \alpha_{k} $ tale che $ w_{k}^{n} = s = 1$? (NB: $ r =1 $ perchè s è unitario cioè il modulo è uguale a 1. NB: dobbiamo trovare $ \alpha_{k}$).\\
$ cos \alpha_{k} $ deve essere uguale a $ cos0 $ e $ sen \alpha_{k} $ deve essere uguale a $ sen0 $. Perciò $\alpha_{k}$ non può essere altro che uguale a $ 0 $ ma anche uguale a tutti i multipli $ 2 \pi$, avrò quindi $ \alpha_{k} = k \frac{ 2 \pi}{n}$ con $ k = 0,1,...,n-1 $.\\

Per $ k = 0$ avremo che $ w_{0} = 1 $ perchè ho proprio $ cos0 + sen0$. \\

Per $ k = 1$ avremo che $ w_{1} = cos(\frac{2 \pi}{n}) + j sen( \frac{2 \pi}{n}) $.\\

\begin{figure}[h]
	\centering
	\includegraphics[scale=0.5]{immagini/esempioSUnitario}
	\caption{ n=16, graficamente le radici 16-esime di s=1 }
	\label{fig: esempioRadiciSUNitario}
\end{figure}

Tuttociò è anche dovuto al \textbf{teorema fondamentale dell'algebra} (lo vedremo fra poco) che dice che $ x^{n} = 1$ (con $ x^{n}$ visto come polinomio) avrà n soluzioni complesse. \\

\pagebreak

\section{ Funzioni a variabile complessa }

\subsection*{Def: funzione a variabile complessa}
Definiamo una funzione f a variabile complessa come:
$ f: D(f) \longrightarrow \mathbb{C} $ con $ D(f)\subset \mathbb{C} $ e $ D(f) $ insieme aperto.

\subsection*{Def: insieme aperto}
$ D(f) $ è un insieme aperto $ \Leftrightarrow \forall s_{0} \in D(f) , D(f)\subset \mathbb{C} , \exists $ un disco $ B_{\rho}(s_{0}) \subset D(f)$. \\
$ B_{\rho}(s_{0})$ può essere:\\
- il disco centrato in $ s_{0}$ di raggio $ \rho $\\
- \{ $ s \in \mathbb{C} / |s-s_{0}| < \rho $ \}\\

\textbf{NB: disco = intorno sferico aperto in Analisi 2, aperto = non considero il perimetro}

\begin{figure}[h]
	\centering
	\includegraphics[scale=0.75]{immagini/disco}
	\caption{ Esempio di disco }
	\label{fig: esempioDisco}
\end{figure}


\textbf{NB: In analisi 2 avevamo: } \\
$ B_{\rho}(x_{0}) = $ \{ $ x \in \mathbb{R} / |x-x_{0}| < \rho $ \} $ = ( x_{0}-\rho ... x_{0}+\rho) $

\begin{figure}[h]
	\centering
	\includegraphics[scale=0.5]{immagini/intornoSfericoAperto}
	\caption{ Intorno sferico aperto }
	\label{fig: intornoSfericoAperto}
\end{figure}

\subsection*{Esempi di insieme aperto}
1) $ D(f) = \mathbb{C} $ cioè tutto il piano complesso.\\
2) $ D(f) = B_{\rho}(s_{0})$ cioè il disco (visto prima).\\

\pagebreak

3) $ D(f) = $ \{ $ s \in \mathbb{C} : \rho_{1} < |s-s_{0}| < \rho_{2} $ \} cioè una corona circolare. \\
\begin{figure}[h]
	\centering
	\includegraphics[scale=0.75]{immagini/coronaCircolare}
	\caption{ Esempio di insieme aperto: Corona circolare }
	\label{fig: coronaCircolare}
\end{figure}


4) $ D(f) = $ \{ $ s \in \mathbb{C} : Re(s)> \lambda, \lambda \in \mathbb{R} $ \} cioè il semipiano a destra (Questo è il dominio di definizione della trasformata di Laplace). \\

\begin{figure}[h]
	\centering
	\includegraphics[scale=0.75]{immagini/dominioDefLaplace}
	\caption{ Esempio di insieme aperto: dominio di definizione della trasformata di Laplace }
	\label{fig: dominioDefLaplace}
\end{figure}

5) $D(f) = $ \{ $ s \in \mathbb{C} : Re(s)< \lambda, \lambda \in \mathbb{R} $ \} cioè il semipiano a sinistra. \\ 
6) $D(f) = $ \{ $ s \in \mathbb{C} : Im(s) > Im(j\lambda ), \lambda \in \mathbb{R}, Im(j\lambda ) \in \mathbb{C} $ \} cioè il semipiano superiore. \\

\pagebreak

7) $D(f) = $ \{ $ s \in \mathbb{C} : Im(s) < Im(j\lambda ), \lambda \in \mathbb{R}, Im(j\lambda ) \in \mathbb{C} $ \} cioè il semipiano inferiore (Questo è il dominio di definizione della trasformata di Fourier).\\

\begin{figure}[h]
	\centering
	\includegraphics[scale=0.75]{immagini/dominioDefFourier}
	\caption{ Esempio di insieme aperto: dominio di definizione della trasformata di Fourier }
	\label{fig: dominioDefFourier}
\end{figure}

8) $ D(f) = \mathbb{C} \setminus$  \{ $ s_{1},...,s_{n}  $ \}  cioè il piano complesso meno un insieme finito di punti.


\subsection*{Esempi di funzioni a variabili complesse}
1) Funzione identità: $ f(s) = s$ con $ D(f) = \mathbb{C}$.\\
Cioè $ s=x+jy \mapsto s=x+jy $.\\

2) $ f(s) = s^{2}$ con $ D(f) = \mathbb{C}$.\\
Cioè $ s^{2} = (x+jy)^{2} = x^{2}-y^{2} +2jxy $, $ Re(f(s)) = x^{2}-y^{2} $ e $ Im(f(s)) = 2jxy $. \\

3) $ f(s) = s^{n}$ con $ D(f) = \mathbb{C}$.\\
Cioè $ s^{n} = (x+jy)^{n} = \sum_{k=0}^n \binom{n}{k} x^{k} (jy)^{n-k} $ dove ricordo che $ \binom{n}{k} = \frac{n!}{k! (n-k)!} $ il coefficiente binomiale. \\
NB: in coordinate polari diventa \\
Se $ s = \rho (cos\theta + j sen\theta) $ \\
allora $ s^{n} = \rho^{n} (cos(n\theta) + j sen(n\theta ) ) $ \\

4) $ f(s) = \frac{1}{s}$ con $ D(f) = \mathbb{C} \setminus$ \{ 0 \}.\\
Cioè $ \frac{1}{s} = \frac{ \bar{s} }{ |s|^{2}} = \frac{x-jy}{x^{2} + y^{2}} = \frac{ x }{ x^{2} + y^{2} } - j \frac{ y }{ x^{2} + y^{2} } $.\\

5) Funzioni polinomiali:\\
$ P(s) = \sum_{k=0}^n a_{k} s^{k} = a_{n} s^{n} + a_{n-1} s^{n-1} +...+a_{1} s^{1}+a_{0} s^{0} $ con $ a_{k} \in \mathbb{C} $ e $ a_{n} \neq 0$. \\

\textbf{Teorema fondamentale dell'algebra}\\
Ogni polinomio P(s) di grado n ha n radici complesse e si può decomporre in un unico modo:\\
$ P(s) = a_{n} (s- \lambda_{1})^{u_{1}} (s- \lambda_{2})^{u_{2}} ... (s- \lambda_{r})^{u_{r}} $ \\
dove $ \lambda_{1}, \lambda_{2},...,\lambda_{r} $ sono le radici ( cioè $ \forall k $ che va da 1 a r, $ P(\lambda_{k}) = 0$) e $ u_{1},...,u_{r} $ sono le molteplicità.\\

\pagebreak

\textbf{Da questo teorema si evince che:}\\
Tutti i monomi del polinomio sono linearmente indipendenti perchè possiamo vedere lo spazio dei polinomi come uno spazio infinito dimensionale in cui i monomi sono dei vettori. Ogni vettore lo puoi vedere come un vettore della base canonica quindi indipendente dagli altri. Puoi vedere i vettori come coordinate e quindi un monomio del polinomio è una coordinata diversa e a se stante dalle altre. Trovo quindi che $ u_{1}+u_{2}+...+u_{r} = n $.\\
Se prendo $ \lambda_{k} $ in quel punto il polinomio darà zero perchè è una sua radice, cioè $ P(\lambda_{k}) = 0 $. Se derivo il polinomio "gli tolgo una dimensione" ma il monomio non scompare e quindi $ \lambda_{k} $ sarà uno zero anche per la darivata. Avrò che $ P(\lambda_{k}) = P'(\lambda_{k}) = ... = P^{(u_{k}-1)}(\lambda_{k}) = 0 $. \\
Es: $ d(x-2)^3/dx = 3(x-2)^2 $, $ \lambda = 2 $ è radice del polinomio e della sua derivata.\\
Questo però vale fino a $ u_{k}-1 $. Es: $ d 3(x-2)^2/dx = 6(x-2) $ vale ma se derivo ancora $  d 6(x-2)/dx = 6 $ non vale più, cioè $ \lambda_{k} = 2$ non annula questa derivata essendo costante(=6). \\
Vediamo quindi che la derivata di grado $ u_{k} $ cioè uguale alla molteplicità, è la prima derivata non nulla, cioè $ P^{u_{k}}(\lambda_{k}) \neq 0 $.\\

\textbf{OSS1:} nel caso di radici semplici (cioè hanno tutte molteplicità = 1) avrò che:\\
$ P(s) = a_{n} (s- \lambda_{1}) (s- \lambda_{2}) ... (s- \lambda_{r}) $

%TODO: non ho capito questa osservazione
\textbf{OSS2: BOOO???} Se $ a_{k} \in \mathbb{R}$ e $ k=0,...,n $ \\
allora $ \lambda_{k}$ sono tutte radici reali oppure complessi coniugati.\\

\textbf{ES1}\\
$ P_{1}(s) = s^{2} -2s +1 = (s-1)^{2}$\\
Avrò quindi $ \lambda_{1} = 1$ con $ u_{1} = 2$\\
Con $ s=1 $ ho $ P_{1}(1) = 0$.\\
Derivo $ P'_{1}(s) = 2s - 2$.\\
Con $ s=1 $ ho $ P'_{1}(1) = 0$.\\

\textbf{ES2}\\
$ P_{2}(s) = s^{2} +1$\\
Avrò quindi $ \lambda_{1,2} = \pm j$ e $ P_{2}(s) = (s-j)(s+j)$ cioè $ P_{2}( \lambda_{1})=P_{2}( \lambda_{2})=0$\\

6) Funzioni razionali:\\
$ f(s) = \frac{ P(s)}{Q(s)}$ con $ D(f) = \mathbb{C} \setminus$ \{ $ \lambda_{1},..., \lambda_{r}$ \} dove $ \lambda_{k}$ è la radice di $ Q(s)$, cioè $ Q( \lambda_{k})=0$ con $ k=0,...,r$ .\\

\textbf{ES}\\
$ f(s) = \frac{ s^{2}+1}{ s^{2} -1}= \frac{ (s-j)(s+j) }{ (s-1)(s+1)} $ con $ D(f) = \mathbb{C} \setminus$ \{ $ \pm 1$ \}. \\

7) Funzioni come serie di potenze:\\
$ f(s) = \sum_{k=0}^\infty a_{k} (s-s_0)^{k}$ con $ D(f)= \mathbb{C} $. \\
Si associa $r>0$, chiamato \textbf{raggio di convergenza} tale che la serie converge $ \forall s \in Br(s_0) = $ \{ $ s \in \mathbb{C} : |s-s_0| < r$ \} con $ D(f)=Br(s_0)$. La serie non converge per $ s \notin Br(s_0)$.\\

\textbf{ES A): Serie geometrica}\\
$ f(s) = \sum_{k=0}^\infty s^{k} = \sum_{k=0}^\infty (s-0)^{k}$ cioè $s_0= 0$.\\
In questo caso $r=1$ e $ D(f)=B_1(0)=$ \{ $ s \in \mathbb{C} : |s| < 1$ \}.

\pagebreak

\begin{figure}[h]
	\centering
	\includegraphics[scale=0.5]{immagini/raggioDiConvergenza}
	\caption{ Graficamente la regione in cui la sommatoria converge }
	\label{fig: raggioDiConvergenza}
\end{figure}

\begin{proof}[Dim:] voglio dimostrare che r=1 è proprio il raggio di convergenza.\\
	Scriviamo $\sum_{k=0}^\infty s^{k} $ come $ \lim_{N \to \infty} \sum_{k=0}^N s^{k} $ per avere questa sommatoria applichiamo un po' di magia di Gregorio (che non fa mai male):\\
	
	$ s^2 -1 = (s-1)(s+1)$\\
	$ s^3 -1 = (s-1)(s^2+s+1 )$\\
	$ s^n -1 = (s-1)(s^{n-1}+ s^{n-2}+...+1)$\\
	Prendiamo quest'ultima equazione, rielaborandola diventa:\\
	$ \frac{ -(s^n -1) }{-(s-1) } = s^{n-1}+ s^{n-2}+...+1$\\
	$ \frac{ 1-s^n }{ 1-s } = s^{n-1}+ s^{n-2}+...+1$\\
	Sostituendo n=N+1 troviamo che:\\
	$ \frac{ 1-s^{N+1} }{ 1-s } = s^N+ s^{N-1}+...+1 = \sum_{k=0}^N s^{k}$\\
	Quindi $ \sum_{k=0}^N s^{k} = \frac{ 1-s^{N+1} }{ 1-s } $
	Ci basta adesso mettere il limite da entrambe le parti:\\
	 $\lim_{N \to \infty} \sum_{k=0}^N s^{k} = \lim_{N \to \infty} \frac{ 1-s^{N+1} }{ 1-s } $, questo ultimo limite dà $ \frac{1}{1-s} $ se $ |s|<1$.\\
	
	Il raggio di convergenza è proprio r=s=1 cioè con $ |r| < 1$ la serie converge (dà un numero finito).\\
	
\end{proof}

\begin{proof}[Dim:] Spiego qui in breve perchè $\lim_{N \to \infty} \frac{ 1-s^{N+1} }{ 1-s } = \frac{1}{1-s} $ se $ |s|<1$.\\
	
	Andiamo per casi:\\
	1) Se $ s>1 $:\\
	Prendiamo come esempio la funzione $ 2^N $.\\
	
	\begin{figure}[h]
		\centering
		\includegraphics[scale=0.5]{immagini/esp1}
		\caption{ Se s>1 }
		\label{fig: esp1}
	\end{figure}
	
	\pagebreak
	
	Come si può vedere dal grafico il suo limite sarà $ \lim_{N \to \infty} 2^N = \infty $.\\
	Allora in questo caso $ \lim_{N \to \infty} \frac{ 1-s^{N+1} }{ 1-s } = \infty $.\\
	
	2) Se $ 0<s<1 $:\\
	Prendiamo come esempio la funzione $ (\frac{1}{2})^N $.\\
	
	\begin{figure}[h]
		\centering
		\includegraphics[scale=0.5]{immagini/esp2}
		\caption{ Se 0<s<1 }
		\label{fig: esp2}
	\end{figure}

	Come si può vedere dal grafico il suo limite sarà $ \lim_{N \to \infty} (\frac{1}{2})^N = 0 $.\\
	Allora in questo caso $ \lim_{N \to \infty} \frac{ 1-s^{N+1} }{ 1-s } = \frac{1}{1-s} $.\\
	
	3) Se $ -1<s<0 $:\\
	Prendiamo come esempio la funzione $ (-\frac{1}{2})^N $.\\
	
	\begin{figure}[h]
		\centering
		\includegraphics[scale=0.5]{immagini/esp3}
		\caption{ Se -1<s<0 }
		\label{fig: esp3}
	\end{figure}
	
	Come si può vedere dal grafico il suo limite sarà $ \lim_{N \to \infty} (-\frac{1}{2})^N = 0 $.\\
	Allora in questo caso $ \lim_{N \to \infty} \frac{ 1-s^{N+1} }{ 1-s } = \frac{1}{1-s} $.\\
	
	4) Se $ s<-1 $:\\
	Prendiamo come esempio la funzione $ (-2)^N $.\\
	
	\begin{figure}[h]
		\centering
		\includegraphics[scale=0.5]{immagini/esp4}
		\caption{ Se s<-1 }
		\label{fig: esp4}
	\end{figure}

	\pagebreak
	
	Come si può vedere dal grafico il suo limite sarà $ \lim_{N \to \infty} (-2)^N = \infty $.\\
	Allora in questo caso $ \lim_{N \to \infty} \frac{ 1-s^{N+1} }{ 1-s } = \infty $.\\
	
	I casi in cui il limite è finito sono 2) e 3), che infatti danno $ |s| < 1$.\\
	
\end{proof}

\textbf{ES B):}\\
Se $ s \longrightarrow -s$ allora $ f(s) = \sum_{k=0}^\infty (-s)^{k} = \sum_{k=0}^\infty (-1)^{k} (s)^k = 1 -s + s^2 - s^3 + s^4 - s^5...$\\
$ = 1 + s^2 + s^4+ ... - s (1 + s^2 + s^4 + ...) = *$\\
NB: $ 1 + s^2 + s^4+ ... = \sum_{k=0}^\infty s^{2k} = \frac{1}{1+s^2}  $\\
Quindi avrò che $* = \frac{1}{1+s^2} - s \frac{1}{1+s^2}$
$ = \frac{1-s}{1+s^2} = \frac{1}{1+s}$\\
Riassumendo ho che $  \sum_{k=0}^\infty (-s)^{k} = \frac{1}{1+s}$.\\
Il raggio di convergenza è come prima $r=1$.\\

\textbf{ES C):}\\
Se $ s \longrightarrow s/a$ allora $ f(s) = \sum_{k=0}^\infty (\frac{s}{a})^k = \frac{1}{1-\frac{s}{a}} = \frac{1}{\frac{a-s}{a}} = \frac{a}{a-s}$\\
Il limite converge con $ |\frac{s}{a}| <1 $.\\
Con raggio di convergenza $ r = |a|$.

%TODO: ci sarebbe un disegno qui ma non lo capisco
\begin{figure}[h]
	\centering
	\includegraphics[scale=0.75]{immagini/raggioDiConv2}
	\caption{ Graficamente la regione in cui la sommatoria converge }
	\label{fig: raggioDiConvergenza}
\end{figure}

\pagebreak

\textbf{Prima di andare avanti ripassiamo Taylor e Taylor-Mc Laurin}\\
Data una funzione sufficientemente regolare $ f(x):\mathbb{D} \longrightarrow \mathbb{R} $ con $ \mathbb{D} \subset \mathbb{R} $, è sempre possibile approssimarla in un intorno di un dato punto $ x_0 \in \mathbb{D} $, con polinomi g(x) di grado n. Perchè allora si studiano gli sviluppi di Taylor? Il motivo è che, tra tutti i polinomi di grado n, quello di Taylor è quello che meglio stima la funzione di partenza f in un intorno di $ x_0 $.\\
Possiamo scrivere la serie di Taylor come:\\
 $ f(x) = f(x_0) + \frac{ f'(x_0)}{1!} (x-x_0) + \frac{ f''(x_0)}{2!} (x-x_0)^2 +... = \sum_{n=0}^\infty \frac{ f^{(n)}(x_0) }{n!} (x-x_0)^n$\\
Con sviluppo di Taylor-McLaurin si intende uno sviluppo di Taylor con centro $x_0=0$.
In generale avremo quindi che lo sviluppo di Taylor-McLaurin è:\\
$ f(x) = f(0) + \frac{ f'(0)}{1!} x + \frac{ f''(0)}{2!} x^2+... + \frac{ f^{n}(0)}{n!} x^n+ o(x^n)$ \\

\textbf{ES D): Serie di potenze: funzione esponenziale}\\
$ f(s) = e^{s} = \sum_{k=0}^\infty \frac{s^{k}}{k!} = 1 +s + \frac{s^2}{2} + \frac{s^3}{3!} + \frac{s^4}{4!}+... $.\\
Il raggio di convergenza è $ r = +\infty$ e $ D(f)= \mathbb{C}$.\\

\begin{proof}[Dim:] Dimostriamo che $ e^{s} $ è proprio $\sum_{k=0}^\infty \frac{s^{k}}{k!} $\\ 
	Avendo appena visto la serie di Taylor-McLaurin la dimostrazione è facile. Infatti lo sviluppo in serie di $ e^{s} $ è proprio:\\
	$ e^{x} = e^0 + e^0 x + e^0 \frac{x^2}{2} + e^0 \frac{x^3}{3!} + e^0 \frac{x^4}{4!} +... $ \\
	$ = 1 + x + \frac{x^2}{2} + \frac{x^3}{3!} + \frac{x^4}{4!}+... = \sum_{n=0}^\infty \frac{x^{n}}{n!}$ 
\end{proof}


\textbf{ES E): Funzioni trigonometriche viste come serie}\\
Possiamo scrivere la funzione coseno usando Mc Laurin come:\\
$ g(s) = cos(s) = cos(0) + cos'(0)s + \frac{ cos''(0)}{2!} x^2 + \frac{ cos'''(0)}{3!} x^3 + \frac{ cos''''(0)}{4!} x^4 + ... $\\

Svolgiamo i calcoli: $ cos(0) = 1$ , $ cos'(0)s = -sen(0) = 0 $, $\frac{ cos''(0)}{2!} x^2 = - \frac{ cos(0)}{2!} x^2 = - \frac{ x^2}{2!}$, $ \frac{ cos'''(0)}{3!} x^3 = 0 $ e $ \frac{ cos''''(0)}{4!} x^4 = \frac{ x^4}{4!} $.

Quindi avrò $ g(s) = cos(s) = 1 + 0 - \frac{s^2}{2!} + 0 + \frac{s^4}{4!} + 0 -... $\\


Da notare come i segni si continuino ad alternare, questo è dovuto alle derivate del coseno e del seno:\\
$ (cos(s))' = - sen(s)$\\
$ (sen(s))' = cos(s)$\\

Riscrivendolo avrò che $  g(s) = cos(s) = 1 - \frac{s^2}{2!} + \frac{s^4}{4!} -...$, questo si può scrivere anche come $ \sum_{k=0}^\infty (-1)^k \dfrac{s^{2k}}{(2k)!}$. \\

Possiamo vedere anche il seno come serie:\\
$ g(s)=sen(s)= \sum_{k=0}^\infty (-1)^k \dfrac{s^{2k+1}}{(2k+1)!} = s - \frac{s^3}{3!}+\frac{s^5}{5!}-...$

\pagebreak

\textbf{COMANDO MATLAB:}\\

\begin{figure}[h]
	\centering
	\includegraphics{immagini/comando1}
	\caption{ Il comando syms crea variabili, in questo caso x }
	\label{fig: Primo comando Matlab}
\end{figure}

\textbf{Rappresentazione esponenziale di un numero complesso}\\
Mettiamo nell'equazione $ e^s = \sum_{k=0}^\infty \frac{s^k}{k!}$, $s=jw$ un numero complesso immaginario puro.\\
Verrà fuori che $ e^{jw} = 1 + jw + \frac{(jw)^2}{2!}+ \frac{(jw)^3}{3!}+ \frac{(jw)^4}{4!}+... $.\\
Svolgiamo i calcoli: $ j^0 = 1$ , $ j^1 = j $ ,$j^2=-1 $, $ j^3=-j $, $j^4 = -jj= 1$ e così via...\\
Come si può vedere c'è un pattern che continua a ripetersi 1,j,-1,-j. Vogliamo avere le coppie (1,-1) da una parte e (j,-j) da un'altra. Raggruppiamo tutti i fattori senza j a sinistra e isoliamo j (puoi anche vedere che tutti gli esponenti pari vanno a sinistra e tutti quelli dispari vanno fra parentesi a destra), ci verrà fuori che: \\
$ e^{jw} = 1 +jw-\frac{(w)^2}{2!} -j\frac{(w)^3}{3!}+\frac{(w)^4}{4!}+j\frac{(w)^5}{5!}-... $\\
$ = 1-\frac{(w)^2}{2!}+\frac{(jw)^4}{4!}-\frac{(jw)^6}{6!}+...+j(w - \frac{w^3}{3!}+\frac{w^5}{5!}-...)$\\
Tutto quello a sinistra è $ cos(w) $ mentre tutto quello nella parentesi è $ sen(w)$, avrò quindi che:\\
$ e^{jw} = cosw +jsenw$.\\
NB: $ Re(e^{jw}) = cosw $ e $ Im(e^{jw}) = senw$.\\

\begin{figure}[h]
	\centering
	\includegraphics[scale=0.5]{immagini/rappEsp}
	\caption{ Rappresentazione grafica }
	\label{fig: rappEsp}
\end{figure}

\textbf{Nel caso generale:}\\
$ s = \rho ( cos \theta + sen \theta ) = \rho e^{j \theta} $.\\

\textbf{Casi particolari}\\
1) Se $ w = \pi $ allora $ e^{j \pi} = cos( \pi) + j sen(\pi) = -1$. Cioè $ e^{j \pi} +1=0 $ che è l'\textbf{identità di Eurelo}.\\
2) Se $ w = \pi/2 $ allora $ e^{j\frac{\pi}{2}} = cos( \frac{\pi}{2}) + j sen(\frac{\pi}{2}) = j$.\\
3) Se $ w = -w $ allora $ e^{-jw} = cos( -w) + j sen(-w) = cosw-jsenw $, cioè è il coniugato complesso. Infatti $ e^{jw} e^{-jw} = e^{0} = 1 $.\\
4) Sommiamo fra loro $e^{jw}$ e $e^{-jw}$ verrà fuori che:\\
$ e^{jw} + e^{-jw} = 2 cosw $ allora $ cosw = \frac{e^{jw} + e^{-jw}}{2} $.\\
5) Se invece li sottraiamo vediamo che:\\
$ e^{jw} - e^{-jw} = cosw+jsenw-(cosw-senw)=2jsenw $ allora $ senw = \frac{e^{jw} - e^{-jw}}{2j} $.\\


\textbf{Conviene vedere tutte le funzioni come complesse visto che avrò a che fare con equazioni differenziali.}




















\chapter{Segnali elementari}
\section{A tempo continuo}

\subsubsection{Segnali sinusoidali o fasori}

$ v(t)= Acos(wt+ \phi)$ con $t \in \mathbb{R}$.\\
Chiameremo: ampiezza $ A>0$, fase $ \phi $ (e se $ \phi > 0$ ho il segnale traslato a sinistra) , codominio $ [-A...A]$, periodo $ T = \frac{2 \pi}{w} $, pulsazione $w=\frac{2 \pi}{T} $ e frequenza $ f = \frac{1}{T}$ (NB: più grande è $T $ più piccola è f).\\
NB: $ w = 2 \pi f $ \\
NB: Come visto nell'Introduzione posso vedere il coseno come $  cosw = \frac{e^{jw} + e^{-jw}}{2}  $, quindi in questo caso $ cos(wt + \phi) = \frac{e^{jw+ \phi} + e^{-jw + \phi}}{2} $.\\

\subsubsection{Segnali sinusoidali modulati esponenzialmente}

$ v(t)= A e^{ \sigma t} cos(wt+ \phi)$ con $A>0$.\\
NB: In questo caso $T = \frac{2 \pi}{w} $ non è il periodo!\\

\begin{equation*}
v(t)=
\begin{cases} 
 \mbox{Converge, se } \sigma < 0 \rightarrow \mbox{ Il sistema è stabile }\\ 
 \mbox{Diverge, se } \sigma > 0 \rightarrow \mbox{ Il sistema è instabile}
\end{cases} 
\end{equation*}
Cioè $  \lim_{t \to \infty} v(t)=0$, solo se $ \sigma <0$.

\begin{figure}[h]
	\centering
	\includegraphics[scale=0.75]{immagini/segnSinNeg}
	\caption{ Andamento del segnale con $ \sigma <0$ }
	\label{fig: segnSinNeg}
\end{figure}

\pagebreak

\begin{figure}[h]
	\centering
	\includegraphics[scale=0.75]{immagini/segnSinPos}
	\caption{ Andamento del segnale con $ \sigma >0$ }
	\label{fig: segnSinPos}
\end{figure}

Come visto per il precedente segnale posso scrivere il coseno come $  cosw = \frac{e^{jw} + e^{-jw}}{2}  $.\\
In questo caso posso scrivere il segnale come:\\
$ v(t)= \frac{A}{2} e^{ \sigma t} e^{j(wt+ \phi)} + \frac{A}{2} e^{ \sigma t} e^{-j(wt+ \phi)}$\\
$= \frac{A}{2} e^{ \sigma t} e^{ jwt} e^{j \phi} + \frac{A}{2} e^{ \sigma t} e^{ -jwt} e^{-j \phi}$\\
$= \frac{A}{2} e^{j \phi} e^{( \sigma + jw) t} + \frac{A}{2} e^{-j \phi} e^{( \sigma - jw) t}$\\
Con $ s = \sigma + jw $ e $ \bar{s} = \sigma - jw $ posso scrivere che:\\
$= \frac{A}{2} e^{j \phi} e^{st} + \frac{A}{2} e^{-j \phi} e^{\bar{s} t}$\\
Cioè il segnale è la combinazione lineare di esponenziali complesse di cui una è il complesso conugato dell'altra.\\
 

\subsection{Funzioni generalizzate o distribuzioni}

Con distribuzioni intendiamo limiti di successioni di funzioni.\\

\subsubsection{Gradino unitario}
	
	\begin{equation*}
	\delta_{-1}(t)=
	\begin{cases} 
	1, \mbox{ se } t \geq 0\\ 
	0, \mbox{ se } t < 0
	\end{cases} 
	\end{equation*}
	
	\begin{figure}[h]
		\centering
		\includegraphics[scale=0.5]{immagini/gradinoContinuo}
		\caption{ Gradino unitario}
		\label{fig: gradinoContinuo}
	\end{figure}
	
	Proviamo ora a traslarla, con $ t_0 > 0$.\\
	
	Per avere un segnale in ritardo, dovrò traslare a destra e quindi sottraggo $t_0$.\\
	
	\begin{equation*}
	\delta_{-1}(t - t_0 )=
	\begin{cases} 
	1, \mbox{ se } t \geq t_0\\ 
	0, \mbox{ se } t < t_0
	\end{cases} 
	\end{equation*}
	
	\begin{figure}[h]
		\centering
		\includegraphics[scale=0.5]{immagini/gradinoContinuoMeno}
		\caption{ Gradino unitario $ \delta_{-1}(t - t_0 ) $ }
		\label{fig: gradinoContinuoMeno}
	\end{figure}
	
	Per avere un segnale in anticipo, dovrò traslare a sinistra e quindi sommo $t_0$.\\
	
	\begin{equation*}
	\delta_{-1}(t + t_0)=
	\begin{cases} 
	1, \mbox{ se } t \geq -t_0\\ 
	0, \mbox{ se } t < -t_0
	\end{cases} 
	\end{equation*}
	
	\begin{figure}[h]
		\centering
		\includegraphics[scale=0.5]{immagini/gradinoContinuoPiu}
		\caption{ Gradino unitario $ \delta_{-1}(t + t_0 ) $ }
		\label{fig: gradinoContinuoPiu}
	\end{figure}
	
\pagebreak
	
	\textbf{COMANDO MATLAB:}\\
	
	\begin{figure}[h]
		\centering
		\includegraphics[scale=0.75]{immagini/comando2}
		\caption{ Il comando syms crea una variabile t, heaviside è la funzione per il gradino, fplot "plotta" cioè fa il grafico fra [-2...2]  }
		\label{fig: comando2}
	\end{figure}
	
	NB: da notare come la funzione gradino non è continua ma riusciremo comunque a derivarla.\\

\subsubsection{Funzione rettangolare di ampiezza e durata unitaria}
	
	\begin{equation*}
	\varPi(t)=
	\begin{cases} 
	1, \mbox{ se } -\frac{1}{2} \leq t \leq \frac{1}{2}\\ 
	0, \mbox{ altrimenti }
	\end{cases} 
	\end{equation*}
	
	\begin{figure}[h]
		\centering
		\includegraphics[scale=0.5]{immagini/rettangolo}
		\caption{ Funzione rettangolare di ampiezza e durata unitaria $ \varPi(t) $ }
		\label{fig: rettangolo}
	\end{figure}
	
	NB: è una combinazione lineare di due gradini:\\
	$ \varPi(t)= \delta_{-1}(t + \frac{1}{2}) - \delta_{-1}(t-\frac{1}{2}) $\\
	
	In generale il segnale sarà $ A \varPi( \frac{t-t_0}{T}) $, che mi dà una finestra di ampiezza A, durata T e centrata in $t_0$.\\
	
	\begin{equation*}
	A \varPi( \frac{t-t_0}{T})=
	\begin{cases} 
	A, \mbox{ se } -\frac{1}{2} \leq \frac{t-t_0}{T} \leq \frac{1}{2} \rightarrow -\frac{T}{2}+t_0 \leq t \leq \frac{T}{2}+ t_0 \\ 
	0, \mbox{ altrimenti }
	\end{cases} 
	\end{equation*}
	
	\begin{figure}[h]
		\centering
		\includegraphics[scale=0.5]{immagini/rettangoloGenerale}
		\caption{ Funzione rettangolare in generale $ A \varPi( \frac{t-t_0}{T}) $ }
		\label{fig: rettangoloGenerale}
	\end{figure}


\subsubsection{Lambda: impulso triangolare di ampiezza e area unitaria}
	
	\begin{equation*}
	\varLambda(t)=
	\begin{cases} 
	0, \mbox{ se } t \leq -1\\ 
	1-|t|, \mbox{ se } -1 \leq t \leq 1\\ 
	0, \mbox{ se }  t > 1
	\end{cases} 
	\end{equation*}
	
	\begin{figure}[h]
		\centering
		\includegraphics[scale=0.5]{immagini/lambda}
		\caption{ Funzione lambda $ \varLambda(t) $ }
		\label{fig: lambda}
	\end{figure}
	
	In generale il segnale sarà $ A \varLambda(t-t_0) $.\\
	
	\begin{equation*}
	A \varLambda( \frac{t-t_0}{T})=
	\begin{cases} 
	0, \mbox{ se } t \leq t_0-T\\ 
	\dfrac{A}{T}(|t|-t_0+T), \mbox{ se } t_0-T \leq t \leq t_0+T\\ 
	0, \mbox{ se }  t > t_0+T
	\end{cases} 
	\end{equation*}
	
	\begin{figure}[h]
		\centering
		\includegraphics[scale=0.5]{immagini/lambdaGenerale}
		\caption{ Funzione lambda in generale $ A \varLambda( \frac{t-t_0}{T}) $ }
		\label{fig: lambdaGenerale}
	\end{figure}

\pagebreak

\subsubsection{Rampa unitaria}

	\begin{equation*}
	\delta_{-2}(t)=
	\begin{cases} 
	t, \mbox{ se } t \geq 0\\ 
	0, \mbox{ altrimenti }
	\end{cases} 
	\end{equation*}
	
	\begin{figure}[h]
		\centering
		\includegraphics[scale=0.5]{immagini/rampaContinua}
		\caption{ Funzione rampa unitaria $\delta_{-2}(t) $ }
		\label{fig: rampaContinua}
	\end{figure}

\subsubsection{Riassunto: colleghiamo le distribuzioni}

	Notiamo che $ \delta_{-2}(t) = \int_{ -\infty}^{t} \delta_{-1}(\tau)  \, d \tau  $, sappiamo anche che per $ \tau <0 $ abbiamo $ \delta_{-1}(\tau) = 0 $ quindi $ \int_{ 0}^{t} \delta_{-1}(\tau)  \, d \tau $. Da 0 a t, ho $ \delta_{-1}(\tau) = 1 $ allora $ \int_{ 0}^{t} 1  \, d \tau = [ \tau]_0^t  = t $. \\
	Quindi l'integrale del gradino darà la rampa e la derivata della rampa darà il gradino:\\
	$ \delta_{-1}(t) =   \frac{d }{dt}\delta_{-2}(t)  $\\
	
	NB i numeri vicino alla delta saranno:\\
	Impulso o Delta $ \rightarrow 0 $ (lo zero viene omesso)\\
	Funzione costante o gradino $\rightarrow -1 $\\
	Retta o rampa $ \rightarrow -2 $\\
	Parabola $ \rightarrow -3 $\\
	Quest'ultima infatti la possiamo scrivere come:\\
	$ \delta_{-3}(t) = \int_{ -\infty}^{t} \delta_{-2}(\tau) \, d \tau = \int_{ 0}^{t} \tau \, d \tau = \frac{t^2}{2}$ .\\
	
	\begin{equation*}
	\delta_{-3}(t)=
	\begin{cases} 
	\frac{t^2}{2}, \mbox{ se } t \geq 0\\ 
	0, \mbox{ altrimenti }
	\end{cases} 
	\end{equation*}
	
	$ \frac{t^2}{2} $ è proprio una parabola.\\
	Come visto prima $ \delta_{-2}(t) = \frac{d }{dt}\delta_{-3}(t) $.\\
	
	In generale, andando avanti a integrare avrò che:\\
	
	\begin{equation*}
	\delta_{-k}(t)=
	\begin{cases} 
	\frac{t^{k-1}}{ (k-1)! }, \mbox{ se } t \geq 0\\ 
	0, \mbox{ altrimenti }
	\end{cases} 
	\end{equation*}
	
	Avrò quindi una serie di polinomi sempre con un grado in più.\\
	
\pagebreak

	Riassumendo quindi avrò:\\
	
	\begin{figure}[h]
		\centering
		\includegraphics[scale=0.5]{immagini/riassuntoDistribuzioni}
		\caption{ Relazioni fra distribuzioni }
		\label{fig: riassuntoDistribuzioni}
	\end{figure} 

\subsubsection{Delta o impulso di Dirac}
	
	Abbiamo visto fino adesso funzioni generalizzate nel senso delle distribuzioni, cioè il limite di una serie di funzioni.\\
	Definiamo ora l'impulso di Dirac $ \delta(t)$ (sarebbe $ \delta_0 (t)$) come la derivata di $ \delta_{-1} (t)$ dal punto di vista delle distribuzioni (funzioni generealizzate, cioè limiti di successioni di funzioni).\\
	
	\textbf{OSS:} $ \delta_{-1} (t)$ come fuonzione standard non è continua e quindi non derivabile in $ t=0 $ ma nel senso delle distribuzioni si. Vogliamo vedere quindi $ \delta_{-1} (t)$ come il limite di $ \delta_{ \epsilon} (t)$.\\
	
	\begin{figure}[h]
		\centering
		\includegraphics[scale=0.5]{immagini/deltaEpsilon}
		\caption{ $ \delta_{ \epsilon} (t)$ }
		\label{fig: deltaEpsilon}
	\end{figure}
	
	\begin{equation*}
	\delta_{ \epsilon}(t)=
	\begin{cases} 
	0, \mbox{ se } t < - \epsilon \\ 
	\frac{1}{2 \epsilon}t + \frac{1}{2}, \mbox{ se } - \epsilon \leq t < \epsilon\\
	1, \mbox{ se } t \geq \epsilon
	\end{cases} 
	\end{equation*}
	
	\begin{equation*}
	\delta_{-1}(t)= \lim_{ \epsilon \to 0} \delta_{\epsilon}(t)
	\end{equation*}

	Voglio ora una successione con $ n \in \mathbb{N} $ quindi sostituisco $ \epsilon = \frac{1}{n}$.\\
	Ottengo quindi $ \delta_{-1}(t)= \lim_{ n \to \infty} \delta_{\frac{1}{n}}(t) $ (ci sarà utile più avanti).\\
	
	Derivo ora $\delta_{-1}(t)= \lim_{ \epsilon \to 0} \delta_{\epsilon}(t) $: la parte a sinistra è facile da derivare perchè da prima so che $ \delta (t) = \frac{d}{dt} \delta_{-1}(t) $ mentre la parte a destra la dovremo fare per tratti (vedi com'è $ \delta_{\epsilon}(t) $ e lascia stare il limite).\\
	Ci verrà fuori che:
	\begin{equation*}
	\delta(t)= \lim_{ \epsilon \to 0} \frac{1}{2 \epsilon}  \varPi_{\epsilon}(\frac{t}{2 \epsilon})
	\end{equation*}
	
	\begin{figure}[h]
		\centering
		\includegraphics[scale=0.5]{immagini/rettangoloEpsilon}
		\caption{ $ \frac{1}{2 \epsilon}  \varPi_{\epsilon}(\frac{t}{2 \epsilon}) $ }
		\label{fig: rettangoloEpsilon}
	\end{figure}

	\textbf{OSS:} $ \forall \epsilon$, l'area è $\dfrac{1}{2 \epsilon} 2 \epsilon= 1 $ quindi l'area: è indipendente da $ \epsilon$, è costante e non cambia al variare di $ \epsilon $.\\
	
	\textbf{OSS:} per $ \epsilon \rightarrow 0 $ con $ \epsilon > 0 $, ho che $\dfrac{1}{2 \epsilon} \rightarrow \infty $ quindi l'area è sempre 1 ma l'ampiezza tende a $ + \infty $.\\
	
	\textbf{OSS:} $  \int_{ -\infty}^{ \infty} \delta(t)  \, dt = 1   $.\\
	
	\begin{figure}[h]
		\centering
		\includegraphics[scale=0.5]{immagini/delta}
		\caption{ Impulso di Dirac }
		\label{fig: delta}
	\end{figure}
	
	\textbf{NB:} può essere definita anche come il limite della seguente successione di funzioni:\\
	\begin{equation*}
	\forall n \in \mathbb{N} f_n(t)=
	\begin{cases} 
	\frac{n}{2}, \mbox{ se } -\frac{1}{n}\leq t \leq \frac{1}{n}\\
	0, \mbox{ altrimenti }
	\end{cases} 
	\end{equation*}
	
\pagebreak

	\begin{figure}[h]
		\centering
		\includegraphics[scale=0.5]{immagini/deltaSuccessione}
		\caption{ $ f_n(t) $ }
		\label{fig: deltaSuccessione}
	\end{figure}

	Il limite quindi sarà:\\
	\begin{equation*}
	\lim_{ n \rightarrow \infty} f_n(t) = \delta(t) 
	\end{equation*}

\subsubsection{Proprietà dell'impulso}

	\textbf{1. } $\forall t \in \mathbb{R} \setminus $ \{ $ 0 $ \} cioè per $ t \neq 0 $ ho $ \delta(t)=0 $. \\
	
	\textbf{2. } (con O intendiamo l'origine)
	
	\begin{equation*}
	\int_{ a}^{ b} \delta( \tau)  \, d\tau =
	\begin{cases} 
	1, \mbox{ se } O \in (a...b)\\
	0, \mbox{ se } O \notin (a...b)
	\end{cases} 
	\end{equation*}
	
	\textbf{3. } Ho che $ \delta(t)=\delta(-t) $, quindi $ \delta(t) $ è pari.\\
	
	\textbf{4. Proprietà del campionamento}\\
	Se $ v(t) $ è continua in $ t_0 $ allora $ \forall t \in \mathbb{R}$:
	\begin{equation*}
	v(t) \delta (t-t_0) = v(t_0) \delta (t-t_0)
	\end{equation*}
	
	%TODO: da questo disegno in poi è tutto misterioso
	\begin{figure}[h]
		\centering
		\includegraphics[scale=0.5]{immagini/campionamento}
		\caption{ Proprietà del campionamento }
		\label{fig: campionamento}
	\end{figure}

	\begin{equation*}
	v(t_0) = \int_{ - \infty}^{ \infty} v(t_0) \delta( \tau -t_0)  \, d\tau
	\end{equation*}
	
	\textbf{Conseguenze: }\\
	
	\textbf{1. } Per $ t=0 $ avrò che $ v(t) \delta (t) = v(t_0) \delta (t) $.\\
	
	\textbf{2. } Integriamo $ v(t) \delta (t-t_0) = v(t_0) \delta (t-t_0) $ verrà fuori che:\\
	$ \int_{ - \infty}^{ \infty} v( \tau ) \delta( \tau -t_0)  \, d\tau = \int_{ - \infty}^{ \infty} v( t_0 ) \delta( \tau -t_0)  \, d\tau$\\
	Di questa equazione posso rielaborare la parte destra considerando che $ v(t_0)$ non dipende da $ \tau $ e l'integrale di $ \delta (t-t_0) $ è 1. Quindi viene che:\\
	$ \int_{ - \infty}^{ \infty} v( \tau ) \delta( \tau -t_0)  \, d\tau = v(t_0)$\\
	
	\textbf{ $ \Rightarrow $ Proprietà di riproducibilità dell'impulso }\\
	%TODO: mi sa che qui c'è un errore, guarda bene i tuoi appunti
	Se $ v(t)$ è continua per $ \forall t \in \mathbb{R} $\\
	allora
	\begin{equation*}
	v(t) = \int_{ - \infty}^{ \infty} v( t ) \delta( \tau -t)  \, d \tau
	\end{equation*}
	\textbf{NB:} invece che $ t_0$ ho messo t perchè lo voglio generico avendo v(t) continua.\\
	
	\begin{proof}[Dim]
		Parto con un arteficio (da prendere così come ce l'ha dato il prof): $ v(t)=v(t_0)+( v(t) - v(t_0) )$\\
		Moltiplico da entrambe le parti per $ \delta (t-t_0)$:\\
		$ v(t) \delta (t-t_0) =v(t_0) \delta (t-t_0) +( v(t) - v(t_0) ) \delta (t-t_0) $\\
		Cambio t con $ \tau $ e integro:\\
		$ \int_{ - \infty}^{ \infty} v(\tau) \delta (\tau-t_0) \, d\tau 
		= \int_{ - \infty}^{ \infty} v(t_0) \delta (\tau-t_0) \, d\tau 
		+ \int_{ - \infty}^{ \infty} ( v(\tau) - v(t_0) ) \delta (\tau-t_0)\, d\tau $\\
		Qui devo fare varie considerazioni:\\
		1. Nel secondo integrale $ v(t_0) $ non dipende da $ \tau $ quindi posso portarlo fuori dall'integrale. \\
		2. Sempre nel secondo integrale, l'integrale di $ \delta (\tau-t_0) $ dà 1.\\
		3. Nel terzo integrale:\\
		Se ho $ \tau = t $ allora $ ( v(\tau) - v(t_0) ) = 0 $. \\
		Se ho $ \tau \neq t $ allora $ \delta (\tau-t_0) = 0 $. \\
		Allora il terzo integrale scompare.\\
		Quindi in fine avrò trovato che:\\
		$ \int_{ - \infty}^{ \infty} v(\tau) \delta (\tau-t_0) \, d\tau = v(t_0) $\\
		Questo vale per $ \forall t_0 $ per cui v(t) è continua (rendo quindi $t_0$ generica mettendo al suo posto t).\\ Quindi posso scrivere che:\\
		$ v(t) = \int_{ - \infty}^{ \infty} v(\tau) \delta (\tau-t) \, d\tau$
	\end{proof}

\subsubsection{Impulso centrato in $t_0$ e di area A: $ A \delta (t-t_0)$}

	\begin{figure}[h]
		\centering
		\includegraphics[scale=0.5]{immagini/deltaGenerica}
		\caption{ $ A \delta (t-t_0)$ }
		\label{fig: deltaGenerica}
	\end{figure}

\pagebreak

	\textbf{MATLAB}\\

	\begin{figure}[h]
		\centering
		\includegraphics{immagini/comando3}
		\caption{Il comando syms mi crea una variabile t, diff mi fà la derivata di heaviside cioè il gradino. Come risultato ho dirac cioè proprio l'impulso.  }
		\label{fig: comando3}
	\end{figure}

\section{A tempo discreto}

Un segnale a tempo discreto è così definito (useremo k al posto di t per distinguerli):\\
$ v(k): \mathbb{Z} \rightarrow \mathbb{R} $

\subsubsection{Impulso discreto unitario (impulso di Kronecker)}

	\begin{equation*}
	\delta(k)=
	\begin{cases} 
	1, \mbox{ se } k=0 \\
	0, \mbox{ se } k \neq 0
	\end{cases} 
	\end{equation*}

	\begin{figure}[h]
	\centering
	\includegraphics[scale=0.5]{immagini/impulsoDiscreto}
	\caption{ $ \delta(k)$ }
	\label{fig: impulsoDiscreto}
	\end{figure}

\subsubsection{Gradino discreto}

	\begin{equation*}
	\delta_{-1}(k)=
	\begin{cases} 
	1, \mbox{ se } k \geq 0 \\
	0, \mbox{ se } k < 0
	\end{cases} 
	\end{equation*}

\pagebreak
	
	\begin{figure}[h]
		\centering
		\includegraphics[scale=0.5]{immagini/gradino}
		\caption{ $ \delta_{-1}(k)$ }
		\label{fig: gradino}
	\end{figure}

	\textbf{OSS:}\\
	Se prendo $ k=-1 $, avrò $ \delta_{-1}(-1) = 0 $.\\
	Se prendo $ k=0 $, avrò $ \delta_{-1}(0) = \delta(0) = 1 $.\\
	Se prendo $ k=1 $, avrò $ \delta_{-1}(1) = \delta(0) = 1 $.\\
	
	%TODO: non ho capito
	\begin{equation*}
	\delta_{-1}(k)= \sum_{i=-\infty}^{k} \delta(i)
	\end{equation*}

\subsubsection{Rampa discreta}

	\begin{equation*}
	\delta_{-2}(k)=
	\begin{cases} 
	k, \mbox{ se } k \geq 0 \\
	0, \mbox{ se } k < 0
	\end{cases} 
	\end{equation*}
	
	\begin{figure}[h]
		\centering
		\includegraphics[scale=0.5]{immagini/rampaDiscreta}
		\caption{ $ \delta_{-2}(k)$ }
		\label{fig: rampaDiscreta}
	\end{figure}
	
	\textbf{OSS:}\\
	$ \delta_{-2}(0) = \sum_{i=-\infty}^{-1} \delta_{-1}(i) =0 $\\
	$ \delta_{-2}(1) = \sum_{i=-\infty}^{0} \delta_{-1}(i) = \delta_{-1}(0) $\\
	$ \delta_{-2}(2) = \sum_{i=-\infty}^{1} \delta_{-1}(i) = \delta_{-1}(0) + \delta_{-1}(1) $\\
	
	Si deduce quindi che:\\
	\begin{equation*}
	\delta_{-2}(k)= \sum_{i=-\infty}^{k-1} \delta_{-1}(i)
	\end{equation*}
	
	Ma allora questo significa che:\\
	\begin{equation*}
	\delta_{-2}(k)= \sum_{i=-\infty}^{ \infty} \sum_{j=-\infty}^{k-1} \delta(j)
	\end{equation*}
	
	Se $ k=0 $, allora $ \delta_{-2}(0) =0  $\\
	Se $ k=1 $, allora $ \delta_{-2}(1) = \delta_{-1}(0) = \delta(0) = 1  $\\
	Se $ k=2 $, allora $ \delta_{-2}(2) = \delta_{-1}(0) + \delta_{-1}(1) = \delta(0) + \delta(0) + \delta(1) = 1+1+0=2  $\\
	
	\textbf{Problema del campionamento ma in modo discreto }\\
	
	\begin{figure}[h]
		\centering
		\includegraphics[scale=0.5]{immagini/campionamentoDiscreto}
		\caption{ Segnale a tempo discreto, $\delta(i-3)$ è l'impulso traslato in 3}
		\label{fig: campionamentoDiscreto}
	\end{figure}
	
	Il segnale v(k) ora è discreto (successione, $ k \in \mathbb{Z}$).\\
	Possiamo scrivere il segnale come:\\
	\begin{equation*}
	v(k)= \sum_{i=-\infty}^{ \infty} v(i) \delta(i-k)
	\end{equation*}
	
	In un determinato punto $ k=3$ sarà:\\
	\begin{equation*}
	v(3)= \sum_{i=-\infty}^{ \infty} v(i) \delta(i-3)
	\end{equation*}
	Tutti i termini della sommatoria saranno 0 a parte in 3.\\
	$ ...+0+0+0+ v(3) \delta(3-3)+0+0+0+...= v(3) \delta(0) = v(3)1 = v(3)$

\subsubsection{Successioni esponenziali}

	L'analogo segnale discreto è:
	\begin{equation*}
	v(k)= Ae^{j \phi} \lambda^k
	\end{equation*}
	con $ k \in \mathbb{Z} $, l'ampiezza $ A \in \mathbb{R}_+ $, $ \phi \in \mathbb{R} $ e $ \lambda \in \mathbb{C}$.\\
	
	Ricordiamo che $ \lambda $ essendo un numero complesso posso scriverlo anche come:\\
	 $ \rho ( cos\theta + j sen\theta) = \rho e^{j \theta} $\\
	Il segnale quindi diventerà:\\
	$ v(k)= Ae^{j \theta} \lambda^k = Ae^{j \theta} \rho^{k} e^{j \theta k} = Ae^{j \theta} e^{ (ln(\rho)+j\theta) k} $\\
	\textbf{NB:} nell'ultimo passaggio ho usato $ \rho = e^{ln(\rho)}$.\\
	\textbf{OSS:} v(k) può essere visto come la versione campionata (con periodo di campionamento unitario) del segnale esponenziale continuo.\\
	$ v(t) = Ae^{j \theta} e^{ ut} $ dove $ u=ln( \rho)+j \theta $.\\
	

\subsubsection{Successioni sinusoidali}

	L'analogo segnale discreto è:
	\begin{equation*}
	v(k)= Acos( \theta k + \phi)
	\end{equation*}
	con l'ampiezza $ A>0 $, $ \theta $ è la pulsazione e $ \phi$ la fase.\\
	
	v(k) è periodico se e solo se $ \theta = \frac{2 \pi n}{N}$, dove $n \in \mathbb{N} $, cioè è un multiplo razionale di $ 2 \pi $.\\
	$ N>0 $ è i periodo.\\
	

\subsubsection{Successioni sinusoidali modulate esponenzialmente}

	L'analogo segnale discreto è:
	\begin{equation*}
	v(k)= A \rho^k cos( \theta k + \phi)
	\end{equation*}
	con $ A>0 $, $ \rho>0 $, $ \theta$ e $\phi \in \mathbb{R}$.\\
	
	v(k) non è periodico perchè ho un esponenziale, si dice che v(k) è pseudo-periodico perchè devo comunque guardare la frequenza del campionamento.











\documentclass[10pt,a4paper]{article}
\usepackage[utf8]{inputenc}
\usepackage[italian]{babel}
\usepackage{amsmath}
\usepackage{amsfonts}
\usepackage{amssymb}
\usepackage{graphicx}
\begin{document}
	\section{Sistemi}
	\paragraph{Sistemi a tempo continuo}
	\subparagraph{Proprietà dei sistemi a tempo continuo}
\end{document}
\chapter{Trasformata Di Laplace}
%\section{Ivan_Love_You}
%\documentclass[a4paper]{report}
%\usepackage[T1]{fontenc}
%\usepackage[utf8]{inputenc}
%\usepackage[italian]{babel}
%\usepackage{mathrsfs}
%\usepackage{amsthm}
%\usepackage{amsmath}
%\usepackage{amsfonts}
%\usepackage{cancel}
%ivan è bravo <3 LODI



%\begin{document}
%\chapter{La trasformata di Laplace [TdL]} % TODO: chapter

\begin{definizione}
   Se $v:\mathbb{R}\to\mathbb{C}$ è localmente sommabile in $[0, +\infty)$ oppure \\$\Bigl(\displaystyle\int_a^bv(t)\,dt < +\infty\quad\forall a,b \in[0, +\infty)\Bigr)$ si definisce la TdL unilatera del segnale $v(t)$\\
   $\displaystyle\LA[v(t)](s) := V(S) = \intZeroInfinity v(t)e^{-st}\,dt$ , dove $s = \sigma + jw$ è una variabile complessa
   $V$ è definita per quei valori di $s$ per cui l'integrale è ben definito\\ $\displaystyle\Leftrightarrow \intZeroInfinity v(t)e^{-st}\,dt < +\infty$\\
   Tale regione del piano complesso è chiamata \emph{regione di convergenza (RdC)}\\
   Si può dimostrare che la RdC è un semipiano aperto del tipo \\$\mathrm{RdC} = \{s\in\mathbb{C}/\Re(s)>\alpha\}$ dove $\alpha\in\mathbb{R}$ ascissa di convoluzione della TdL
   %TODO: Figura
   \begin{proof}[Dim]
      Dimostriamo per combinazioni lineari di funzioni esponenziali
      \[
         \begin{split}
            & v(t) = \sum_{i = 1}^nc_ie^{\lambda it} \qquad \lambda_i = \sigma_i + jw_i\in\mathbb{C}, i = \overline{1,n}\\
            & V(s) = \intZeroInfinity \sum_{i = 1}^nc_ie^{\lambda it}e^{-st}\,dt = \sum_{i = 1}^nc_i\intZeroInfinity e^{\lambda it}e^{-st}\,dt\\
            & \mathrm{Dimostriamo\, che\, } \Bigg\lvert\intZeroInfinity e^{\lambda_it}e^{-st}\,dt\Bigg\rvert < +\infty\quad\forall i \\
            & \intZeroInfinity e^{\sigma_it}\cdot \underbrace{e^{jw_it}}_{\lvert\,\,\rvert = 1}\cdot e^{-\sigma t}\cdot \underbrace{e^{-jwt}}_{\lvert\,\,\rvert = 1}\,dt = \intZeroInfinity e^{(\sigma_i - \sigma)t} = \frac{e^{(\sigma_i - \sigma)t}}{\sigma_i - \sigma}\Big|_{0^-}^{+\infty} =\\
            & = \underbrace{\lim_{t\to+\infty}\frac{e^{(\sigma_i - \sigma)t}}{\sigma_i - \sigma}}_{\mathrm{converge\, per\, \sigma_i - \sigma < 0 \,\Leftrightarrow\, \sigma > \sigma_i}} - \frac{1}{\sigma_i - \sigma}
         \end{split}
      \]
      Scegliendo $\alpha = sup\{\Re(\lambda_i) : i = \overline{1,n}\} \Leftarrow$ TdL converge $\displaystyle\forall v(t) = \sum_{i = 1}^nc_ie^{\lambda_it}$ \AdC
      \begin{osservazione}
         \begin{enumerate}
            \item I sistemi stabili hanno $\displaystyle \alpha < 0(\Leftrightarrow \Re(\lambda_i) < \alpha < 0\quad \forall i = \overline{1,n}) \Rightarrow jw\in\mathrm{\,RdC\,}\forall w\in\mathbb{R}$
            \item $v(t) (\mathrm{time}) \leftrightarrow V(s)(\mathrm{complex})$
         \end{enumerate}
      \end{osservazione}
   \end{proof}
\end{definizione}

\section{Proprietà della trasformata di Laplace}
\subsection{Linearità}
\[
   \begin{split}
      \LA[a_1v_1(t) + a_2v_2(t)] = a_1\LA[v_1(t)] + a_2\LA[v_2(t)]\\
      \mathrm{RDC} = {s\in\mathbb{C} / \Re(s)>\alpha}\mathrm{, dove}\,\alpha\ge\{\alpha_1, \alpha_2\}
   \end{split}
\]
%------------------------------------------------------------------------------------------------------------------------------------------------------------------------------
\subsection{Time shifting (Ritardo temporale)}
Se $v(t)$ ammette TdL allora, $v(t - \tau)$ ammette TdL e $\LA[v(t) - \tau] = e^{-s\tau}V(s)$\, $\tau>0$\\
L'\AdC di $v(t - \tau)$ è la stessa di $v(t)$
\begin{proof}[Dim]
   \[
      \begin{split}
         \LA[v(t) - \tau] & = \intZeroInfinity v(t - \tau)e^{-st}\,dt = \int_{0^-}^\tau v(t - \tau)e^{-st}\,dt\,+\,\int_\tau^{+\infty}v(t - \tau)e^{-st}\,dt =\\
         & = \int_\tau^{+\infty}v(t - \tau)e^{-st}\,dt = \intZeroInfinity v(e^{-s(x + \tau)}\,dx = \intZeroInfinity v(t)e^{-s\tau}e^{-st}\,dt =\\
         & x = t - \tau \Rightarrow dt = dx \\
         & poi \\ % TODO: inserire tabella
         & x = t \Rightarrow dx = dt\\
         & = e^{-s\tau}\overbrace{\int_\tau^{+\infty}v(t)e^{-st}\,dt}^{V(s)} = e^{-s\tau}V(s)
      \end{split}
   \]
\end{proof}
%------------------------------------------------------------------------------------------------------------------------------------------------------------------------------
\subsection{Moltiplicazione per una funzione esponenziale (Frequency shifting)}
   $\LA[e^{\lambda t}v(t)] = V(s - \tau)$\\
   $\alpha_2 = \alpha + \Re(\lambda)$, dove $\lambda$ è AdC di $v(t)$\\
   $\alpha_2 = $ \AdC di $e^{\lambda t}v(t)$
\begin{proof}[Dim]
   \[
      \intZeroInfinity e^{\lambda t}v(t)e^{-st}\,dt = \intZeroInfinity v(t)e^{-(s - \lambda)t}\,dt = V(s - \lambda)
   \]
\end{proof}
%------------------------------------------------------------------------------------------------------------------------------------------------------------------------------
\subsection{Cambiamento di scala}

$\LA[v(rt)] = \frac{1}{r}V\Bigl(\frac{s}{r}\Bigr)$\\
$\alpha_2 = r\alpha$ ($\alpha$ \AdC di $v(t)$)

\begin{proof}[Dim]
   \[
      \begin{split}
         &\intZeroInfinity v(rt)e^{-st}\,dt = \intZeroInfinity \frac{1}{r}v(x)e^{-\frac{s}{t}}\,dx = \frac{1}{r}V\Bigl(\frac{s}{r}\Bigr)\\
         & rt = x\\
         & t = \frac{x}{r}\,dx = \frac{1}{r}\,dt %TODO: tabella
      \end{split}
   \]
\end{proof}
%------------------------------------------------------------------------------------------------------------------------------------------------------------------------------
\subsection{Proprità della derivata}
Se $v(t)$ ammette TdL ed esiste ed è finito $v(0^-) = \lim_{t\to 0} \Rightarrow \frac{dv(t)}{dt}$ ammette TdL e
\[
   \begin{split}
      & \LA\Bigl[\frac{dv(t)}{dt}\Bigr] = s\LA[v(t)] - v(0^-)\\
      & \alpha_2 = \LA\Bigl[\frac{dv(t)}{dt}\Bigr]\\
      & \alpha = \LA[v(t)]
   \end{split}
\]
L'\AdC $(\alpha_2 \le \alpha)$

\begin{proof}[Dim]
   \[
      \begin{split}
         \LA\Bigl[\frac{dv(t)}{dt}\Bigr] & = \intZeroInfinity \frac{dv(t)}{dt}e^{-st}\,dt = v(t) - e^{-st}\Big|_{0^-}^{+\infty} - (-s)\underbrace{\intZeroInfinity v(t)e^{-st}\,dt}_{V(s)} =\\
         & = \underbrace{\lim_{t\to +\infty}[v(t)e^{-st}|_{0^-}^{t}]}_{-v(0^-)} + sV(s) = -v(0^-) + sV(s)\\
      \end{split}
   \]
\end{proof}
%------------------------------------------------------------------------------------------------------------------------------------------------------------------------------
\subsection*{5.1$\quad$ Proprità della derivata seconda ed n-esima}
\[
   \LA\Bigl[\frac{d^2v(t)}{dt^2}\Bigr] = s^2\LA[v(t)] - sv(0^-) - \frac{dv(0^-)}{dt}
\]
\begin{proof}[Dim]
   \[
      \begin{split}
         \LA\Bigl[\frac{d}{dt}\Bigl[\frac{dv(t)}{dt}\Bigr]\Bigr]\overset{(5)}{=} & s\overbrace{\LA\Bigl[\frac{dv(t)}{dt}\Bigr]}^{sV(s) - v(0^-)} - \frac{dv(0^-)}{dt} = s(s\LA[V(t)] - v(0^-)) - \frac{dv(0^-)}{dt} = \\
         = & s^2V(s) - sv(0^-) - \frac{dv(0^-)}{dt}
      \end{split}
   \]
   In generale,
   \[
      \LA\Bigl[\frac{d^iv(t)}{dt^i}\Bigr] = s^i\LA[v(t)] - \sum_{k = 0}^{i - 1}\frac{d^kv(t)}{dt^k}\Big|_{t = 0^-}s^{i - 1 - k}
   \]
\end{proof}
%------------------------------------------------------------------------------------------------------------------------------------------------------------------------------
\subsection{Moltiplicazione per una funzione polinomiale}
\begin{proof}[Dim]
\[
   \LA[tv(t)] = - \frac{dV(s)}{ds}
\]
Con la stessa \RdC
   \[
      \begin{split}
         \frac{dV(s)}{ds} & = \frac{d}{ds}\Bigl[\intZeroInfinity v(t)e^{-st}dt\Bigr] = \intZeroInfinity \frac{\delta e^{-st}}{\delta s}v(t)dt = \intZeroInfinity -te^{-st}v(t) dt =\\
         & = \intZeroInfinity[-tv(t)]e^{-st}dt = - \LA[tv(t)]
      \end{split}
   \]
   In generale,
   \[
      \LA[t^iv(t)] = (-1)^i\frac{d^iV(s)}{ds^i}
   \]
\end{proof}
%------------------------------------------------------------------------------------------------------------------------------------------------------------------------------
\subsection{Integrale nel dominio del tempo}
Se $v(t)$ ha la TdL $V(s)$ per $\Re(s) > \alpha \Longrightarrow \int_{0^-}^tv(\tau)d\tau$ ha TdL $\alpha_1 = max\{0, \alpha\}$ e $\LA[\int_{0^-}^tv(\tau)d\tau] = \frac{V(s)}{s}$
\begin{proof}[Dim]
   \[
      \begin{split}
         & v_1(t) = \int_{0^-}^tv(\tau)d\tau \Rightarrow v_1'(t) = v(t)\\
         & v_1(0^-) = 0\\
         & \LA[v(t)] = \LA\Bigl[\frac{dv_1(t)}{dt}\Bigr] = s\LA[v_1(t)] - \overbrace{v_1(0^-)}^{0} = \LA\Bigl[\int_{0^-}^tv(\tau)d\tau\Bigr] = \frac{V(s)}{s}
      \end{split}
   \]
\end{proof}
%------------------------------------------------------------------------------------------------------------------------------------------------------------------------------
\subsection{Integrale nel dominio complesso $\mathbb{C}$}
Se esiste $\lim_{t\to 0^-}\frac{v(t)}{t}$ allora,
\[
   \LA\Bigl[\frac{v(t)}{t}\Bigr] = \int_s^{+\infty}V(s)\,ds
\]
\begin{proof}[Dim]
   \[
      \begin{split}
         & V(s) = \intZeroInfinity v(t)e^{-st}\,dt= \int_s^{+\infty}V(s)\,ds = \int_s^{+\infty}\Bigl[\intZeroInfinity v(t)e^{-st}\,dt\Bigr]\,ds = \\
         & = \int_s^{+\infty}v(t)\Bigl[\intZeroInfinity e^{-st}\,dt\Bigr]\,ds = \intZeroInfinity v(t)\Bigl[-\frac{1}{t}e^{-st}\Big|_s^{+\infty}\Bigr]\,dt = \\ %TODO: controlla veridicità
         & = \intZeroInfinity v(t)\frac{e^{-st}}{t}\,dt = \LA\Bigl[\frac{v(t)}{t}\Bigr]
      \end{split}
   \]
\end{proof}
%------------------------------------------------------------------------------------------------------------------------------------------------------------------------------
\subsection{Teorema del valore iniziale}
Se $v(t)$ ha Tdl, se $\exists \lim_{s\to \infty}v(t)$, finito $\Rightarrow \lim_{t\to 0^-}v(t) = \lim_{s\to \infty}sV(s)$
\begin{proof}[Dim]
   \[
      \begin{split}
         & \LA\Bigl[\frac{dv(t)}{dt}\Bigr] = sV(s) - v(0^-)\qquad [5]\\
         & \lim_{s\to \infty}[sV(s) - v(0^-)] = \lim_{s\to \infty}\intZeroInfinity\frac{dv(t)}{dt}e^{-st}\,dt = \lim_{s\to \infty} \Biggl(\lim_{\substack{T\to \infty \\ \varepsilon\to 0^-}}\int_\varepsilon^T\frac{dv(t)}{dt}e^{-st}\,dt\Biggr) \\
         & = \lim_{\substack{T\to \infty \\ \varepsilon\to 0^-}}\Biggl[\int_\varepsilon^T\frac{dv(t)}{dt}\Bigl(\underbrace{\lim_{s\to \infty}e^{-st}\,dt}_0\Bigr)\Biggr] = 0 \Rightarrow \lim_{s\to \infty}[sV(s) - v(0^-)] = 0 \\
         & \lim_{s\to \infty}sV(s) = v(0^-)
      \end{split}
   \]
\end{proof}
%------------------------------------------------------------------------------------------------------------------------------------------------------------------------------
\subsection{Teorema del valore finale}
Se $v(t)$ ha Tdl, e $\lim_{s\to \infty}v(t)$ esiste ed è finito $\Rightarrow \lim_{t\to \infty}v(t) = \lim_{s\to 0^+}sV(s)$
\begin{proof}[Dim]
   \[
      \begin{split}
         & \LA\Bigl[\frac{dv(t)}{dt}\Bigr] = sV(s) - v(0^-)\qquad [5]\\
         & \lim_{s\to 0^+}[sV(s) - v(0^-)] = \lim_{s\to 0^+}\intZeroInfinity\frac{dv(t)}{dt}e^{-st}\,dt = \lim_{\substack{T\to \infty \\ \varepsilon\to 0^-}}\Biggl[\int_\varepsilon^T\frac{dv(t)}{dt}\Bigl(\underbrace{\lim_{s\to 0^+}e^{-st}\,dt}_1\Bigr)\Biggr]\\
         & = \lim_{\substack{T\to \infty \\ \varepsilon\to 0^-}} \Bigl[[v(t)|_\varepsilon^T\Bigr] = \lim_{\substack{T\to \infty \\ \varepsilon\to 0^-}}[v(t) - v(\varepsilon)] = \lim_{T\to\infty}v(T) - v(0^-)\\
         & \lim_{s\to 0^+}sV(s) - \cancel{v(0^-)} = \lim_{T\to\infty}v(T) - \cancel{v(0^-)}
      \end{split}
   \]
\end{proof}
%------------------------------------------------------------------------------------------------------------------------------------------------------------------------------
\subsection{Convoluzione nel dominio del tempo (Prodotto nelle frequenze)}
Se $v_1(t)$, $v_2(t)$ sono nulle per $t < 0$ e hanno TdL $V_1(s)$, $V_2(s)$ $\Rightarrow [v_1 * v_2](t)$ ha TdL:\\
\[
   \underbrace{\LA[(v_1 * v_2)(t)]}_{Convoluzione\, nel\, tempo} = \underbrace{V_1(s)\cdot V_2(s)}_{Moltiplicazione\, nelle\, frequenze}
\]
\begin{proof}[Dim]
   \[
      \begin{split}
         &\LA[(v_1 * v_2)(t)] =\\
         & = \LA\Biggl[\intZeroInfinity \underbrace{v_1(\tau)}_{=\, 0\, per\, t < 0}v_2(t - \tau)\,d\tau\Biggr] = \intZeroInfinity\Biggl[\intZeroInfinity v_1(\tau)v_2(t - \tau)\,d\tau\Biggr]e^{-st}\,dt\\
         & = \intZeroInfinity v_2(\tau)\Biggl[\intZeroInfinity v_2(t - \tau)e^{-st}\,dt\Biggr]\,d\tau \rightarrow \intZeroInfinity v_1(\tau)\Biggl[\intZeroInfinity v_2(\lambda)e^{-s(\lambda + \tau)}\,d\lambda\Biggr]\,d\tau \\ 
         & t - \tau = \lambda \\
         & t = \lambda + \tau \\
         & dt = d\lambda \\
         & = \intZeroInfinity v_1(\tau)e^{-st}\Biggl[\underbrace{\intZeroInfinity v_2(\lambda)e^{-s\lambda}\,d\lambda}_{\LA[v_2(t)]}\Biggr] = \LA[v_2(t)]\intZeroInfinity v(\tau)e^{-st}\,d\tau = V_2(s)\cdot V_1(s)
      \end{split}
   \]
\end{proof}

%===================================================================================================================================================================================

\section{Esempi di trasformazioni notevoli}
   \begin{enumerate}
      \item[a.] Impulso ideale unitario $\delta(t)$\\
         $\displaystyle\LA[\delta(t)] = \intZeroInfinity\delta(t)e^{-st}\,dt \overset{\mathrm{Campionamento\,per\,} v(t) = e^{-st}}{=} e^{s\cdot 0} = 1$\\
         $V(s) = 1$ RdC = $\mathbb{C}$
      \item[b.] Gradino unitario $\gradino$\\
         $\displaystyle\LA[\gradino(t)] = \intZeroInfinity\underbrace{\gradino(t)}_{1}e^{-st}\,dt = \intZeroInfinity e^{-st} = $\\
         $\displaystyle = -\frac{e^{-st}}{s}\Big|_{0^-}^{+\infty} = 0 - \Bigl(-\frac{1}{s}\Bigr) = \frac{1}{s}$
      \item[c.] Impulso unitario\\
         $\LA[\delta(t - t_0)] = e^{-st_0}\LA[\overbrace{\delta(t)}^{1}] = e^{-st_0}$
      \item[d.] Esponenziale causale $v(t) = e^{\lambda t}\gradino(t), \lambda\in\mathbb{C}$\\
         $\displaystyle\LA[e^{\lambda t}\gradino(t)] \overset{(3)}{=} V(s - \lambda) = \frac{1}{s -\lambda}$\\
         RdC = $\{s\in\mathbb{C} : \Re(s) > \Re(\lambda)\}$
      \item[e.] Esponenziale causale moltiplicata per una funzione polinomiale\\
         \begin{center}
            $\displaystyle v(t) = \frac{t^\ell}{\ell!}e^{\lambda t}\gradino(t)$
         \end{center}
         \[
            \begin{split}
               &\LA\Bigl[\frac{t^\ell}{\ell!}e^{\lambda t}\gradino(t)\Bigr] \overset{(1)}{=} \frac{1}{\ell}\LA[t^\ell \overbrace{e^{\lambda t}\gradino(t)}^{v(t)}] \overset{(6)}{=} \frac{(-1)^\ell}{\ell!}\cdot\frac{d^\ell}{ds^\ell}\LA[e^{\lambda t}\gradino(t)] =\\
               &\overset{(d)}{=} \frac{(-1)^\ell}{\ell!}\cdot\frac{d^\ell}{ds^\ell}\Bigl[\frac{1}{s - \lambda}\Bigr] = \frac{(-1)^\ell}{\cancel{\ell!}}\cdot\cancel{\ell!}\frac{1}{(s - \lambda)^{\ell + 1}} = \frac{1}{(s - \lambda)^{\ell + 1}}
            \end{split}
         \]
         Esempio
         \[
            \begin{split}
               \LA[te^{\lambda t}\gradino(t)] = \frac{1}{(s - \lambda)^{2}}\\
               \LA\Bigl[\frac{t^2}{2!}e^{\lambda t}\gradino(t)\Bigr] = \frac{1}{(s - \lambda)^{3}}
            \end{split}
         \]
         Per $\lambda = 0$
         \[
            \begin{split}
               \LA\Bigl[\frac{t^\ell}{\ell!}\gradino(t)\Bigr] = \frac{1}{s^{\ell + 1}}\\
               \LA[t^\ell\gradino(t)] = \frac{\ell!}{s^{\ell + 1}}
            \end{split}
         \]
   \end{enumerate}
   \section{Sistemi LTI causali analisi dominio complesso o nel dominio delle frequenze}
   \begin{center}
      $\displaystyle a_n\der{v(t)}{n} + \dots + a_0v(t) = b_m\der{u(t)}{m} + \dots + b_0u(t)\qquad(1)$
   \end{center}
      $a_n, b_m \neq 0 \quad n\ge m \quad u(t)$ ingresso nullo per $t < 0\quad (u(t) = u(t)\gradino)$\\
      Condizioni iniziali $\displaystyle= v(0^-), \der{v(0^-)}, \dots, \derN{v(0^-)}{n-1}$\\
      Se $u(t)$ ha TdL $v(t)$ ammette TdL $\bigr(v(t)$ ristretto per $t\ge 0\bigl)$\\
      $U(s) = \LA[u(t)]\qquad e\qquad V(s) = \LA[v(t)]$\\
      (Proprietà 5) $\displaystyle\qquad \LA\Bigr[\derN{v(t)}{i}\Bigl] = s^iV(s) - \sum_{k = 0}^{i - 1}\derN{v(t)}{k}\Big|_{t = 0^-}s^{i - 1 - k}, i = \overline{1,n}$\\
      Applichiamo la TdL a (1), ma solo all'uscita
      \[
         \begin{split}
            &\displaystyle a_n\Biggr[s^nV(s) - \sum_{k = 0}^{n - 1}\derN{v(t)}{k}\Big|_{t = 0^-}s^{n - 1 - k}\Biggl] + a_{n - 1}\Biggr[s^nV(s) - \sum_{k = 0}^{n - 2}\derN{v(t)}{k}\Big|_{t = 0^-}s^{n - 2 - k}\Biggl] + \dots + a_0V(s) =\\
            & = b_ms^mU(s) + b_{m - 1}s^{m - 1}U(s) + \dots + b_0U(s)\\\\
            & (\overbrace{a_ns^n + a_{n - 1}s^{n - 1} + \dots + a_0}^{d(s) \mathrm{\,\,polinomio\,\,di\,\,grado\,\,}n})V(s) -\\
            & \overbrace{a_nv(0^-)s^{n-1} - \Bigl[a_{n-1}v(0^-) + a_n\der{v(t)}\Big|_{t = 0^-}\Bigr]s^{n - 2} - \dots + \Biggl[\sum_{k = 0}^{n - 1}a_{k + 1}\derN{v(t)}{k}\Big|_{t = 0^-}\Biggr]}^{p(s)\mathrm{\,\,polinomio\,\,di\,\,grado\,\,}n-1} = \\
            & = (\overbrace{b_ms^m + b_{m - 1}s^{m - 1} + \dots + b_0}^{n(s) \mathrm{\,\,polinomio\,\,di\,\,grado\,\,}m})U(s)\\
            & \Rightarrow d(s)V(s) - p(s) = n(s)U(s)\qquad\Big|_{\mathrm{Divido\,\,per\,\,}d(s)}\\
            & \Rightarrow V(s) = \frac{p(s)}{d(s)}+ \frac{n(s)}{d(s)}U(s)
         \end{split}
      \]
      \begin{osservazione}
         \begin{enumerate}
            \item $\displaystyle\frac{p(s)}{d(s)}$ dipende soltanto dalle condizioni iniziali su $V$ e dal polinomio caratteristico $\Rightarrow$ rappresento la TdL della risposta libera
               \[V_\ell(s) = \frac{p(s)}{d(s)}\]
            \item $\displaystyle\frac{n(s)}{d(s)}U(s)$ dipende soltanto dal sistema e dall'ingresso $\Rightarrow$ TdL della risposta forzata
               \[V_f(s) = \frac{n(s)}{d(s)}U(s)\]
               \[V(s) = \frac{p(s)}{d(s)} + \frac{n(s)}{d(s)}U(s)\]
            \item Sappiamo che $v_f(t) = [h * u](t) \xRightarrow{(11)} V_f(s) = H(s)U(s) \Rightarrow H(s) = \frac{n(s)}{d(s)}$ è la TdL di $h(t)$ (risposta impulsiva)\\
            (Funzione di trasferimento del sistema)
            \[H(s) = \frac{b_ms^m + b_{m - 1}s^{m - 1} + \dots + b_0}{a_ns^n + a_{n - 1}s^{n - 1} + \dots + a_0}\]
            (Funzione razionale in $s \in \mathbb{C}$)
         \end{enumerate}
      \end{osservazione}
      \emph{Esempio} $\qquad\displaystyle\derN{v(t)}{3} + \derN{v(t)}{2} = \der{u(t)}$
      \[
         \begin{split}
            &\mathrm{TdL} \Rightarrow s^3V(s) - s^2v(0^-) - s\dot{v}(0^-) - \ddot{v}(0^-) + s^2V(s) - sv(0^-) - \dot{v}(0^-) = U(s) \rightarrow \mathrm{\,\,Niente\,\,condizioni\,\,iniziali}\\
            &d(s) = s^3 + s^2\\
            &p(s) = s^2v(0^-) + [\dot{v}(0^-) + v(0^-)]s + \ddot{v}(0^-) + \dot{v}(0^-)\\
            &n(s) = s\\
            &H(s) = \frac{s}{s^3 + s^2} = \frac{1}{s^2 + s}\\
            &H(s) = \LA[h(t)]\\
            &h(t) = d_0\delta(t) + \sum_{i = 0}^{r}\sum_{\ell = 0}^{\mu - 1}\cdot d_{i,\ell}\cdot e^{\lambda_it}\cdot\frac{t^\ell}{\ell}\cdot\gradino(t)\xRightarrow{TdL} H(s) = d_0\delta(t) + \sum_{i = 0}^{r}\sum_{\ell = 0}^{\mu - 1}d_{i, \ell}\cdot\frac{1}{(s - \lambda_i)^{\ell + 1}}
            \end{split}
      \]
      $d_0$ compare solo se $n = m$\\
      $\lambda_i$ sono le radici di $\displaystyle\sum_{i = 0}^{n}a_is^i = 0\qquad$ (equazione caratteristica)\\
      \[
         \displaystyle\xRightarrow[\mathrm{fondamentale}]{\mathrm{Teorema}}\sum_{i = 0}^{n}a_is^i = a_n(s - \lambda_1)^{\mu_1}\cdot...\cdot(s - \lambda_r)^{\mu_r} \Rightarrow H(s) = \frac{\overline{n}(s)}{(s - \lambda_1)^{\mu_1}\cdot...\cdot(s - \lambda_r)^{\mu_r}}, \mathrm{\,\,dove\,\,}\overline{n}(s) = \frac{n(s)}{a_n}
      \]
      Posso anche scrivere
      \[
         \frac{(d(s)) \rightarrow b_m(s - p_1)\cdot...\cdot(s - p_m)}{(n(s)) \rightarrow a_n(s - z_1)\cdot...\cdot(s - z_n)}
      \]
      $z_i, i = \overline{1, m}$ zeri della TdL\\
      $p_i, i = \overline{1, n}$ poli della TdL\\
      Possiamo ridefinire la molteplicità:\\
      $\alpha\in\mathbb{C}$ un polo di molteplicità $k\in\mathbb{N}$ se $\displaystyle\lim_{s\to\infty}(s - \alpha)^{k - 1}H(s) = \infty\quad\lim_{s\to\infty}(s - \alpha)^kH(s) < +\infty$\\
      $\beta\in\mathbb{C}$ uno zero di molteplicità $k\in\mathbb{N}$ se $\displaystyle\lim_{s\to\beta}\frac{1}{(s - \beta)^{k - 1}}H(s) = 0\quad\lim_{s\to\beta}\frac{1}{(s - \beta)^k}H(s) \neq 0$
      \begin{center}
         \begin{tabular}{lll}
            \toprule
            \_ & Poli & Zeri \\
            \midrule
            $\lambda_i$ & $\alpha$ & $\beta$ \\
            $\mu_i$ & k & k \\
            \bottomrule
         \end{tabular}
      \end{center}
      \begin{osservazione}
         $z_i\in\{p_1,\dots,p_n\} \Rightarrow \frac{n(s)}{d(s)}$ è riucibile\\
         $\Rightarrow \{\mathrm{zeri\,\,}H(s)\} \subseteq \{\mathrm{zeri\,\,di\,\,}n(s)\}$\\
         $\Rightarrow \{\mathrm{poli\,\,}H(s)\} \subseteq \{\mathrm{zeri\,\,di\,\,}d(s)\}$
      \end{osservazione}
      \emph{Proprietà}: Il sistema è BIBO stabile se e solo se tutti i poli hanno parte reale minore di 0 $\forall i, \Re(p_i) < 0$\\
      \NB: I poli di $H(s)$ sono zeri di $d(s)$
%\end{document}

\chapter{Trasformata di Fourier}

	% Giorno: 17/4
\section{La serie di Fourier e la trasformata di Fourier}

	\subsubsection{La serie di Fourier}
	\textbf{Wikipedia:}\\
	La serie di Fourier è una rappresentazione di una \textbf{funzione periodica} mediante una combinazione lineare di funzioni sinusoidali.\\
	\textbf{Prof:}\\
	I sistemi LTI e BIBO stabili trasformano i fasori (esponenziali con esponente immaginario puro) in fasori con la stessa frequenza ma cambiando ampiezza e fase in base alla \textbf{funzione di trasferimento (fdT)}.
	La fdT è $ H(s) |_{s=j \omega}$ valutata in $ j \omega $ oppure in $ j 2 \pi f$.\\
	\subsubsection{La trasformata di Fourier}
	\textbf{Wikipedia:}\\
	La trasformata di Fourier è uno degli strumenti matematici maggiormente utilizzati nell'ambito delle scienze pure e applicate. Essa permette di scrivere una \textbf{funzione dipendente dal tempo} nel dominio delle frequenze, e per fare ciò decompone la funzione nella base delle funzioni esponenziali con un prodotto scalare. Questa rappresentazione viene chiamata spesso \textbf{spettro della funzione}.\\
	La trasformata è invertibile: a partire dalla trasformata di una funzione ${\hat x}$ è possibile risalire alla funzione $x$ tramite il teorema di inversione di Fourier.\\
	Grazie alla trasformata di Fourier è possibile individuare un criterio per compiere un \textbf{campionamento} in grado di digitalizzare un segnale senza ridurne il contenuto informativo: ciò è alla base dell'intera teoria dell'informazione che si avvale, inoltre, della trasformata di Fourier (in particolare della sua variante discreta) per l'elaborazione di segnali numerici.\\
	La trasformata di Fourier $ \FO[ x(t)] (f) $ di una funzione $ x ( t ) $ è equivalente al valutare la trasformata di Laplace bilatera $ \LA $ di $x $ ponendo $ s=j\omega $, e tale definizione è valida se e solo se la regione di convergenza della trasformata di Laplace contiene l'asse immaginario.\\
	\textbf{Prof:}\\
	La serie di Fourier ci permette di rappresentare qualsiasi segnale continuo come serie (o integrale) di fasori.\\
	Useremo f al posto di $ \omega (\omega = 2 \pi f)$, i fasori quindi saranno $ e^{j2\pi ft}$ con $ t \in \mathbb{R}$.\\
	$ H(j \omega) = H(2\pi f) := H(f) $ sarà la nostra fdT.\\
	Dividiamo quindi i segnali in:\\
	1) \textbf{periodici} $\Rightarrow$ possiamo usare la \textbf{serie di Fourier}.\\
	2) \textbf{non periodici} $\Rightarrow$ possiamo usare la \textbf{trasformata di Fourier}.\\
	Domande:\\
	1) La somma dei segnali seriodici è anch'essa periodica?\\
	2) I segnali periodici come si possono scrivere come somma di fasori?\\
	3) Possiamo rappresentare i segnali non periodici con funzioni elementari periodiche (cioè con i fasori)?\\
	
	Iniziamo con rispondere alla prima domanda usando un'esempio.\\
	\textbf{ES1: la somma è periodica?}\\
	$ v(t) 
	= 2 \cos ( 400 \pi t + \phi_1) + 3 \cos ( 600 \pi t + \phi_2)$ 
	con $ t \in \mathbb{R}$.\\
	Abbiamo che $ \omega_1 = 400 \pi $ e $ \omega_2 = 600 \pi $.\\
	Ricordiamo che $ \omega = 2 \pi f = \frac{2 \pi }{T}$, quindi $ T_1 = \frac{1}{200}$ e $ T_2 = \frac{1}{300} $.\\
	\textbf{Il segnale $ v(t)$ è periodico se il rapporto $ \frac{T_1}{T_2}$ è un numero razionale diverso da 1.}\\
	In questo caso, $ \frac{T_1}{T_2} = \frac{ 300 }{ 200 } = \frac{3}{2}$.\\
	Ora che sappiamo che è periodica vogliamo trovare il periodo di $ v(t)$.\\
	\textbf{Il periodo di una somma di segnali periodici è il minimo comune multiplo dei periodi dei singoli segnali}.\\
	In questo caso,
	$
		T
		= mcm(\frac{1}{200},\frac{1}{300})
		= mcm(\frac{3}{600},\frac{2}{600})
		= \frac{1}{600} mcm(3,2)
		= \frac{6}{600}
		= \frac{1}{100}
	$.\\
	\textbf{ES2: la somma è periodica?}\\
	$ v(t) 
	= 2 \sin ( \sqrt{2} t + \phi_1) + 3 \cos ( 2 t + \phi_2)$.\\
	Abbiamo che $ \omega_1 = \sqrt{2} $ e $ \omega_2 = 2 \pi $ e quindi $ T_1 = \frac{ 2 \pi}{ \sqrt{2} }$ e $ T_2 = \frac{ 2 \pi}{ 2} $.\\
	In questo caso, $ \frac{T_1}{T_2} = \frac{ 2 }{ \sqrt{2} } $ è un rapporto irrazionale quindi $ v(t)$ non è periodico.\\
	
	\subsection{Segnale periodico}
	$
		v(t)
		= \sum_{k= -\infty}^{\infty} v_k \cos ( 2 \pi f_0 kt + \phi_k)
		= \sum_{k= -\infty}^{\infty} v_k e^{ j 2 \pi f_0 t}
	$ è periodico con $ v_k \in \mathbb{C}$ per ogni $ k \in \mathbb{Z}, \phi_k \in \mathbb{R} $ e $ t \in \mathbb{R}$ se:\\
	$
		\frac{ 2 \pi f_0 k_1}{2 \pi f_0 k_2}
		= \frac{k_1}{k_2}
		\in \mathbb{Q}
	$ con $k_1,k_2 \in \mathbb{Z}$.\\
	
	\textbf{OSS1:} Se $ v_k \in \mathbb{C}$ per ogni $ k \in \mathbb{Z}$ allora $ v_k = |v_k|e^{j arg(v_k)} $.\\
	Posso quindi riscrivere il segnale $ v(t) = \sum_{k= -\infty}^{\infty} |v_k| e^{j(2 \pi f_0 t + arg(v_k) )}$.\\
	\textbf{OSS2:} Le frequenze sono $f_0,2f_0,3f_0,... $ ma anche $-f_0,-2f_0,-3f_0,... $.\\
	Proviamo ora a rispondere alla seconda domanda cioè: il segnale periodico $ v(t)$ come si può scrivere come somma di fasori?\\
	Per rispondere a questa domanda enunciamo un teorema.
	
	\subsection{Teorema: Se un segnale è periodico posso scriverlo come somma di fasori}
	Sia $ v(t) $, con $t \in \mathbb{R} $, un segnale periodico con $T_0=\frac{1}{f_0} $.\\
	Se:\\
	1) $ v(t) $ è generalmente continua ( cioè ha un numero finito di discontinuità)\\
	2) $ v(t) $ è generalmente derivabile con la derivata continua e limitata su $ [t_0,t_0+T_0] $ con $ t_0 \in \mathbb{R}$\\
	Allora soddisfa:\\
	i) $ \int_{t_0}^{ t_0+T_0} |v(t)| dt < +\infty $, cioè l'integrale converge.\\
	Si dice che $ v(t) $ è sommabile su un periodo.\\
	ii) $ \int_{t_0}^{ t_0+T_0} |v(t)|^2 dt < +\infty $, cioè $ v(t) $ è al quadrato sommabile.\\
	iii) $ \sum_{k= -\infty}^{\infty} v_k e^{j 2 \pi k f_0 t} $ \textbf{ equazione di sintesi}\\
	per ogni $t \in \mathbb{R} $ e dove\\
	$ \frac{1}{T_0}\int_{t_0}^{ t_0+T_0} v(t) e^{-j 2 \pi f_0 t} dt $ \textbf{ equazione di analisi}\\
	dove $ k \in \mathbb{Z}$.\\
	$ v_k $ si chiamano \textbf{ i coefficienti dello sviluppo in serie di Fourier}.\\
	Se $ v(t) $ non è continua in t allora:\\
	$ \frac{v(t^-)+v(t^+)}{2}= \sum_{k= -\infty}^{\infty} v_k e^{- j 2 \pi k f_0 t} $.\\
	
	\textbf{OSS1:}\\
	Con $ k=0$, $ 
		v_0
		= \frac{1}{T_0}\int_{t_0}^{ t_0+T_0} v(t) e^0 dt
		=\int_{t_0}^{ t_0+T_0} v(t) dt
	$ cioè il valor medio di un periodo.\\ % è sparito \frac{1}{T_0} non so perchè
	\textbf{OSS2:}\\
	Se $ v(t) $ è reale, cioè $ v(t) \in \mathbb{R}$, allora:\\
	$ v_{-k} =\frac{1}{T_0} \int_{t_0}^{ t_0+T_0} v(t) e^{-j 2 \pi(-k) f_0 t} dt =\overline{v_k}  $ con $ k \in \mathbb{Z}$.\\
	NB: $ k \in \left \{ ...,-3,-2,-1,0,1,2,3,... \right \} = \mathbb{Z} $\\
	Allora $|v_k|=|v_{-k}| $ e $arg(v_k)= - arg(v_{-k}) $.\\
	%TODO: immagine coniugato complesso
	Possiamo quindi riscrivere l'equazione di sintesi come:\\
	%TODO: nell'equazione sotto manca un segno?
	$
		v(t)
		=\sum_{k= -\infty}^{-1} |v_{-k}| e^{ -? j arg(v_{-k})} e^{2 \pi j k f_0 t}+ v_0+ \sum_{k= 1}^{ \infty} |v_{k}| e^{j arg(v_{k})} e^{2 \pi j k f_0 t}
		= v_0+ 2\sum_{k= 1}^{ \infty} |v_{k}| \cos( 2 \pi k f_0 t + arg(v_{k}))
	$\\
	NB: qui sopra abbiamo usato Eurelo.\\
	Scriviamo $ A_k = 2 \Re(v_k) = 2 |v_k| \cos (arg(v_k))$ e
	$ B_k = -2 \Im(v_k) = -2 |v_k| \sin (arg(v_k))$
	con per ogni $ k \in \mathbb{Z}$ e $ A_k,B_k \in \mathbb{R}$\\
	NB: $ \cos(a+b) = \cos a \cos b -\sin a\sin b$\\
	L'equazione quindi diventa:\\
	$
		v(t)
		= v_0+ \sum_{k= 1}^{ \infty} A_k \cos( 2 \pi k f_0 t)+ \sum_{k= 1}^{ \infty} B_k \sin ( 2 \pi k f_0 t)
	$\\
	
	\textbf{OSS3:}\\
	Se $ v(t) $ è pari allora $ \overline{v_k} = v_k = v_{-k} $, ciò significa che $ v_k $ ha la parte immaginaria nulla. Quindi $ B_k=0 $ per ogni k, l'equazione sarà quindi:\\
	$
		v(t)
		= v_0+ \sum_{k= 1}^{ \infty} A_k \cos( 2 \pi k f_0 t)
	$\\
	con v reale e pari.\\
	
	\textbf{OSS4:}\\ 
	Se $ v(t) $ è dispari allora $ v_0=0 $ e quindi:\\
	$
		v(t)
		= \sum_{k= 1}^{ \infty} B_k \sin( 2 \pi k f_0 t)
	$\\
	
	\textbf{OSS5:}\\
	Nelle applicazioni pratiche useremo la serie troncata:\\
	$ v_L(t) = \sum_{k= -L}^{ L} V_k e^{j 2 \pi k f_0 t}$\\
	Si può dimostrare che i coefficienti $ V_k = v_k $ sono quelli che minimizzano l'errore quadratico medio (MSE) dove:\\ 
	$
	MSE(V_L(t), v(t))
	= \frac{1}{T_0} \int_{t_0}^{ t_0+T_0} |v(t) - v_L(t)|^2 dt
	$ dove $ |v(t) - v_L(t)|^2 $ è l'energia di $ v(t) - v_L(t)$ .\\
	Abbiamo anche che $ \lim_{L \to \infty}  MSE(V_L(t), v(t)) = 0$.
	
\section{Potenza di un segnale}
	
	Sia $ v(t) $ al quadrato sommabile, di periodo $ T_0$.\\
	Allora definiamo la sua potenza come:\\
	$
		P_v
		:= \lim_{T \to \infty} \frac{1}{ 2 T} \int_{-T}^{ T} |v(t)|^2 dt
		= \frac{1}{T_0} \int_{t_0}^{t_0+T_0} |v(t)|^2 dt
	$\\
	Si può dimostrare (teorema di Parseval) che:\\
	$
		P_v
		= \sum_{k= -\infty}^{ \infty}  |v_k|^2
	$\\

\section{Risposta di un sistema LTI ad un segnale periodico}
	Se $ H(2 \pi fj) $ è la risposta in frequenza e $ u(t) = A \cos (2 \pi k f_0 t + \phi) $ è l'ingresso allora:\\
	$
		v(t)
		= A |H(f)| \cos ( 2 \pi k f_0 t + \phi + arg( H(f)))
	$.\\
	Se $ u(t) = u_0 + \sum_{k= 1}^{ \infty} |u_k| \cos ( 2 \pi k f_0 t + arg(u_k)) $ allora:\\
	$
	v(t)
	= H(0)u_0 + 2 \sum_{k= 1}^{ \infty} |H(k f_0)| |u_k| \cos ( 2 \pi k f_0 t + arg(u_k) + arg(H(kf_0)))
	$.\\

\section{Condizioni di esistenza della trasformata di Fourier}
	
	Queste condizioni sono differenti ma alternative.\\
	
	%TODO: nell'eq sotto di va un modulo?
	i) $ \int_{- \infty}^{ \infty} v(t) dt < +\infty$ (sommabile) e v è a variazione limitata (si può esprimere come differenza di funzioni limitate e non decrescenti).\\
	
	ii) $ \int_{- \infty}^{ \infty} |v(t)|^2 dt < \infty$ (al quadrato sommabile), v è un segnale di energia.\\
	%TODO: cosa si intende con un segnale di energia?
	
	iii) $ \int_{- \infty}^{ \infty} |v(t)|^2 dt = +\infty$ ma $ \lim_{T \to \infty} \frac{1}{2T}   \int_{- T}^{ T} |v(t)|^2 dt < +\infty $, v è un segnale di potenza (ha energia finita). In questo caso per calcolare la trasformata di Fourier di v(t) bisogna "finestrare" il segnale.\\
	
\section{Trasformate di Fourier notevoli $ \rightarrow $ sotto le condizioni i) e ii) }
	
	\subsubsection{a) TdF dell'impulso}
	
	$ \FO [ \delta(t) ] = \int_{- \infty}^{ \infty} \delta (t) e^{-2 \pi j f t} dt = e^{-2 \pi j f  0} = 1 $\\
	
	%TODO: immagine, dall'impulso al grafico di v(f)=1
	
	\subsubsection{b) TdF dell'esponenziale complesso causale}
	
	$ v(t) = A e^{ j \phi} e^{ \lambda t} \delta_{-1}(t) $ con $ A \in \mathbb{R}^*_+, \phi \in \mathbb{R}, \lambda \in \mathbb{C}, \Re (\lambda ) < 0 $ (quest'ultima mi dà la stabilità quindi l'integrale converge).\\
	
	$ \FO [ v(t) ] = \int_{- \infty}^{ \infty} A e^{ j \phi} e^{ \lambda t} \delta_{-1}(t) e^{-2 \pi jft} dt $\\
	Notiamo qui che "possiamo portare fuori" $ A e^{ j \phi} $ e che l'integrale può andare da 0 a $ + \infty$ perchè ho $\delta_{-1}(t) $, la trasformata così diventa:\\
	$ A e^{ j \phi} \int_{ 0 }^{ \infty}  e^{ \lambda t -2 \pi jft} dt = \frac{A e^{j \phi}}{j2 \pi f - \lambda} $\\
	OSS: invece che svolgere l'integrale posso:\\
	$ \LA [ e^{\lambda t} \delta_{-1}(t)] |_{s=j \omega} = \frac{1}{s-\lambda} |_{s=j \omega} = \frac{1}{j2 \pi f-\lambda}$
	
	\subsubsection{c) TdF dell'esponenziale complesso anticausale (non posso usare la TdL)}
	
	%TODO: immagine, grafico della delta anticausale 
	
	$ \FO [ A e^{ j \phi} e^{ \lambda t} \delta_{-1}(-t) ] =  \frac{ -Ae^{j \phi}}{j2 \pi f-\lambda} $\\
	
	
	\subsubsection{d) TdF della finestra di ampiezza A e base T}
	$ A, T \in \mathbb{R}^*_+ $.\\
	$ v(t) = A \prod (\frac{t}{T}) $\\
	
	%TODO: immagine -> funzione rettangolo
	
	$ \FO [  A \prod (\frac{t}{T}) ] = A \int_{- \frac{T}{2}}^{ \frac{T}{2}} e^{-2 \pi jft} dt = \frac{-A}{j2 \pi f} e^{-2 \pi jft} |^{\frac{T}{2}}_{-\frac{T}{2}}  = - \frac{A}{j2 \pi f} [ e^{- \pi jf T } - e^{\pi jf T }]  $\\
	Possiamo usare qui la formula di Eurelo (sezione A dell'appendice: ripasso dei numeri complessi).\\
	$ = + \frac{A}{ \pi f} \sin ( \pi f T) $\\
	Moltiplico e divido per T, così da aver la funzione sinc.\\
	$ = + A T \frac{ \sin ( \pi f T) }{ \pi f T } = AT sinc(f T) $\\
	
	%TODO: dalla funzione rettangolo alla sinc

\section{Trasformate di Fourier di segnali di potenza $ \rightarrow $ sotto la condizione iii), cioè $ \Re ( \lambda ) < 0$ }

	Il segnale va moltiplicato per la finestra $ \prod (\frac{t}{T} )$, facciamo la trasformata e poi facciamo il limite per T che tende a $ \infty$.\\
	
	\subsubsection{a) TdF di un segnale continuo A}
	1) Finestriamo: $ v(t) = A$, lo finestriamo con $ \prod (\frac{t}{T} )$.\\
	NB: definiamo $ v_T(t)$ come il segnale già finestrato.\\
	$ v_T(t) = A \prod (\frac{t}{T} )$\\
	%TODO: immagine della funzione finestrata
	2) Facciamo la trasformata: $ \FO [ v_T(t) ] = AT sinc(fT)$.\\
	3) Facciamo il limite: $ \lim_{T \to \infty} AT sinc(fT) = A \delta (t)$.\\
	
	%TODO: immagine della funzione sinc quando T tende a infinito
	
	%TODO: immagine trasformata dalla costante alla delta
	
	NB: in pratica non ho mai un segnale costante perchè i segnali nella realtà sono sempre causali e che prima o poi finiscono.\\
	OSS: v(t)=A ha un'energia limitata e frequenza nulla.\\
	
	\subsubsection{b) TdF di un esponenziale complesso}
	$ v(t) = A e^{j2\pi f_0 t}$\\
	%TODO: da controllare i calcoli
	1) e 2) Finestriamo e facciamo la trasformata:\\
	 $ \FO [ v_T(t)  ] = A \int_{- \frac{T}{2}}^{ \frac{T}{2}} e^{-j2\pi (f-f_0) t} dt  = AT sinc( (f-f_0) t) $.\\
	3) Facciamo il limite: $ \lim_{T \to \infty} \FO [ v_T(t)  ]  = \lim_{T \to \infty} AT sinc( (f-f_0) t) = A \delta (f-f_0)$.\\
	
	\subsubsection{c) TdF del coseno}
	$ v(t) = A \cos (2 \pi f_0 t) = A \frac{e^{j 2 \pi f_0 t} + e^{-j 2 \pi f_0 t}}{ 2}$\\
	In questo caso possiamo ricondurci a b) perchè è la somma di due esponenziali, quindi:\\
	$ \lim_{T \to \infty} \FO [ v_T(t)  ]  = \frac{A}{2} \delta (f-f_0) + \frac{A}{2} \delta (f+f_0)$.\\
	
	%TODO: immagine della trasformata
	
	\subsubsection{d) TdF del seno}
	$ v(t) = A \sin (2 \pi f_0 t) $\\
	Possiamo di nuovo ricondurci a b) perchè è la somma di due esponenziali (Eurelo), quindi:\\
	$ \lim_{T \to \infty} \FO [ v_T(t)  ]  = \frac{A}{2}j[ \delta (f+f_0) - \delta (f-f_0) ]$.\\
	
	%TODO: immagine della trasformata
	
	%TODO: come rendere carino questo riassunto?
	\subsubsection{Riassunto}
	
	TdF dell'impulso:\\
	$ \quad \FO [ \delta(t) ] = 1 $\\
	TdF dell'esponenziale complesso causale:\\
	$ \FO [  A e^{ j \phi} e^{ \lambda t} \delta_{-1}(t) ] = \frac{A e^{j \phi}}{j2 \pi f - \lambda} $\\
	TdF dell'esponenziale complesso anticausale:\\
	$ \FO [ A e^{ j \phi} e^{ \lambda t} \delta_{-1}(-t) ] =  \frac{ -Ae^{j \phi}}{j2 \pi f-\lambda} $\\
	TdF della finestra di ampiezza A e base T:\\
	$ \FO [  A \prod (\frac{t}{T}) ] = AT sinc(f T) $\\
	TdF di un segnale continuo A:\\
	$ v(t) = A $\\
	$ \lim_{T \to \infty} \FO [  v(t) \prod (\frac{t}{T}) ] = A \delta(t) $\\
	TdF di un esponenziale complesso:\\
	$ v(t) = A e^{j2\pi f_0 t}$\\
	$ \lim_{T \to \infty} \FO [  v(t) \prod (\frac{t}{T}) ] = A \delta (f-f_0)$.\\
	TdF del coseno:\\
	$ v(t) = A \cos (2 \pi f_0 t) = A \frac{e^{j 2 \pi f_0 t} + e^{-j 2 \pi f_0 t}}{ 2}$\\
	$ \lim_{T \to \infty} \FO [  v(t) \prod (\frac{t}{T}) ] = \frac{A}{2} \delta (f-f_0) + \frac{A}{2} \delta (f+f_0)$.\\
	TdF del seno:\\
	$ v(t) = A \sin (2 \pi f_0 t)$\\
	$ \lim_{T \to \infty} \FO [  v(t) \prod (\frac{t}{T}) ] = \frac{A}{2}j[ \delta (f+f_0) - \delta (f-f_0) ]$.\\

\section{ Proprietà della trasformata di Fourier }

	\subsubsection{1) Linearità}
	
	$ a_1 v_1 (t) + a_2 v_2 (t) \xrightarrow{ \FO} a_1 v_1 (f) + a_2 v_2 (f) $
	
	\subsubsection{2) Riflesso, coniugato ed entrambi insieme }
	
	Riflesso: $ v(-t) \xrightarrow{ \FO} v(-f) $\\
	Coniugato: $ \overline{v(t)} \xrightarrow{ \FO} \overline{v(-f)} $\\
	Riflesso e coniugato: $ \overline{v(-t)} \xrightarrow{ \FO} \overline{v(f)} $\\
	
	\subsubsection{3) Cambiamento di scala}
	
	$ v(rt) \xrightarrow{ \FO} \frac{1}{r} v(\frac{f}{r}) $ con $ r \neq 0 $\\
	Estensione $ \xrightarrow{ \FO}$ Compressione
	
	%TODO: immagine grafico delle delta nel dominio delle frequenze
	
	\subsubsection{4) Convoluzione}
	
	$ [ v_1 * v_2 ](t) \xrightarrow{ \FO} v_1(f)v_2(f) $\\
	Convoluzione $ \xrightarrow{ \FO}$ Moltiplicazione
	
	\subsubsection{5) Modulazione generalizzata}
	
	$ v_1(t)v_2(t) \xrightarrow{ \FO} [ v_1 * v_2 ](f) $\\
	Moltiplicazione $ \xrightarrow{ \FO}$ Convoluzione
	
	\subsubsection{6) Ritardo temporale}
	
	$ v(t-t_0) \xrightarrow{ \FO} e^{-2 \pi f t_0} v(f) $\\
	Ritardo $ \xrightarrow{ \FO}$ Moltiplicazione per un'esponenziale
	
	\subsubsection{7) Traslazione sul dominio delle frequenze (o delle trasformate), proprietà di modulazione}
	
	$ v(t) e^{ j 2 \pi f_0 t} \xrightarrow{ \FO}  v(f-f_0)$\\
	Moltiplicazione per un'esponenziale $ \xrightarrow{ \FO}$ Ritardo nelle frequenze\\
	OSS: usiamo la proprietà 5) \\
	$ v(t) e^{ j 2 \pi f_0 t} \xrightarrow{ \FO } v(f) * \delta(f-f_0) = v(f-f_0)$\\
	NB: moltiplichiamo per il coseno (vale la linearità)\\
	$ v(t) \cos ( 2 \pi f_0 t) \xrightarrow{ \FO}  \frac{1}{2}v(f-f_0) + \frac{1}{2}v(f+f_0)$\\
	
	%TODO: immagine del coseno X immagine delle due delta = immagine di due piccoli coseni
	
	\subsubsection{8) Derivazione}
	
	Derivata prima:\\
	$ \frac{d v(t)}{dt} \xrightarrow{ \FO} (j2\pi f) v(f) $\\
	Derivata generalizzata:\\
	$ \frac{d^k v(t)}{dt} \xrightarrow{ \FO} (j2\pi f)^k v(f) $\\
	Derivata $ \xrightarrow{ \FO}$ Moltiplicazione
	
	\subsubsection{9) Integrazione}
	
	$ \int_{- \infty}^{ t} v( \tau) d\tau \xrightarrow{ \FO} \frac{v(f)}{j2\pi f} + \frac{1}{2} v(0) \delta(f) $\\
	Integrazione $ \xrightarrow{ \FO}$ Somma ???
	%TODO: non sapevo esattamente cosa mettere qui

	% Giorno: 8/5
\section{ Replicazione e campionamento }

	\subsubsection{Def. Treno campionatore ideale}
	
	Si definisce $  \tilde{\delta}_T (t) = \sum_{k= -\infty}^{\infty} \delta (t - kT) $ e lo chiamiamo treno campionatore ideale con $ k \in \mathbb{Z} $.\\
	E' una serie di impulsi localizzati in $kT$ con $T>0$.\\
	%TODO: immagine del treno
	Consideriamo gli impulsi come distribuzioni e quindi come una serie di box.\\
	Perciò $ \tilde{\delta}_T (t) $ si può scrivere come:\\
	
	$  \lim_{ \tau \to \infty} \sum_{k= -\infty}^{\infty} \frac{1}{\tau} \prod (\frac{t-kT}{\tau}) $\\
	
	Sia $ v_\tau (t) = \sum_{k= -\infty}^{\infty} \frac{1}{\tau} \prod (\frac{t-kT}{\tau}) $, è periodico, quindi si può esprimere usando la serie di Fourier:\\
	
	$
		v_\tau (t)
		= \frac{1}{T}\sum_{k= -\infty}^{\infty} sinc(\frac{k \tau}{T}) e^{j \frac{2\pi}{T} kt}
	$\\
	
	%TODO: qua secondo me qualcosa non torna
	Da notare come:
	$
		v_k
		= \int_{- \infty}^{ \infty} v_r(t) e^{ -j \frac{2 \pi}{T} t} dt
		= \frac{1}{T}sinc(\frac{k \tau}{T})
	$\\
	
	$
		\tilde{\delta}_T (t)
		= \lim_{ \tau \to \infty} v_r(t)
		= \lim_{ \tau \to \infty} \frac{1}{T} \sum_{k= -\infty}^{\infty} sinc(\frac{k \tau}{T}) e^{j \frac{2 \pi}{T}k t}
		= \frac{1}{T} \sum_{k= -\infty}^{\infty} e^{j \frac{2 \pi}{T}k t}
	$\\

	%TODO: immagine lo stesso treno ma con le box
	
	Ora lo trasformiamo con Fourier:\\
	$ \FO [ \tilde{\delta}_T (t) ] = \frac{1}{T} \sum_{k= -\infty}^{\infty} \delta (f-\frac{k}{T}) $\\
	La TdF del treno compianatore ideale è un treno campionatore ideale (nelle frequenze) in cui gli impulsi hanno area $ \frac{1}{T}$ e sono equiserparati di $ \frac{1}{T} $.\\
	
	%TODO: immagine dal dominio del tempo a quello delle frequenze
	
	\subsubsection{Replicazione di un segnale}
	
	%TODO: immagine della box replicata
	
	Scelgo un segnale semplice che voglio replicare:\\
	$ [rep_T v](t) = \sum_{k= -\infty}^{\infty} v(t-kT) = \sum_{k= -\infty}^{\infty} v(t)* \delta(t-kT)$\\
	In parole povere: "sposto il segnale", prendo tutti i grafici del segnale spostato e li sommo insieme.\\
	
	%TODO: immagine del segnale sospostato
	
	\subsubsection{Campionamento - sampling in inglese}
	
	%TODO: immagine di un segnale campionato
	
	$ [samp_T v](t) \overset{\mathit{def}}{=} \sum_{k= -\infty}^{\infty} v(kT) = \sum_{k= -\infty}^{\infty} v(t)\delta(t-kT)= v(t)\tilde{\delta}_T (t)$\\
	NB: l'ultimo passaggio è possibile grazie ala proprietà di campionamento dell'impulso.\\
	Siamo passati da un segnale discreto ad avere un segnale continuo.\\
	
	Applicando le proprietà della TdF possiamo vedere che:\\
	$ [rep_T v](t) = [v*\tilde{\delta}_T](t) $\\
	usando la quarta proprietà (convoluzione) viene fuori che:\\
	$ v(f) = \frac{1}{T} \tilde{\delta}_{\frac{1}{T}} (f) = \frac{1}{T} [samp_{\frac{1}{T}} v](f)$\\
	$ [samp_T v](t) =  v(t)  \tilde{\delta}_T (t)$\\
	usando la quinta proprietà (modulazione) viene fuori che:\\
	$ v(f) * \frac{1}{T} \tilde{\delta}_{\frac{1}{T}} (f) = \frac{1}{T} [rep_{\frac{1}{T}} v](f) $\\
	
	%TODO: immagine, segnale nel dominio delle frequenze (una sola gobba di cammello)
	
	Usando il segnale campionato che grafico viene fuori?\\
	
	%TODO: immagine, tante gobbe di cammello
	
	Come posso ricostruire un segnale campionato senza perdere l'informazione? Posso "finestrare" con una box l'unico segnale nell'origine.\\
	Ho due problemi:\\
	1) un segnale reale è limitato\\
	2) può capitare che il segnale centrale sia troppo vicino al suo successivo e al suo predecessore. Essi quindi si sofrappongono \textbf{(Aliasing)}\\
	
	%TODO: immagine sovrapposizione del segnale
	
	Per ovviare a quest'ultimo problema enunciamo il prossimo teorema.
	
	\subsubsection{Teorema di campionamento (Shannon)}
	
	Dato un segnale continuo $ v_a(t)$ (la "a" sta per analogico) e la sua versione campionata $ v(k) = v_a(kT)$ con $ k \in \mathbb{Z} $. La fraquenza di campionamento è $ f_c= \frac{1}{T} $.\\
	Se:\\
	1) $ v_a(t) $ è limitato in banda cioè $ V_a(f) = \FO [v_a(t)]$ e $ \exists B>0 $ (il più piccolo) tale che $ V_a(f) =0 $ per ogni f tale che $ |f|>B$.\\
	Se non ho B (banda) ho aliasing (ciò però non basta).\\
	2) $ f_c > 2B $, $ 2B $ è chiamata \textbf{frequenza di Nyquist}.\\
	Allora:\\
	il segnale $ v_a(t) $ può essere ricostruito a partire da $ v(k) $, cioè la sua versione campionata. Possiamo fare ciò usando il \textbf{filtro di ricostruzione (la box)}:\\
	$ H_r(f) = T \prod (\frac{f}{2 f_L}) = \frac{1}{f_c} \prod (\frac{f}{2 f_L})$\\
	a condizione che $ f_L $ sia $ B \leq f_L \leq f_c - B$.\\
	NB: sto finestrando nelle frequenze.
	
	%TODO: immagini con i vari casi, guarda video Youtube
	
	Con questo teorema possiamo ricostruire il segnale nel continuo senza timori.\\
		$ [samp_T v_a](t)
		= \sum_{k= -\infty}^{\infty} v_a(kT) \delta(t-kT)
		= \sum_{k= -\infty}^{\infty} v(k) \delta(t-kT) $.\\
		$ h_r(t)
		= \FO^{-1} [\frac{1}{f} \prod (\frac{f}{2 f_L})]
		= sinc(\frac{t}{T})$
	 con $ f_L = \frac{f_c}{2}$.\\
	allora avrò ricostruito il segnale analogico:\\
		$ v_a(t)
		= [samp_T v_a * h_r] (t)
		= [ \sum_{k= -\infty}^{\infty} v(k) \delta(t-kT) * h_r](t)
		= \sum_{k= -\infty}^{\infty} v(k) sinc(\frac{t-kT}{T})$\\
	
	\textbf{Formula di interpolazione ideale (o di Shannon):}
	$ v_a(t) = \sum_{k= -\infty}^{\infty} v(k) sinc(\frac{t-kT}{T})$\\
	
	Problemi nella pratica:\\
	1) sinc non è causale perchè ha supporto infinito.\\
	2) la sommatoria è infinita (va da $ $ a $ $).\\
	3) un segnale limitato nel tempo non è limitato in banda.\\
	
	%TODO: immagine, sinc sovrapposte
\chapter{Diagrammi di Bode}
\section{Rappresentazione della funzione di trasferimento}

%TODO: Io la riscrivo, ma forse è meglio fare riferimento al capitolo 3 e magari metterci un link sul capitolo 3 e sottolineare la formula della funzione di trasferimento

%TODO: inserire una riga d'intestazione per non fare iniziare la sezione con una formula centrale

\begin{equation}
	H(j\omega) \dot{=}\int_{-\infty}^{+\infty} \! h(t)e^{-j\omega t} \d t
\end{equation}
dato uno specifico $\omega$ abbiamo un risultato nel piano complesso:
\[
	H(j\omega) \big\vert_{\omega = \omega_k}
\]

%TODO non so dove piazzare questo diagramma nel contesto
%TODO: fare immagine
\begin{figure}[H]
	\centering
	\includegraphics[width=0.7\linewidth]{immagini/cap6_Bode/diagNyquist}
	\caption{ Diagramma di nyquist: la funzione di trasferimento viene rappresentata come una curva parametrica al variare di $\omega$.  }
	\label{fig:diagNyquist}
\end{figure}

La rappresentazione su grafici rende più facile l'interpretazione della funzione di trasferimento. Per semplicità si può dividere il segnale nel modulo e nella fase.

\begin{equation*}
\begin{align}
 A(\omega)&=\abs{H(j\omega)}=\Abs{\int_{-\infty}^{+\infty} \! h(t)e^{-j\omega t} \d t} &\text{funzione pari}\\	
 \Phi (\omega)&= \arg(H(j\omega)) & \text{funzione dispari}
\end{align}
\end{equation*}

Rappresentiamo un numero complesso $z=a+jb$

\begin{equation*}
	\includegraphics[width=0.3\linewidth]{immagini/cap6_Bode/argCompl}\quad
	\arg(z) =
	\begin{cases}
		\arctan \big(\frac{b}{a}\big) &,a>0, b\in \R\\
		\frac{\pi}{2}&,a=0, b>0\\
		-\frac{\pi}{2}&,a=0, b<0\\
		\arctan \big(\frac{b}{a}\big)+\pi &,a<0, b\ge 0\\
		\arctan \big(\frac{b}{a}\big)-\pi &,a<0, b< 0
	\end{cases}
\end{equation*}

Vista l'ampiezza delle possibili frequenze, usualmente si usa una scala logaritmica per esse. Vediamo come usare il logaritmo nei numeri complessi.

Nei reali: 
\begin{equation*}
	y=\ln (x) \iff x e^y, \quad x \in \R, y \in \R 
\end{equation*}
Nei complessi: %TODO da risistemare
\begin{gather*}
	w = \ln (z) \iff z=e^w \quad z \in \C^*, w \in \C \\
	z=\rho e^{j \theta} \quad w=u+jv\\
	z= \rho e^{j \theta} = e^w=e^{u+jv} = e^u e^{jv}\\
	\rho = e^u \rightarrow u = \ln(\rho) \rightarrow u =\ln(\abs{z})\\
	e^{j\theta} = e^{jv} \rightarrow v = \theta \rightarrow v= \arg (z)\\
	\ln (z) = w = u+jv = \ln (\abs{z}) + j \arg (z)
\end{gather*}

%TODO eh?
Logaritmo principale: dove $ -\pi < \arg (z) < \pi $ (oppure $ 0< \arg (x) <2\pi $)

\subsubsection{Proprietà del logaritmo}
\begin{enumerate}
	\item $ \ln ( b c) = \ln (b) + \ln (c) $
	\item $ \ln \Big(\frac{b}{c}\Big) = \ln (b) - \ln (c) $
	\item $ \ln (b^c) = c \ln (b) \quad ,c \in \R$
	\item $ \log_a (b)  = \frac{\log_c (b)}{\log_c (a)}=\frac{1)}{\log_c (a)}\log_c (b)$
\end{enumerate}

Per analogia:
\begin{enumerate}
	\item $ \arg ( b c) = \arg (b) + \arg (c) $
	\item $ \arg \Big(\frac{b}{c}\Big) = \arg (b) - \arg (c) $
	\item $ \arg (b^c) = c \arg (b) \quad ,c \in \R$
	\item Non c'è cambio di base
\end{enumerate}

\subsection{Decibel}
L'unità di misura dell'ampiezza nei diagrammi di Bode è il Decibel
\[
\abs{H(j\omega)}_{dB} = 20 \log_{10}\abs{H(j\omega)}
\]

%TODO nemmeno qui non so dove piazzare questo diagramma nel contesto
%TODO: fare immagine
\begin{figure}[H]
	\centering
	\includegraphics[width=0.7\linewidth]{immagini/cap6_Bode/diagNichols}
	\caption{ Rappresentazione logaritmica nel Diagramma di Nichols  }
	\label{fig:diagNichols}
\end{figure}

\section{Diagrammi di Bode}
%TODO: fare immagine
\begin{figure}[H]
	\centering
	\includegraphics[width=0.7\linewidth]{immagini/cap6_Bode/diagBode}
	\label{fig:diagBode}
\end{figure}

%TODO introduzione migliore da fare

%TODO non so come introdurre i diseng sulle decadi
\begin{figure}[H]
	\centering
	\includegraphics[width=0.7\linewidth]{immagini/cap6_Bode/schDecade}
	\label{fig:schDecade}
\end{figure}
Considerando un punto nella decade tra $ 0 $ e $ 1 $ specifichaimo per gli esercizi
\[ 
	\log 2 \simeq 0,3 \quad \log 3 \simeq 0,5 \quad\log 5 \simeq 0,7 \quad\log 8 \simeq 0,9 
 \]
\begin{figure}[H]
	\centering
	\includegraphics[width=0.7\linewidth]{immagini/cap6_Bode/schDecade2}
	\label{fig:schDecade2}
\end{figure}

\[ 
	\ln (H(j\omega)) = \underbrace{\ln (A(\omega))}_{\text{daigramma Ampiezze}} + j\,\underbrace{\Phi (\omega)}_{\text{diagramma Fase}}
 \]
 
 In caso di un sistema:
 \begin{figure}[H]
 	\centering
 	\includegraphics[width=0.7\linewidth]{immagini/cap6_Bode/sist1}
 	\label{fig:sist1}
 \end{figure}

\[ 
	H(j \omega)=A(j \omega)B(j \omega)
 \]
 di conseguenza: 
 \[ 
	\ln H(j \omega)=\ln A(j \omega) + \ln B(j \omega)
 \]
 
 
 %TODO Questo cos'é? Un altra section? Non so che titolo mettere
 
 Possiamo scrivere $ H(s) $ come:
 \[  
 	K \, \frac{(s-z_1)\,(s-z_2)\,\dots \,(s-z_m)}{(p-z_1)\,(p-z_2)\,\dots \,(p-z_m)}
 \]
 in forma \emph{irriducibile} con $K \in \R  $, $ z_i $ zeri di $ H(s) $ e $ p_i $ poli di $ H(s) $
 
 Possiamo avere tre casi:
 \begin{enumerate}
 	\item $ z_1=0 $ e/o $ p_i = 0 $ cioè poli nell' \\
 	\item $ (s-z_i) $ e/o $ (s-p_i) $ non nulli ($ \ne 0  \wedge \in \R$)\\
 	\item poli complessi coniugati: $ (s-z_i)(s-\overline{z_i}) $ e/o  $ (s-p_i)(s-\overline{p_i}) $
 \end{enumerate}

\subsubsection{Caso 1}
Avremo un termine di questo tipo al denominatore:
 $ \frac{1}{s^\nu}  $
con 
\begin{itemize}
	\item $ \nu =0 $ Se $ H(s) $ non ha poli o zeri nell'origine\\
	\item $ \nu >0 $ Se $ H(s) $ ha polo nell'origine di molteplicità $ \nu $\\
	\item $ \nu <0 $ Se $ H(s) $ ha zeri nell'origine di molteplicità $\nu$
\end{itemize}

\subsubsection{Caso 2}
Zeri: $ s-z_i \quad z_i \ne 0 $
\[ 
	(s-z_i)=(-z_i)\Big(1+s \frac{1}{-z_1} \Big)=(-z_i)(1+s \tau'_i) \qquad \tau'_i \dot{=} \frac{1}{-z_i}
 \]
 
 (In questo caso $ \tau'_i $ non ha seignificato fisico) %TODO ho capito bene?
 
 Equivalentemente per i poli: $ s-p_i \quad p_i \ne 0 $
 \[ 
 (s-p_i)=(-z_i)\Big(1+s \frac{1}{-p_1} \Big)=(-p_i)(1+s \tau_i) \qquad \tau_i \dot{=} \frac{1}{-p_i} \, \text{Costante di tempo}
 \]
 %TODO nei miei appunti è l al posto di e e nell'ultima parte p al posto di Tau, purtroppo non ho trovato sul libro un riferimento 
 \[ (s-p_i) \rightarrow e^{p_i t} \rightarrow e^{-\frac{t}{\tau_1}} \, \text{decadimento esponenziale}\]
 
 \subsubsection{Caso 3}
 \[ 
 	(s-z_k)(s-\overline{z_k}) = s^2-z_ks-\overline{z_k}s+z_k \,\overline{z_k} = \Re(z_k) s+\abs{z_k}^2 = \abs{z_k}^2\Big(1-\frac{2 \Re(z_k)}{\abs{z_k}}\frac{s}{\abs{z_k}}+\frac{s^2}{\abs{z_k}} \Big)
 \]

 Definiamo $ \omega'_n \dot{=} \abs{z_k} $ come \emph{pulsazione naturale} e $ \zeta_k \dot{=} \frac{-\Re(z_k)}{\abs{z_k}} $ come \emph{fattore di smorzamento}, di conseguenza abbiamo: 
 \[ 
 	(s-z_k)(s-\overline{z_k}) = \abs{z_k}^2\Big(1+2\zeta'_k\frac{s}{\omega'_{nk}}+\frac{s^2}{\omega^{'2}_{nk}} \Big)
  \]
  
  Analogalmente con $ \omega_n \dot{=} \abs{p_k} $ e $ \zeta_k \dot{=} \frac{-\Re(p_k)}{\abs{p_k}} $ abbiamo:
   \[ 
  (s-p_k)(s-\overline{p_k}) = \abs{p_k}^2\Big(1+2\zeta_k\frac{s}{\omega_{nk}}+\frac{s^2}{\omega^{2}_{nk}} \Big)
  \]
  
  Ora possiamo riscrivere la funzione di trasferimento in $ s $:
  \[ 
  	H(s) = K_B \, \frac{\prod_i (1+s\tau'_i)^{\mu'_i}\,  \prod_k \Big(1+2\zeta'_k\frac{s}{\omega'_{nk}}+\frac{s^2}{\omega^{'2}_{nk}} \Big)^{\mu'_k}}{s^\nu\, \prod_i (1+s\tau_i)^{\mu_i}\,  \prod_k \Big(1+2\zeta_k\frac{s}{\omega_{nk}}+\frac{s^2}{\omega^{2}_{nk}} \Big)^{\mu_k} }
   \]
\chapter{Schemi}
\section{Schemi a Blocchi}
Serve a rappresentare un sistema graficamente dati ingressi e uscite

\begin{example}
	
	dato un segnale $ u(t)=\delta_{-2}(t) $ ottengo un blocco così:
	
	%%%%%%%%%% immagine %%%%%%%
	\begin{center}
			\tikzset{every picture/.style={line width=0.75pt}} %set default line width to 0.75pt        
	
		\begin{tikzpicture}[x=0.75pt,y=0.75pt,yscale=-1,xscale=1]
		%uncomment if require: \path (0,59.33332824707031); %set diagram left start at 0, and has height of 59.33332824707031
		
		%Shape: Rectangle [id:dp9183683659728177] 
		\draw   (288,6) -- (358,6) -- (358,46) -- (288,46) -- cycle ;
		%Straight Lines [id:da4783012359570893] 
		\draw    (358,26) -- (410.5,26) ;
		\draw [shift={(412.5,26)}, rotate = 180] [color={rgb, 255:red, 0; green, 0; blue, 0 }  ][line width=0.75]    (10.93,-3.29) .. controls (6.95,-1.4) and (3.31,-0.3) .. (0,0) .. controls (3.31,0.3) and (6.95,1.4) .. (10.93,3.29)   ;
		
		%Straight Lines [id:da8992957067336038] 
		\draw    (212.5,26) -- (284.5,26) ;
		\draw [shift={(286.5,26)}, rotate = 180] [color={rgb, 255:red, 0; green, 0; blue, 0 }  ][line width=0.75]    (10.93,-3.29) .. controls (6.95,-1.4) and (3.31,-0.3) .. (0,0) .. controls (3.31,0.3) and (6.95,1.4) .. (10.93,3.29)   ;
		
		
		% Text Node
		\draw (323,26) node   {$\frac{d}{dt}$};
		% Text Node
		\draw (249,12.32) node   {$u$};
		% Text Node
		\draw (386,12.32) node   {$y$};
		
		\end{tikzpicture}
	\end{center}
	ottengo:  $ y(t)=\delta_{-1}(t) $
\end{example}

\begin{example}
	Sistema massa molla smorzatore
	
	\begin{center}
	
	
	\tikzset{every picture/.style={line width=0.75pt}} %set default line width to 0.75pt        
	
		\begin{tikzpicture}[x=0.75pt,y=0.75pt,yscale=-1,xscale=1]
		%uncomment if require: \path (0,509); %set diagram left start at 0, and has height of 509
		
		%Shape: Axis 2D [id:dp2949613832069744] 
		\draw  (208,184.53) -- (458.5,184.53)(231.8,90) -- (231.8,199.5) (451.5,179.53) -- (458.5,184.53) -- (451.5,189.53) (226.8,97) -- (231.8,90) -- (236.8,97)  ;
		%Shape: Rectangle [id:dp8682403506536873] 
		\draw   (316.5,111) -- (386.5,111) -- (386.5,151) -- (316.5,151) -- cycle ;
		%Straight Lines [id:da4773229437516673] 
		\draw    (232,131) -- (316,131) ;
		
		
		%Shape: Wave [id:dp5588027634095529] 
		\draw   (249,131) .. controls (250.3,136.25) and (251.55,141.25) .. (253,141.25) .. controls (254.45,141.25) and (255.7,136.25) .. (257,131) .. controls (258.3,125.75) and (259.55,120.75) .. (261,120.75) .. controls (262.45,120.75) and (263.7,125.75) .. (265,131) .. controls (266.3,136.25) and (267.55,141.25) .. (269,141.25) .. controls (270.45,141.25) and (271.7,136.25) .. (273,131) .. controls (274.3,125.75) and (275.55,120.75) .. (277,120.75) .. controls (278.45,120.75) and (279.7,125.75) .. (281,131) .. controls (282.3,136.25) and (283.55,141.25) .. (285,141.25) .. controls (286.45,141.25) and (287.7,136.25) .. (289,131) .. controls (290.3,125.75) and (291.55,120.75) .. (293,120.75) .. controls (294.45,120.75) and (295.7,125.75) .. (297,131) .. controls (297.17,131.67) and (297.33,132.34) .. (297.5,133) ;
		%Straight Lines [id:da6778535466302025] 
		\draw    (231,152) -- (428.5,152) ;
		
		%Straight Lines [id:da06443124093800989] 
		\draw    (385.5,131) -- (409.5,131) ;
		\draw [shift={(411.5,131)}, rotate = 180] [color={rgb, 255:red, 0; green, 0; blue, 0 }  ][line width=0.75]    (10.93,-4.9) .. controls (6.95,-2.3) and (3.31,-0.67) .. (0,0) .. controls (3.31,0.67) and (6.95,2.3) .. (10.93,4.9)   ;
		
		%Straight Lines [id:da47502722033668277] 
		\draw    (316.5,151) -- (386.5,151) (321.5,147) -- (321.5,155)(326.5,147) -- (326.5,155)(331.5,147) -- (331.5,155)(336.5,147) -- (336.5,155)(341.5,147) -- (341.5,155)(346.5,147) -- (346.5,155)(351.5,147) -- (351.5,155)(356.5,147) -- (356.5,155)(361.5,147) -- (361.5,155)(366.5,147) -- (366.5,155)(371.5,147) -- (371.5,155)(376.5,147) -- (376.5,155)(381.5,147) -- (381.5,155) ;		
		
		% Text Node
		\draw (351.5,131) node   {$M$};
		% Text Node
		\draw (272,107) node   {$k$};
		% Text Node
		\draw (352,164.5) node   {$b$};
		% Text Node
		\draw (428.5,131) node   {$F_{ext}$};
		% Text Node
		\draw (462.5,175) node   {$x$};
	\end{tikzpicture}
	\end{center}

	con $ k $ costante della molla, $ f_{ext} $ forza applicata, $ M $ massa e $ b $ attrito.
	
	Scriviamo l'equazione del sistema in riferimento alla posizione $ x $ ricordandoci che la velocità è la prima derivata dello spazio mentre l'accellerazione ne è la seconda:
	\[
		\sum F = M \cdot a \Rightarrow F_{ext}-kx-bx'= Mx'' \Rightarrow F_{ext}=kx+bx'+ Mx''
	\]
	\[
		\LA \rightarrow F_{ext}=kX(s)+sbX(s)+ s^2MX(s) 
		= X(s) (k+sb+s^2M) \Rightarrow X(s)=F_{ext} \frac{1}{k+sb+s^2M}
	\]
	
	Trasformiamo nello schema a blocchi:
	
	\begin{center}
		\tikzset{every picture/.style={line width=0.75pt}} %set default line width to 0.75pt        
		
		\begin{tikzpicture}[x=0.75pt,y=0.75pt,yscale=-1,xscale=1]
		%uncomment if require: \path (0,300); %set diagram left start at 0, and has height of 300
		
		%Shape: Rectangle [id:dp6464909345742553] 
		\draw   (285,103) -- (355,103) -- (355,143) -- (285,143) -- cycle ;
		%Straight Lines [id:da3909182970545064] 
		\draw    (220.25,123) -- (283,123) ;
		\draw [shift={(285,123)}, rotate = 180] [color={rgb, 255:red, 0; green, 0; blue, 0 }  ][line width=0.75]    (10.93,-3.29) .. controls (6.95,-1.4) and (3.31,-0.3) .. (0,0) .. controls (3.31,0.3) and (6.95,1.4) .. (10.93,3.29)   ;
		
		%Straight Lines [id:da20464322978360183] 
		\draw    (355,123) -- (417.75,123) ;
		\draw [shift={(419.75,123)}, rotate = 180] [color={rgb, 255:red, 0; green, 0; blue, 0 }  ][line width=0.75]    (10.93,-3.29) .. controls (6.95,-1.4) and (3.31,-0.3) .. (0,0) .. controls (3.31,0.3) and (6.95,1.4) .. (10.93,3.29)   ;
		
		
		% Text Node
		\draw (320,123) node   {$\frac{1}{k+sb+s^{2} M}$};
		% Text Node
		\draw (253,111.25) node   {$F_{ext}$};
		% Text Node
		\draw (385,111.25) node   {$X( s)$};
		\end{tikzpicture}
	\end{center}
\end{example}

\subsection{Diagramma ad anello chiuso}

Se l'ingresso non dipende dall'uscita si dice ad \emph{Anello Aperto}, altrimenti ad \emph{Anello Chiuso} (o retroazionato). Quest'ultimo è quello più usato.

\begin{center}
	\tikzset{every picture/.style={line width=0.75pt}} %set default line width to 0.75pt        
	
	\begin{tikzpicture}[x=0.75pt,y=0.75pt,yscale=-1,xscale=1]
	%uncomment if require: \path (0,292); %set diagram left start at 0, and has height of 292
	
	%Shape: Rectangle [id:dp3155084585495189] 
	\draw   (324.5,123) -- (394.5,123) -- (394.5,163) -- (324.5,163) -- cycle ;
	%Straight Lines [id:da11063592261012656] 
	\draw    (260.25,143) -- (323,143) ;
	\draw [shift={(325,143)}, rotate = 180] [color={rgb, 255:red, 0; green, 0; blue, 0 }  ][line width=0.75]    (10.93,-3.29) .. controls (6.95,-1.4) and (3.31,-0.3) .. (0,0) .. controls (3.31,0.3) and (6.95,1.4) .. (10.93,3.29)   ;
	
	%Straight Lines [id:da8163995279030745] 
	\draw    (395,143) -- (487.3,143) ;
	\draw [shift={(489.3,143)}, rotate = 180] [color={rgb, 255:red, 0; green, 0; blue, 0 }  ][line width=0.75]    (10.93,-3.29) .. controls (6.95,-1.4) and (3.31,-0.3) .. (0,0) .. controls (3.31,0.3) and (6.95,1.4) .. (10.93,3.29)   ;
	
	%Flowchart: Connector [id:dp7546342860954656] 
	\draw   (227.95,143) .. controls (227.95,134.08) and (235.18,126.85) .. (244.1,126.85) .. controls (253.02,126.85) and (260.25,134.08) .. (260.25,143) .. controls (260.25,151.92) and (253.02,159.15) .. (244.1,159.15) .. controls (235.18,159.15) and (227.95,151.92) .. (227.95,143) -- cycle ;
	%Straight Lines [id:da6036888305055024] 
	\draw    (163.2,143) -- (225.95,143) ;
	\draw [shift={(227.95,143)}, rotate = 180] [color={rgb, 255:red, 0; green, 0; blue, 0 }  ][line width=0.75]    (10.93,-3.29) .. controls (6.95,-1.4) and (3.31,-0.3) .. (0,0) .. controls (3.31,0.3) and (6.95,1.4) .. (10.93,3.29)   ;
	
	%Straight Lines [id:da10222491957386626] 
	\draw    (244.1,219.75) -- (244.1,161.15) ;
	\draw [shift={(244.1,159.15)}, rotate = 450] [color={rgb, 255:red, 0; green, 0; blue, 0 }  ][line width=0.75]    (10.93,-3.29) .. controls (6.95,-1.4) and (3.31,-0.3) .. (0,0) .. controls (3.31,0.3) and (6.95,1.4) .. (10.93,3.29)   ;
	
	%Shape: Rectangle [id:dp3211173615588092] 
	\draw   (324.5,202) -- (394.5,202) -- (394.5,242) -- (324.5,242) -- cycle ;
	%Straight Lines [id:da42233637174500327] 
	\draw    (244.1,219.75) -- (324.3,219.75) ;
	
	
	%Straight Lines [id:da9688852904709178] 
	\draw    (396.5,219.75) -- (451.3,219.75) ;
	
	\draw [shift={(394.5,219.75)}, rotate = 0] [color={rgb, 255:red, 0; green, 0; blue, 0 }  ][line width=0.75]    (10.93,-3.29) .. controls (6.95,-1.4) and (3.31,-0.3) .. (0,0) .. controls (3.31,0.3) and (6.95,1.4) .. (10.93,3.29)   ;
	%Straight Lines [id:da49473026564794287] 
	\draw    (451.3,143) -- (451.3,219.75) ;
	
	
	%Curve Lines [id:da6898335310358139] 
	\draw    (285.5,102) .. controls (288.37,113.46) and (287.58,121.27) .. (289.25,134.15) ;
	\draw [shift={(289.5,136)}, rotate = 261.87] [fill={rgb, 255:red, 0; green, 0; blue, 0 }  ][line width=0.75]  [draw opacity=0] (10.72,-5.15) -- (0,0) -- (10.72,5.15) -- (7.12,0) -- cycle    ;
	
	%Curve Lines [id:da7097329155318699] 
	\draw    (458.5,106) .. controls (451.92,117.28) and (454.18,123.26) .. (452.81,133.07) ;
	\draw [shift={(452.5,135)}, rotate = 280.3] [fill={rgb, 255:red, 0; green, 0; blue, 0 }  ][line width=0.75]  [draw opacity=0] (10.72,-5.15) -- (0,0) -- (10.72,5.15) -- (7.12,0) -- cycle    ;
	
	%Curve Lines [id:da6339380725476107] 
	\draw    (169.5,233) .. controls (172.43,187.17) and (193.41,169.87) .. (214.85,159.76) ;
	\draw [shift={(216.5,159)}, rotate = 515.56] [fill={rgb, 255:red, 0; green, 0; blue, 0 }  ][line width=0.75]  [draw opacity=0] (10.72,-5.15) -- (0,0) -- (10.72,5.15) -- (7.12,0) -- cycle    ;
	
	
	% Text Node
	\draw (359.5,143) node   {$G$};
	% Text Node
	\draw (359.5,222) node   {$H$};
	% Text Node
	\draw (177.5,124) node   {$u$};
	% Text Node
	\draw (482.5,123) node   {$y$};
	% Text Node
	\draw (220.5,129) node   {$+$};
	% Text Node
	\draw (233.5,163) node   {$-$};
	% Text Node
	\draw (192,246) node  [align=left] {{\small Nodo (sommatore/sotrattore)}};
	% Text Node
	\draw (459,95) node  [align=left] {punto di diramazione};
	% Text Node
	\draw (293,87) node  [align=left] {segnale attuatore};
	% Text Node
	\draw (305.5,127) node   {$e$};
	
	\end{tikzpicture}
\end{center}	
	con $ u $ segnale di riferimento, $ G $ controllore e $ H $ elemento di retroazione. Il segnale attuatore è: $ e(t)=u(t)-Hy(t) $
	
\subsection{Segnali di disturbo}

Possono esserci dei segnali di disturbo in entrata. In questo caso prima si trova $ y $ esclusivamente in funzione di $ u $ poi esclusivamente in funzione di $ d $. %TODO: controllare se vera

\begin{center}
	
	
	\tikzset{every picture/.style={line width=0.75pt}} %set default line width to 0.75pt        
	
	\begin{tikzpicture}[x=0.75pt,y=0.75pt,yscale=-1,xscale=1]
	%uncomment if require: \path (0,300); %set diagram left start at 0, and has height of 300
	
	%Shape: Rectangle [id:dp8002427712904623] 
	\draw   (280.5,107) -- (350.5,107) -- (350.5,147) -- (280.5,147) -- cycle ;
	%Straight Lines [id:da652142321436137] 
	\draw    (262.25,127) -- (278.5,127) ;
	\draw [shift={(280.5,127)}, rotate = 180] [color={rgb, 255:red, 0; green, 0; blue, 0 }  ][line width=0.75]    (10.93,-3.29) .. controls (6.95,-1.4) and (3.31,-0.3) .. (0,0) .. controls (3.31,0.3) and (6.95,1.4) .. (10.93,3.29)   ;
	
	%Straight Lines [id:da980428882327369] 
	\draw    (438.5,127) -- (489.3,127) ;
	\draw [shift={(491.3,127)}, rotate = 180] [color={rgb, 255:red, 0; green, 0; blue, 0 }  ][line width=0.75]    (10.93,-3.29) .. controls (6.95,-1.4) and (3.31,-0.3) .. (0,0) .. controls (3.31,0.3) and (6.95,1.4) .. (10.93,3.29)   ;
	
	%Flowchart: Connector [id:dp7103967563362967] 
	\draw   (229.95,127) .. controls (229.95,118.08) and (237.18,110.85) .. (246.1,110.85) .. controls (255.02,110.85) and (262.25,118.08) .. (262.25,127) .. controls (262.25,135.92) and (255.02,143.15) .. (246.1,143.15) .. controls (237.18,143.15) and (229.95,135.92) .. (229.95,127) -- cycle ;
	%Straight Lines [id:da39229442705370654] 
	\draw    (165.2,127) -- (227.95,127) ;
	\draw [shift={(229.95,127)}, rotate = 180] [color={rgb, 255:red, 0; green, 0; blue, 0 }  ][line width=0.75]    (10.93,-3.29) .. controls (6.95,-1.4) and (3.31,-0.3) .. (0,0) .. controls (3.31,0.3) and (6.95,1.4) .. (10.93,3.29)   ;
	
	%Straight Lines [id:da7835139153281818] 
	\draw    (246.1,203.75) -- (246.1,145.15) ;
	\draw [shift={(246.1,143.15)}, rotate = 450] [color={rgb, 255:red, 0; green, 0; blue, 0 }  ][line width=0.75]    (10.93,-3.29) .. controls (6.95,-1.4) and (3.31,-0.3) .. (0,0) .. controls (3.31,0.3) and (6.95,1.4) .. (10.93,3.29)   ;
	
	%Shape: Rectangle [id:dp14775199124860605] 
	\draw   (326.5,186) -- (396.5,186) -- (396.5,226) -- (326.5,226) -- cycle ;
	%Straight Lines [id:da3771415417056869] 
	\draw    (246.1,203.75) -- (326.3,203.75) ;
	
	
	%Straight Lines [id:da585328914471497] 
	\draw    (398.5,203.75) -- (453.3,203.75) ;
	
	\draw [shift={(396.5,203.75)}, rotate = 0] [color={rgb, 255:red, 0; green, 0; blue, 0 }  ][line width=0.75]    (10.93,-3.29) .. controls (6.95,-1.4) and (3.31,-0.3) .. (0,0) .. controls (3.31,0.3) and (6.95,1.4) .. (10.93,3.29)   ;
	%Straight Lines [id:da29894491742670404] 
	\draw    (453.3,127) -- (453.3,203.75) ;
	
	
	%Shape: Rectangle [id:dp719324064196792] 
	\draw   (368.5,107) -- (438.5,107) -- (438.5,147) -- (368.5,147) -- cycle ;
	%Straight Lines [id:da3308344355409414] 
	\draw    (350.5,127) -- (366.5,127) ;
	\draw [shift={(368.5,127)}, rotate = 180] [color={rgb, 255:red, 0; green, 0; blue, 0 }  ][line width=0.75]    (10.93,-3.29) .. controls (6.95,-1.4) and (3.31,-0.3) .. (0,0) .. controls (3.31,0.3) and (6.95,1.4) .. (10.93,3.29)   ;
	
	%Straight Lines [id:da19666294752409486] 
	\draw    (403.5,68) -- (403.5,105) ;
	\draw [shift={(403.5,107)}, rotate = 270] [color={rgb, 255:red, 0; green, 0; blue, 0 }  ][line width=0.75]    (10.93,-3.29) .. controls (6.95,-1.4) and (3.31,-0.3) .. (0,0) .. controls (3.31,0.3) and (6.95,1.4) .. (10.93,3.29)   ;
	
	
	% Text Node
	\draw (315.5,127) node   {$G$};
	% Text Node
	\draw (361.5,206) node   {$H$};
	% Text Node
	\draw (270.5,107) node   {$e$};
	% Text Node
	\draw (484.5,107) node   {$y$};
	% Text Node
	\draw (222.5,113) node   {$+$};
	% Text Node
	\draw (235.5,147) node   {$-$};
	% Text Node
	\draw (178.5,108) node   {$u$};
	% Text Node
	\draw (389.5,76) node   {$d$};
	
	
	\end{tikzpicture}
\end{center}

\section{Risoluzione di sistemi a Blocchi}
In caso di sistemi complessi è più facile ottenere il sistema risultante tramite trasformazioni di blocchi più piccoli.

\subsection{Blocchi in serie/cascata}
\begin{center}
	
	
	\tikzset{every picture/.style={line width=0.75pt}} %set default line width to 0.75pt        
	
	\begin{tikzpicture}[x=0.75pt,y=0.75pt,yscale=-1,xscale=1]
	%uncomment if require: \path (0,106); %set diagram left start at 0, and has height of 106
	
	%Shape: Rectangle [id:dp06254604466462199] 
	\draw   (112.5,24.5) -- (182.5,24.5) -- (182.5,64.5) -- (112.5,64.5) -- cycle ;
	%Shape: Rectangle [id:dp7300837861868514] 
	\draw   (228.5,24.5) -- (298.5,24.5) -- (298.5,64.5) -- (228.5,64.5) -- cycle ;
	%Straight Lines [id:da6782971117734584] 
	\draw    (182.5,44.5) -- (226.5,44.5) ;
	\draw [shift={(228.5,44.5)}, rotate = 180] [color={rgb, 255:red, 0; green, 0; blue, 0 }  ][line width=0.75]    (10.93,-3.29) .. controls (6.95,-1.4) and (3.31,-0.3) .. (0,0) .. controls (3.31,0.3) and (6.95,1.4) .. (10.93,3.29)   ;
	
	%Straight Lines [id:da6363591162519446] 
	\draw    (298.5,44.5) -- (342.5,44.5) ;
	\draw [shift={(344.5,44.5)}, rotate = 180] [color={rgb, 255:red, 0; green, 0; blue, 0 }  ][line width=0.75]    (10.93,-3.29) .. controls (6.95,-1.4) and (3.31,-0.3) .. (0,0) .. controls (3.31,0.3) and (6.95,1.4) .. (10.93,3.29)   ;
	
	%Straight Lines [id:da5764037506135491] 
	\draw    (66.5,44.5) -- (110.5,44.5) ;
	\draw [shift={(112.5,44.5)}, rotate = 180] [color={rgb, 255:red, 0; green, 0; blue, 0 }  ][line width=0.75]    (10.93,-3.29) .. controls (6.95,-1.4) and (3.31,-0.3) .. (0,0) .. controls (3.31,0.3) and (6.95,1.4) .. (10.93,3.29)   ;
	
	%Shape: Rectangle [id:dp8193844196572095] 
	\draw   (448.5,24.5) -- (518.5,24.5) -- (518.5,64.5) -- (448.5,64.5) -- cycle ;
	%Straight Lines [id:da28598519675474865] 
	\draw    (518.5,44.5) -- (562.5,44.5) ;
	\draw [shift={(564.5,44.5)}, rotate = 180] [color={rgb, 255:red, 0; green, 0; blue, 0 }  ][line width=0.75]    (10.93,-3.29) .. controls (6.95,-1.4) and (3.31,-0.3) .. (0,0) .. controls (3.31,0.3) and (6.95,1.4) .. (10.93,3.29)   ;
	
	%Straight Lines [id:da26764641536519096] 
	\draw    (402.5,44.5) -- (446.5,44.5) ;
	\draw [shift={(448.5,44.5)}, rotate = 180] [color={rgb, 255:red, 0; green, 0; blue, 0 }  ][line width=0.75]    (10.93,-3.29) .. controls (6.95,-1.4) and (3.31,-0.3) .. (0,0) .. controls (3.31,0.3) and (6.95,1.4) .. (10.93,3.29)   ;
	
	
	% Text Node
	\draw (147.5,44.5) node   {$G_{1}$};
	% Text Node
	\draw (263.5,44.5) node   {$G_{2}$};
	% Text Node
	\draw (483.5,44.5) node   {$G$};
	% Text Node
	\draw (373,42.67) node   {$\equiv $};
	% Text Node
	\draw (80,31.33) node   {$u$};
	% Text Node
	\draw (327,31.33) node   {$y$};
	% Text Node
	\draw (419.84,31.33) node   {$u$};
	% Text Node
	\draw (544.52,31.33) node   {$y$};
	
	
	\end{tikzpicture}
\end{center}

\[
	G=G_1 \cdot G_2
\]

\subsection{Blocchi in parallelo}
\begin{center}
	
	
	\tikzset{every picture/.style={line width=0.75pt}} %set default line width to 0.75pt        
	
	\begin{tikzpicture}[x=0.75pt,y=0.75pt,yscale=-1,xscale=1]
	%uncomment if require: \path (0,156.66665649414062); %set diagram left start at 0, and has height of 156.66665649414062
	
	%Shape: Rectangle [id:dp2537797370216679] 
	\draw   (153.5,26.5) -- (223.5,26.5) -- (223.5,66.5) -- (153.5,66.5) -- cycle ;
	%Shape: Rectangle [id:dp008849470703853113] 
	\draw   (153.5,96.5) -- (223.5,96.5) -- (223.5,136.5) -- (153.5,136.5) -- cycle ;
	%Straight Lines [id:da920241196233377] 
	\draw    (223.5,46.5) -- (267.5,46.5) ;
	\draw [shift={(269.5,46.5)}, rotate = 180] [color={rgb, 255:red, 0; green, 0; blue, 0 }  ][line width=0.75]    (10.93,-3.29) .. controls (6.95,-1.4) and (3.31,-0.3) .. (0,0) .. controls (3.31,0.3) and (6.95,1.4) .. (10.93,3.29)   ;
	
	%Straight Lines [id:da3956988512580457] 
	\draw    (301.8,46.5) -- (345.8,46.5) ;
	\draw [shift={(347.8,46.5)}, rotate = 180] [color={rgb, 255:red, 0; green, 0; blue, 0 }  ][line width=0.75]    (10.93,-3.29) .. controls (6.95,-1.4) and (3.31,-0.3) .. (0,0) .. controls (3.31,0.3) and (6.95,1.4) .. (10.93,3.29)   ;
	
	%Straight Lines [id:da45330941436301075] 
	\draw    (77.5,46.5) -- (151.5,46.5) ;
	\draw [shift={(153.5,46.5)}, rotate = 180] [color={rgb, 255:red, 0; green, 0; blue, 0 }  ][line width=0.75]    (10.93,-3.29) .. controls (6.95,-1.4) and (3.31,-0.3) .. (0,0) .. controls (3.31,0.3) and (6.95,1.4) .. (10.93,3.29)   ;
	
	%Shape: Rectangle [id:dp7875612791922739] 
	\draw   (453.5,65.5) -- (523.5,65.5) -- (523.5,105.5) -- (453.5,105.5) -- cycle ;
	%Straight Lines [id:da4998467873740766] 
	\draw    (523.5,85.5) -- (567.5,85.5) ;
	\draw [shift={(569.5,85.5)}, rotate = 180] [color={rgb, 255:red, 0; green, 0; blue, 0 }  ][line width=0.75]    (10.93,-3.29) .. controls (6.95,-1.4) and (3.31,-0.3) .. (0,0) .. controls (3.31,0.3) and (6.95,1.4) .. (10.93,3.29)   ;
	
	%Straight Lines [id:da869726945090177] 
	\draw    (407.5,85.5) -- (451.5,85.5) ;
	\draw [shift={(453.5,85.5)}, rotate = 180] [color={rgb, 255:red, 0; green, 0; blue, 0 }  ][line width=0.75]    (10.93,-3.29) .. controls (6.95,-1.4) and (3.31,-0.3) .. (0,0) .. controls (3.31,0.3) and (6.95,1.4) .. (10.93,3.29)   ;
	
	%Flowchart: Connector [id:dp37886477247374284] 
	\draw   (269.5,46.5) .. controls (269.5,37.58) and (276.73,30.35) .. (285.65,30.35) .. controls (294.57,30.35) and (301.8,37.58) .. (301.8,46.5) .. controls (301.8,55.42) and (294.57,62.65) .. (285.65,62.65) .. controls (276.73,62.65) and (269.5,55.42) .. (269.5,46.5) -- cycle ;
	%Straight Lines [id:da4643477020568181] 
	\draw    (285.65,116.5) -- (285.65,64.65) ;
	\draw [shift={(285.65,62.65)}, rotate = 450] [color={rgb, 255:red, 0; green, 0; blue, 0 }  ][line width=0.75]    (10.93,-3.29) .. controls (6.95,-1.4) and (3.31,-0.3) .. (0,0) .. controls (3.31,0.3) and (6.95,1.4) .. (10.93,3.29)   ;
	
	%Straight Lines [id:da15310891193572584] 
	\draw    (223.5,116.5) -- (285.65,116.5) ;
	
	
	%Straight Lines [id:da24531771565102178] 
	\draw    (115.5,116.5) -- (153.5,116.5) ;
	
	
	%Straight Lines [id:da6554585448291919] 
	\draw    (115.5,116.5) -- (115.5,46.5) ;
	
	
	
	% Text Node
	\draw (188.5,46.5) node   {$G_{1}$};
	% Text Node
	\draw (188.5,116.5) node   {$G_{2}$};
	% Text Node
	\draw (488.5,85.5) node   {$G$};
	% Text Node
	\draw (259.5,31) node   {$+$};
	% Text Node
	\draw (272.5,67) node   {$\pm $};
	% Text Node
	\draw (375,81) node   {$\equiv $};
	% Text Node
	\draw (423.84,72.33) node   {$u$};
	% Text Node
	\draw (548.52,72.33) node   {$y$};
	% Text Node
	\draw (88.91,33.33) node   {$u$};
	% Text Node
	\draw (326.28,33.33) node   {$y$};
	
	
	\end{tikzpicture}
\end{center}

\[
G=G_1 \pm G_2
\]


\subsection{Sistema retroazionato}
\begin{center}


\tikzset{every picture/.style={line width=0.75pt}} %set default line width to 0.75pt        

\begin{tikzpicture}[x=0.75pt,y=0.75pt,yscale=-1,xscale=1]
%uncomment if require: \path (0,171.66665649414062); %set diagram left start at 0, and has height of 171.66665649414062

%Shape: Rectangle [id:dp3240175156841656] 
\draw   (181.5,33.5) -- (251.5,33.5) -- (251.5,73.5) -- (181.5,73.5) -- cycle ;
%Shape: Rectangle [id:dp6608672243845539] 
\draw   (181.5,103.5) -- (251.5,103.5) -- (251.5,143.5) -- (181.5,143.5) -- cycle ;
%Straight Lines [id:da4887381227487937] 
\draw    (83.5,53.5) -- (127.5,53.5) ;
\draw [shift={(129.5,53.5)}, rotate = 180] [color={rgb, 255:red, 0; green, 0; blue, 0 }  ][line width=0.75]    (10.93,-3.29) .. controls (6.95,-1.4) and (3.31,-0.3) .. (0,0) .. controls (3.31,0.3) and (6.95,1.4) .. (10.93,3.29)   ;

%Straight Lines [id:da3576050474057568] 
\draw    (251.8,53.5) -- (316.5,53.5) ;
\draw [shift={(318.5,53.5)}, rotate = 180] [color={rgb, 255:red, 0; green, 0; blue, 0 }  ][line width=0.75]    (10.93,-3.29) .. controls (6.95,-1.4) and (3.31,-0.3) .. (0,0) .. controls (3.31,0.3) and (6.95,1.4) .. (10.93,3.29)   ;

%Straight Lines [id:da5570207424626268] 
\draw    (161.8,53.5) -- (179.5,53.5) ;
\draw [shift={(181.5,53.5)}, rotate = 180] [color={rgb, 255:red, 0; green, 0; blue, 0 }  ][line width=0.75]    (10.93,-3.29) .. controls (6.95,-1.4) and (3.31,-0.3) .. (0,0) .. controls (3.31,0.3) and (6.95,1.4) .. (10.93,3.29)   ;

%Shape: Rectangle [id:dp9512547434168017] 
\draw   (437.5,72.5) -- (507.5,72.5) -- (507.5,112.5) -- (437.5,112.5) -- cycle ;
%Straight Lines [id:da9547629996502616] 
\draw    (507.5,92.5) -- (551.5,92.5) ;
\draw [shift={(553.5,92.5)}, rotate = 180] [color={rgb, 255:red, 0; green, 0; blue, 0 }  ][line width=0.75]    (10.93,-3.29) .. controls (6.95,-1.4) and (3.31,-0.3) .. (0,0) .. controls (3.31,0.3) and (6.95,1.4) .. (10.93,3.29)   ;

%Straight Lines [id:da4329180015438556] 
\draw    (391.5,92.5) -- (435.5,92.5) ;
\draw [shift={(437.5,92.5)}, rotate = 180] [color={rgb, 255:red, 0; green, 0; blue, 0 }  ][line width=0.75]    (10.93,-3.29) .. controls (6.95,-1.4) and (3.31,-0.3) .. (0,0) .. controls (3.31,0.3) and (6.95,1.4) .. (10.93,3.29)   ;

%Flowchart: Connector [id:dp15383669633858665] 
\draw   (129.5,53.5) .. controls (129.5,44.58) and (136.73,37.35) .. (145.65,37.35) .. controls (154.57,37.35) and (161.8,44.58) .. (161.8,53.5) .. controls (161.8,62.42) and (154.57,69.65) .. (145.65,69.65) .. controls (136.73,69.65) and (129.5,62.42) .. (129.5,53.5) -- cycle ;
%Straight Lines [id:da4738477696597776] 
\draw    (145.65,123.5) -- (145.65,71.65) ;
\draw [shift={(145.65,69.65)}, rotate = 450] [color={rgb, 255:red, 0; green, 0; blue, 0 }  ][line width=0.75]    (10.93,-3.29) .. controls (6.95,-1.4) and (3.31,-0.3) .. (0,0) .. controls (3.31,0.3) and (6.95,1.4) .. (10.93,3.29)   ;

%Straight Lines [id:da24258510778213926] 
\draw    (251.5,123.5) -- (286.5,123.5) ;


%Straight Lines [id:da3046422054343556] 
\draw    (145.65,123.5) -- (181.5,123.5) ;


%Straight Lines [id:da9939745193683625] 
\draw    (286.5,123.5) -- (286.5,53.5) ;


%Curve Lines [id:da24015808420977258] 
\draw    (97,119.2) .. controls (104.64,110.6) and (121.4,106.57) .. (134.22,109.7) ;
\draw [shift={(136,110.2)}, rotate = 197.1] [fill={rgb, 255:red, 0; green, 0; blue, 0 }  ][line width=0.75]  [draw opacity=0] (10.72,-5.15) -- (0,0) -- (10.72,5.15) -- (7.12,0) -- cycle    ;

%Curve Lines [id:da6376172389939263] 
\draw    (166,26.2) .. controls (165.08,31.72) and (165.85,38.08) .. (169.1,44.52) ;
\draw [shift={(170,46.2)}, rotate = 240.26] [fill={rgb, 255:red, 0; green, 0; blue, 0 }  ][line width=0.75]  [draw opacity=0] (10.72,-5.15) -- (0,0) -- (10.72,5.15) -- (7.12,0) -- cycle    ;


% Text Node
\draw (216.5,53.5) node   {$G_{1}$};
% Text Node
\draw (216.5,123.5) node   {$H$};
% Text Node
\draw (472.5,92.5) node   {$G$};
% Text Node
\draw (119.5,38) node   {$+$};
% Text Node
\draw (132.5,74) node   {$\pm $};
% Text Node
\draw (361,89) node   {$\equiv $};
% Text Node
\draw (408.84,78.33) node   {$u$};
% Text Node
\draw (533.52,78.33) node   {$y$};
% Text Node
\draw (93.82,40.33) node   {$u$};
% Text Node
\draw (299.18,40.33) node   {$y$};
% Text Node
\draw (94.82,131.33) node   {$Hy$};
% Text Node
\draw (171.82,16.33) node   {$u\pm Hy$};


\end{tikzpicture}
\end{center}
\[
	y= G_1(u \pm Hy) 
	\Rightarrow y = G_1 u \pm G_1 Hy
	\Rightarrow y \mp G_1 Hy= G_1 u 
	\Rightarrow y(1\mp G_1 H)= G_1 u
	\Rightarrow y= \frac{G_1 u}{1\mp G_1 H}	
\]
\[
	G = \frac{G_1 }{1\mp G_1 H}
\]
%TODO controllare segni disegno e formula

\subsection{Retroazione unitaria}
\begin{center}
	
	
	\tikzset{every picture/.style={line width=0.75pt}} %set default line width to 0.75pt        
	
	\begin{tikzpicture}[x=0.75pt,y=0.75pt,yscale=-1,xscale=1]
	%uncomment if require: \path (0,142); %set diagram left start at 0, and has height of 142
	
	%Shape: Rectangle [id:dp7400774739258154] 
	\draw   (176.5,52.5) -- (246.5,52.5) -- (246.5,92.5) -- (176.5,92.5) -- cycle ;
	%Straight Lines [id:da932624228102461] 
	\draw    (78.5,72.5) -- (122.5,72.5) ;
	\draw [shift={(124.5,72.5)}, rotate = 180] [color={rgb, 255:red, 0; green, 0; blue, 0 }  ][line width=0.75]    (10.93,-3.29) .. controls (6.95,-1.4) and (3.31,-0.3) .. (0,0) .. controls (3.31,0.3) and (6.95,1.4) .. (10.93,3.29)   ;
	
	%Straight Lines [id:da4699870099733312] 
	\draw    (246.8,72.5) -- (311.5,72.5) ;
	\draw [shift={(313.5,72.5)}, rotate = 180] [color={rgb, 255:red, 0; green, 0; blue, 0 }  ][line width=0.75]    (10.93,-3.29) .. controls (6.95,-1.4) and (3.31,-0.3) .. (0,0) .. controls (3.31,0.3) and (6.95,1.4) .. (10.93,3.29)   ;
	
	%Straight Lines [id:da5969384975066614] 
	\draw    (156.8,72.5) -- (174.5,72.5) ;
	\draw [shift={(176.5,72.5)}, rotate = 180] [color={rgb, 255:red, 0; green, 0; blue, 0 }  ][line width=0.75]    (10.93,-3.29) .. controls (6.95,-1.4) and (3.31,-0.3) .. (0,0) .. controls (3.31,0.3) and (6.95,1.4) .. (10.93,3.29)   ;
	
	%Shape: Rectangle [id:dp3907855174239989] 
	\draw   (442.5,65.5) -- (512.5,65.5) -- (512.5,105.5) -- (442.5,105.5) -- cycle ;
	%Straight Lines [id:da24051365316018458] 
	\draw    (512.5,85.5) -- (556.5,85.5) ;
	\draw [shift={(558.5,85.5)}, rotate = 180] [color={rgb, 255:red, 0; green, 0; blue, 0 }  ][line width=0.75]    (10.93,-3.29) .. controls (6.95,-1.4) and (3.31,-0.3) .. (0,0) .. controls (3.31,0.3) and (6.95,1.4) .. (10.93,3.29)   ;
	
	%Straight Lines [id:da7089642197357304] 
	\draw    (396.5,85.5) -- (440.5,85.5) ;
	\draw [shift={(442.5,85.5)}, rotate = 180] [color={rgb, 255:red, 0; green, 0; blue, 0 }  ][line width=0.75]    (10.93,-3.29) .. controls (6.95,-1.4) and (3.31,-0.3) .. (0,0) .. controls (3.31,0.3) and (6.95,1.4) .. (10.93,3.29)   ;
	
	%Flowchart: Connector [id:dp8994365253854315] 
	\draw   (124.5,72.5) .. controls (124.5,63.58) and (131.73,56.35) .. (140.65,56.35) .. controls (149.57,56.35) and (156.8,63.58) .. (156.8,72.5) .. controls (156.8,81.42) and (149.57,88.65) .. (140.65,88.65) .. controls (131.73,88.65) and (124.5,81.42) .. (124.5,72.5) -- cycle ;
	%Straight Lines [id:da4632748540731406] 
	\draw    (140.65,117.33) -- (140.65,90.65) ;
	\draw [shift={(140.65,88.65)}, rotate = 450] [color={rgb, 255:red, 0; green, 0; blue, 0 }  ][line width=0.75]    (10.93,-3.29) .. controls (6.95,-1.4) and (3.31,-0.3) .. (0,0) .. controls (3.31,0.3) and (6.95,1.4) .. (10.93,3.29)   ;
	
	%Straight Lines [id:da9141929721528332] 
	\draw    (140.65,117.33) -- (281.5,117.33) ;
	
	
	%Straight Lines [id:da19831991360071943] 
	\draw    (281.5,117.33) -- (281.5,72.5) ;
	
	
	
	% Text Node
	\draw (211.5,72.5) node   {$G_{1}$};
	% Text Node
	\draw (477.5,85.5) node   {$G$};
	% Text Node
	\draw (114.5,57) node   {$+$};
	% Text Node
	\draw (127.5,93) node   {$\pm $};
	% Text Node
	\draw (362,84) node   {$\equiv $};
	% Text Node
	\draw (413.84,71.33) node   {$u$};
	% Text Node
	\draw (538.52,71.33) node   {$y$};
	% Text Node
	\draw (89.82,58.33) node   {$u$};
	% Text Node
	\draw (295.18,58.33) node   {$y$};
	
	
	\end{tikzpicture}
\end{center}
\[
G = \frac{G_1}{1\mp G_1}
\]

\subsection{Spostamento punto di prelievo}
\begin{center}
	
	
	\tikzset{every picture/.style={line width=0.75pt}} %set default line width to 0.75pt        
	
	\begin{tikzpicture}[x=0.75pt,y=0.75pt,yscale=-1,xscale=1]
	%uncomment if require: \path (0,373.34375); %set diagram left start at 0, and has height of 373.34375
	
	%Shape: Rectangle [id:dp310777363660222] 
	\draw   (133.5,70.83) -- (203.5,70.83) -- (203.5,110.83) -- (133.5,110.83) -- cycle ;
	%Straight Lines [id:da518533906470976] 
	\draw    (203.8,90.83) -- (268.5,90.83) ;
	\draw [shift={(270.5,90.83)}, rotate = 180] [color={rgb, 255:red, 0; green, 0; blue, 0 }  ][line width=0.75]    (10.93,-3.29) .. controls (6.95,-1.4) and (3.31,-0.3) .. (0,0) .. controls (3.31,0.3) and (6.95,1.4) .. (10.93,3.29)   ;
	
	%Straight Lines [id:da9728704924925071] 
	\draw    (99.5,90.83) -- (131.5,90.83) ;
	\draw [shift={(133.5,90.83)}, rotate = 180] [color={rgb, 255:red, 0; green, 0; blue, 0 }  ][line width=0.75]    (10.93,-3.29) .. controls (6.95,-1.4) and (3.31,-0.3) .. (0,0) .. controls (3.31,0.3) and (6.95,1.4) .. (10.93,3.29)   ;
	
	%Straight Lines [id:da849728541549386] 
	\draw    (101.5,135.67) -- (238.5,135.67) ;
	
	\draw [shift={(99.5,135.67)}, rotate = 0] [color={rgb, 255:red, 0; green, 0; blue, 0 }  ][line width=0.75]    (10.93,-3.29) .. controls (6.95,-1.4) and (3.31,-0.3) .. (0,0) .. controls (3.31,0.3) and (6.95,1.4) .. (10.93,3.29)   ;
	%Straight Lines [id:da849509594002716] 
	\draw    (238.5,135.67) -- (238.5,90.83) ;
	
	
	%Shape: Rectangle [id:dp0636525664044818] 
	\draw   (510,60.5) -- (580,60.5) -- (580,100.5) -- (510,100.5) -- cycle ;
	%Straight Lines [id:da9359886342204105] 
	\draw    (580,80.5) -- (624,80.5) ;
	\draw [shift={(626,80.5)}, rotate = 180] [color={rgb, 255:red, 0; green, 0; blue, 0 }  ][line width=0.75]    (10.93,-3.29) .. controls (6.95,-1.4) and (3.31,-0.3) .. (0,0) .. controls (3.31,0.3) and (6.95,1.4) .. (10.93,3.29)   ;
	
	%Straight Lines [id:da7528345880250529] 
	\draw    (373,80.5) -- (508,80.5) ;
	\draw [shift={(510,80.5)}, rotate = 180] [color={rgb, 255:red, 0; green, 0; blue, 0 }  ][line width=0.75]    (10.93,-3.29) .. controls (6.95,-1.4) and (3.31,-0.3) .. (0,0) .. controls (3.31,0.3) and (6.95,1.4) .. (10.93,3.29)   ;
	
	%Straight Lines [id:da17571556748329886] 
	\draw    (474,125.33) -- (490,125.33) ;
	
	\draw [shift={(472,125.33)}, rotate = 0] [color={rgb, 255:red, 0; green, 0; blue, 0 }  ][line width=0.75]    (10.93,-3.29) .. controls (6.95,-1.4) and (3.31,-0.3) .. (0,0) .. controls (3.31,0.3) and (6.95,1.4) .. (10.93,3.29)   ;
	%Straight Lines [id:da08197563882695591] 
	\draw    (490,125.33) -- (490,80.5) ;
	
	
	%Shape: Rectangle [id:dp012916245589069453] 
	\draw   (402,105.5) -- (472,105.5) -- (472,145.5) -- (402,145.5) -- cycle ;
	%Straight Lines [id:da6813463346049113] 
	\draw    (375,125.33) -- (402,125.33) ;
	
	\draw [shift={(373,125.33)}, rotate = 0] [color={rgb, 255:red, 0; green, 0; blue, 0 }  ][line width=0.75]    (10.93,-3.29) .. controls (6.95,-1.4) and (3.31,-0.3) .. (0,0) .. controls (3.31,0.3) and (6.95,1.4) .. (10.93,3.29)   ;
	%Shape: Rectangle [id:dp4045732441435752] 
	\draw   (155.5,196.33) -- (225.5,196.33) -- (225.5,236.33) -- (155.5,236.33) -- cycle ;
	
	%Straight Lines [id:da016496853100529174] 
	\draw    (225.5,216.33) -- (275.5,216.33) ;
	\draw [shift={(277.5,216.33)}, rotate = 180] [color={rgb, 255:red, 0; green, 0; blue, 0 }  ][line width=0.75]    (10.93,-3.29) .. controls (6.95,-1.4) and (3.31,-0.3) .. (0,0) .. controls (3.31,0.3) and (6.95,1.4) .. (10.93,3.29)   ;
	
	%Straight Lines [id:da7057615327790325] 
	\draw    (92.5,216.33) -- (153.5,216.33) ;
	\draw [shift={(155.5,216.33)}, rotate = 180] [color={rgb, 255:red, 0; green, 0; blue, 0 }  ][line width=0.75]    (10.93,-3.29) .. controls (6.95,-1.4) and (3.31,-0.3) .. (0,0) .. controls (3.31,0.3) and (6.95,1.4) .. (10.93,3.29)   ;
	
	%Straight Lines [id:da6322832188032141] 
	\draw    (124,261.17) -- (275.5,261.17) ;
	\draw [shift={(277.5,261.17)}, rotate = 180] [color={rgb, 255:red, 0; green, 0; blue, 0 }  ][line width=0.75]    (10.93,-4.9) .. controls (6.95,-2.3) and (3.31,-0.67) .. (0,0) .. controls (3.31,0.67) and (6.95,2.3) .. (10.93,4.9)   ;
	
	%Straight Lines [id:da5609261518862365] 
	\draw    (124,261.17) -- (124,216.33) ;
	
	
	%Shape: Rectangle [id:dp3821800884740554] 
	\draw   (391.5,186.5) -- (461.5,186.5) -- (461.5,226.5) -- (391.5,226.5) -- cycle ;
	
	%Straight Lines [id:da1809455253187493] 
	\draw    (461.5,206.5) -- (627.5,206.5) ;
	\draw [shift={(629.5,206.5)}, rotate = 180] [color={rgb, 255:red, 0; green, 0; blue, 0 }  ][line width=0.75]    (10.93,-3.29) .. controls (6.95,-1.4) and (3.31,-0.3) .. (0,0) .. controls (3.31,0.3) and (6.95,1.4) .. (10.93,3.29)   ;
	
	%Straight Lines [id:da7659461130654208] 
	\draw    (510.5,251.33) -- (482.5,251.33) ;
	
	\draw [shift={(512.5,251.33)}, rotate = 180] [color={rgb, 255:red, 0; green, 0; blue, 0 }  ][line width=0.75]    (10.93,-3.29) .. controls (6.95,-1.4) and (3.31,-0.3) .. (0,0) .. controls (3.31,0.3) and (6.95,1.4) .. (10.93,3.29)   ;
	%Straight Lines [id:da9860283239601588] 
	\draw    (482.5,251.33) -- (482.5,206.5) ;
	
	
	%Shape: Rectangle [id:dp04900183758979271] 
	\draw   (512.5,231.5) -- (582.5,231.5) -- (582.5,271.5) -- (512.5,271.5) -- cycle ;
	%Straight Lines [id:da7589011428064532] 
	\draw    (627.5,251.33) -- (582.5,251.33) ;
	
	\draw [shift={(629.5,251.33)}, rotate = 180] [color={rgb, 255:red, 0; green, 0; blue, 0 }  ][line width=0.75]    (10.93,-3.29) .. controls (6.95,-1.4) and (3.31,-0.3) .. (0,0) .. controls (3.31,0.3) and (6.95,1.4) .. (10.93,3.29)   ;
	%Straight Lines [id:da3280510664161822] 
	\draw    (369.5,206.5) -- (389.5,206.5) ;
	\draw [shift={(391.5,206.5)}, rotate = 180] [color={rgb, 255:red, 0; green, 0; blue, 0 }  ][line width=0.75]    (10.93,-3.29) .. controls (6.95,-1.4) and (3.31,-0.3) .. (0,0) .. controls (3.31,0.3) and (6.95,1.4) .. (10.93,3.29)   ;
	
	
	% Text Node
	\draw (168.5,90.83) node   {$G$};
	% Text Node
	\draw (117,75.83) node   {$x$};
	% Text Node
	\draw (117,120.83) node   {$y$};
	% Text Node
	\draw (258.5,75.83) node   {$y$};
	% Text Node
	\draw (322.5,96.25) node   {$\equiv $};
	% Text Node
	\draw (545,80.5) node   {$G$};
	% Text Node
	\draw (391,65) node   {$x$};
	% Text Node
	\draw (391,110.5) node   {$y$};
	% Text Node
	\draw (609,65) node   {$y$};
	% Text Node
	\draw (437,125.5) node   {$G$};
	% Text Node
	\draw (190.5,216.33) node   {$G$};
	% Text Node
	\draw (103.5,199.83) node   {$x$};
	% Text Node
	\draw (255,245.33) node   {$x$};
	% Text Node
	\draw (255,199.83) node   {$y$};
	% Text Node
	\draw (426.5,206.5) node   {$G$};
	% Text Node
	\draw (376.5,190.5) node   {$x$};
	% Text Node
	\draw (607.25,190.5) node   {$y$};
	% Text Node
	\draw (547.5,251.5) node   {$\frac{1}{G}$};
	% Text Node
	\draw (322.5,217.25) node   {$\equiv $};
	% Text Node
	\draw (607.25,235.58) node   {$x$};
	
	
	\end{tikzpicture}
\end{center}

\subsection{Spostamento nodi}
\begin{center}
	
	
	\tikzset{every picture/.style={line width=0.75pt}} %set default line width to 0.75pt        
	
	\begin{tikzpicture}[x=0.75pt,y=0.75pt,yscale=-1,xscale=1]
	%uncomment if require: \path (0,457); %set diagram left start at 0, and has height of 457
	
	%Shape: Rectangle [id:dp6135595090947077] 
	\draw   (79,92.33) -- (149,92.33) -- (149,132.33) -- (79,132.33) -- cycle ;
	%Straight Lines [id:da3626055804507693] 
	\draw    (200.15,112.33) -- (234,112.33) ;
	\draw [shift={(236,112.33)}, rotate = 180] [color={rgb, 255:red, 0; green, 0; blue, 0 }  ][line width=0.75]    (10.93,-3.29) .. controls (6.95,-1.4) and (3.31,-0.3) .. (0,0) .. controls (3.31,0.3) and (6.95,1.4) .. (10.93,3.29)   ;
	
	%Straight Lines [id:da41744824271159975] 
	\draw    (45,112.33) -- (77,112.33) ;
	\draw [shift={(79,112.33)}, rotate = 180] [color={rgb, 255:red, 0; green, 0; blue, 0 }  ][line width=0.75]    (10.93,-3.29) .. controls (6.95,-1.4) and (3.31,-0.3) .. (0,0) .. controls (3.31,0.3) and (6.95,1.4) .. (10.93,3.29)   ;
	
	%Straight Lines [id:da578409866482468] 
	\draw    (45,157.17) -- (184,157.17) ;
	
	
	%Straight Lines [id:da5740906504082484] 
	\draw    (184,157.17) -- (184,130.48) ;
	\draw [shift={(184,128.48)}, rotate = 450] [color={rgb, 255:red, 0; green, 0; blue, 0 }  ][line width=0.75]    (10.93,-3.29) .. controls (6.95,-1.4) and (3.31,-0.3) .. (0,0) .. controls (3.31,0.3) and (6.95,1.4) .. (10.93,3.29)   ;
	
	%Flowchart: Connector [id:dp20292908152923217] 
	\draw   (167.85,112.33) .. controls (167.85,103.41) and (175.08,96.18) .. (184,96.18) .. controls (192.92,96.18) and (200.15,103.41) .. (200.15,112.33) .. controls (200.15,121.25) and (192.92,128.48) .. (184,128.48) .. controls (175.08,128.48) and (167.85,121.25) .. (167.85,112.33) -- cycle ;
	%Straight Lines [id:da8700255313528944] 
	\draw    (149.3,112.33) -- (165.85,112.33) ;
	\draw [shift={(167.85,112.33)}, rotate = 180] [color={rgb, 255:red, 0; green, 0; blue, 0 }  ][line width=0.75]    (10.93,-3.29) .. controls (6.95,-1.4) and (3.31,-0.3) .. (0,0) .. controls (3.31,0.3) and (6.95,1.4) .. (10.93,3.29)   ;
	
	
	%Shape: Rectangle [id:dp009097652213897467] 
	\draw   (492.25,82.5) -- (562.25,82.5) -- (562.25,122.5) -- (492.25,122.5) -- cycle ;
	%Straight Lines [id:da8515328769624075] 
	\draw    (562.25,102.5) -- (606.25,102.5) ;
	\draw [shift={(608.25,102.5)}, rotate = 180] [color={rgb, 255:red, 0; green, 0; blue, 0 }  ][line width=0.75]    (10.93,-3.29) .. controls (6.95,-1.4) and (3.31,-0.3) .. (0,0) .. controls (3.31,0.3) and (6.95,1.4) .. (10.93,3.29)   ;
	
	%Straight Lines [id:da9961697033989048] 
	\draw    (467.4,102.5) -- (490.25,102.5) ;
	\draw [shift={(492.25,102.5)}, rotate = 180] [color={rgb, 255:red, 0; green, 0; blue, 0 }  ][line width=0.75]    (10.93,-3.29) .. controls (6.95,-1.4) and (3.31,-0.3) .. (0,0) .. controls (3.31,0.3) and (6.95,1.4) .. (10.93,3.29)   ;
	
	%Straight Lines [id:da669982976045326] 
	\draw    (433.25,147.33) -- (451.25,147.33) ;
	
	
	%Straight Lines [id:da508943180922476] 
	\draw    (451.25,147.33) -- (451.25,120.65) ;
	\draw [shift={(451.25,118.65)}, rotate = 450] [color={rgb, 255:red, 0; green, 0; blue, 0 }  ][line width=0.75]    (10.93,-3.29) .. controls (6.95,-1.4) and (3.31,-0.3) .. (0,0) .. controls (3.31,0.3) and (6.95,1.4) .. (10.93,3.29)   ;
	
	%Shape: Rectangle [id:dp34034711072843415] 
	\draw   (363.25,127.5) -- (433.25,127.5) -- (433.25,167.5) -- (363.25,167.5) -- cycle ;
	%Straight Lines [id:da9645474122482631] 
	\draw    (334.25,147.33) -- (361.25,147.33) ;
	\draw [shift={(363.25,147.33)}, rotate = 180] [color={rgb, 255:red, 0; green, 0; blue, 0 }  ][line width=0.75]    (10.93,-3.29) .. controls (6.95,-1.4) and (3.31,-0.3) .. (0,0) .. controls (3.31,0.3) and (6.95,1.4) .. (10.93,3.29)   ;
	
	%Flowchart: Connector [id:dp6193436001401342] 
	\draw   (435.1,102.5) .. controls (435.1,93.58) and (442.33,86.35) .. (451.25,86.35) .. controls (460.17,86.35) and (467.4,93.58) .. (467.4,102.5) .. controls (467.4,111.42) and (460.17,118.65) .. (451.25,118.65) .. controls (442.33,118.65) and (435.1,111.42) .. (435.1,102.5) -- cycle ;
	%Straight Lines [id:da6473663361423925] 
	\draw    (335.25,102.5) -- (433.1,102.5) ;
	\draw [shift={(435.1,102.5)}, rotate = 180] [color={rgb, 255:red, 0; green, 0; blue, 0 }  ][line width=0.75]    (10.93,-3.29) .. controls (6.95,-1.4) and (3.31,-0.3) .. (0,0) .. controls (3.31,0.3) and (6.95,1.4) .. (10.93,3.29)   ;
	
	
	%Shape: Rectangle [id:dp712303047260725] 
	\draw   (134.5,219.08) -- (204.5,219.08) -- (204.5,259.08) -- (134.5,259.08) -- cycle ;
	
	%Straight Lines [id:da9828635155497281] 
	\draw    (204.5,239.08) -- (254.5,239.08) ;
	\draw [shift={(256.5,239.08)}, rotate = 180] [color={rgb, 255:red, 0; green, 0; blue, 0 }  ][line width=0.75]    (10.93,-3.29) .. controls (6.95,-1.4) and (3.31,-0.3) .. (0,0) .. controls (3.31,0.3) and (6.95,1.4) .. (10.93,3.29)   ;
	
	%Straight Lines [id:da21237269037472162] 
	\draw    (103.15,239.08) -- (132.5,239.08) ;
	\draw [shift={(134.5,239.08)}, rotate = 180] [color={rgb, 255:red, 0; green, 0; blue, 0 }  ][line width=0.75]    (10.93,-3.29) .. controls (6.95,-1.4) and (3.31,-0.3) .. (0,0) .. controls (3.31,0.3) and (6.95,1.4) .. (10.93,3.29)   ;
	
	%Straight Lines [id:da011840201342099288] 
	\draw    (87,283.92) -- (25.5,283.92) ;
	
	
	%Straight Lines [id:da22460806935495237] 
	\draw    (87,283.92) -- (87,257.23) ;
	\draw [shift={(87,255.23)}, rotate = 450] [color={rgb, 255:red, 0; green, 0; blue, 0 }  ][line width=0.75]    (10.93,-3.29) .. controls (6.95,-1.4) and (3.31,-0.3) .. (0,0) .. controls (3.31,0.3) and (6.95,1.4) .. (10.93,3.29)   ;
	
	%Flowchart: Connector [id:dp26403356946412915] 
	\draw   (70.85,239.08) .. controls (70.85,230.16) and (78.08,222.93) .. (87,222.93) .. controls (95.92,222.93) and (103.15,230.16) .. (103.15,239.08) .. controls (103.15,248) and (95.92,255.23) .. (87,255.23) .. controls (78.08,255.23) and (70.85,248) .. (70.85,239.08) -- cycle ;
	%Straight Lines [id:da7513151428837639] 
	\draw    (24.5,239.08) -- (68.85,239.08) ;
	\draw [shift={(70.85,239.08)}, rotate = 180] [color={rgb, 255:red, 0; green, 0; blue, 0 }  ][line width=0.75]    (10.93,-3.29) .. controls (6.95,-1.4) and (3.31,-0.3) .. (0,0) .. controls (3.31,0.3) and (6.95,1.4) .. (10.93,3.29)   ;
	
	
	%Shape: Rectangle [id:dp14816381333465634] 
	\draw   (415.5,208.5) -- (485.5,208.5) -- (485.5,248.5) -- (415.5,248.5) -- cycle ;
	
	%Straight Lines [id:da24019931650280113] 
	\draw    (485.5,228.5) -- (511.85,228.5) ;
	\draw [shift={(513.85,228.5)}, rotate = 180] [color={rgb, 255:red, 0; green, 0; blue, 0 }  ][line width=0.75]    (10.93,-3.29) .. controls (6.95,-1.4) and (3.31,-0.3) .. (0,0) .. controls (3.31,0.3) and (6.95,1.4) .. (10.93,3.29)   ;
	
	%Straight Lines [id:da039464805055267504] 
	\draw    (530,273.63) -- (486.5,273.33) ;
	
	
	%Straight Lines [id:da08890940325580954] 
	\draw    (530,273.63) -- (530,246.65) ;
	\draw [shift={(530,244.65)}, rotate = 450] [color={rgb, 255:red, 0; green, 0; blue, 0 }  ][line width=0.75]    (10.93,-3.29) .. controls (6.95,-1.4) and (3.31,-0.3) .. (0,0) .. controls (3.31,0.3) and (6.95,1.4) .. (10.93,3.29)   ;
	
	%Shape: Rectangle [id:dp4150429324233136] 
	\draw   (416.5,253.5) -- (486.5,253.5) -- (486.5,293.5) -- (416.5,293.5) -- cycle ;
	%Straight Lines [id:da8006311284636038] 
	\draw    (414.5,273.33) -- (343.5,273.33) ;
	
	\draw [shift={(416.5,273.33)}, rotate = 180] [color={rgb, 255:red, 0; green, 0; blue, 0 }  ][line width=0.75]    (10.93,-3.29) .. controls (6.95,-1.4) and (3.31,-0.3) .. (0,0) .. controls (3.31,0.3) and (6.95,1.4) .. (10.93,3.29)   ;
	%Straight Lines [id:da5766908177542442] 
	\draw    (343.5,228.5) -- (413.5,228.5) ;
	\draw [shift={(415.5,228.5)}, rotate = 180] [color={rgb, 255:red, 0; green, 0; blue, 0 }  ][line width=0.75]    (10.93,-3.29) .. controls (6.95,-1.4) and (3.31,-0.3) .. (0,0) .. controls (3.31,0.3) and (6.95,1.4) .. (10.93,3.29)   ;
	
	%Flowchart: Connector [id:dp8405559687615076] 
	\draw   (513.85,228.5) .. controls (513.85,219.58) and (521.08,212.35) .. (530,212.35) .. controls (538.92,212.35) and (546.15,219.58) .. (546.15,228.5) .. controls (546.15,237.42) and (538.92,244.65) .. (530,244.65) .. controls (521.08,244.65) and (513.85,237.42) .. (513.85,228.5) -- cycle ;
	%Straight Lines [id:da9465923155315663] 
	\draw    (546.15,228.5) -- (597,228.5) ;
	\draw [shift={(599,228.5)}, rotate = 180] [color={rgb, 255:red, 0; green, 0; blue, 0 }  ][line width=0.75]    (10.93,-3.29) .. controls (6.95,-1.4) and (3.31,-0.3) .. (0,0) .. controls (3.31,0.3) and (6.95,1.4) .. (10.93,3.29)   ;
	
	
	
	% Text Node
	\draw (285.5,120.5) node   {$\equiv $};
	% Text Node
	\draw (285.5,246.5) node   {$\equiv $};
	% Text Node
	\draw (358.5,215) node   {$x$};
	% Text Node
	\draw (569.25,215) node   {$z$};
	% Text Node
	\draw (451.5,273.5) node   {$G$};
	% Text Node
	\draw (450.5,228.5) node   {$G$};
	% Text Node
	\draw (36.5,223.58) node   {$x$};
	% Text Node
	\draw (36.5,270.08) node   {$y$};
	% Text Node
	\draw (234,223.58) node   {$z$};
	% Text Node
	\draw (169.5,239.08) node   {$G$};
	% Text Node
	\draw (527.25,102.5) node   {$G$};
	% Text Node
	\draw (347.25,88) node   {$x$};
	% Text Node
	\draw (347.25,133.5) node   {$y$};
	% Text Node
	\draw (594.25,88) node   {$z$};
	% Text Node
	\draw (398.25,147.5) node   {$\frac{1}{G}$};
	% Text Node
	\draw (114,112.33) node   {$G$};
	% Text Node
	\draw (55.5,98.33) node   {$x$};
	% Text Node
	\draw (55.5,142.33) node   {$y$};
	% Text Node
	\draw (223,98.33) node   {$z$};
	% Text Node
	\draw (160,98.5) node   {$+$};
	% Text Node
	\draw (171,132.5) node   {$+$};
	% Text Node
	\draw (65.5,223.75) node   {$+$};
	% Text Node
	\draw (74.5,260.75) node   {$+$};
	% Text Node
	\draw (358.5,260) node   {$y$};
	% Text Node
	\draw (504.5,214) node   {$+$};
	% Text Node
	\draw (517.5,249) node   {$+$};
	% Text Node
	\draw (422.75,88) node   {$+$};
	% Text Node
	\draw (442.75,124) node   {$+$};
	
	
	\end{tikzpicture}
\end{center}

\section{Schemi a Flusso}

È una rappresentazione con linee orientate:

\begin{center}
	
	
	\tikzset{every picture/.style={line width=0.75pt}} %set default line width to 0.75pt        
	
	\begin{tikzpicture}[x=0.75pt,y=0.75pt,yscale=-1,xscale=1]
	%uncomment if require: \path (0,300); %set diagram left start at 0, and has height of 300
	
	%Straight Lines [id:da2645604679508091] 
	\draw    (100,127) -- (164,127) ;
	\draw [shift={(166,127)}, rotate = 180] [color={rgb, 255:red, 0; green, 0; blue, 0 }  ][line width=0.75]    (10.93,-3.29) .. controls (6.95,-1.4) and (3.31,-0.3) .. (0,0) .. controls (3.31,0.3) and (6.95,1.4) .. (10.93,3.29)   ;
	
	%Straight Lines [id:da5282887023752798] 
	\draw    (166,127) -- (219,127) ;
	
	
	%Flowchart: Connector [id:dp24093442897466288] 
	\draw  [fill={rgb, 255:red, 0; green, 0; blue, 0 }  ,fill opacity=1 ] (97.39,127) .. controls (97.39,125.56) and (98.56,124.39) .. (100,124.39) .. controls (101.44,124.39) and (102.61,125.56) .. (102.61,127) .. controls (102.61,128.44) and (101.44,129.61) .. (100,129.61) .. controls (98.56,129.61) and (97.39,128.44) .. (97.39,127) -- cycle ;
	%Flowchart: Connector [id:dp09137978863303431] 
	\draw  [fill={rgb, 255:red, 0; green, 0; blue, 0 }  ,fill opacity=1 ] (216.39,127) .. controls (216.39,125.56) and (217.56,124.39) .. (219,124.39) .. controls (220.44,124.39) and (221.61,125.56) .. (221.61,127) .. controls (221.61,128.44) and (220.44,129.61) .. (219,129.61) .. controls (217.56,129.61) and (216.39,128.44) .. (216.39,127) -- cycle ;
	%Straight Lines [id:da5682374240056367] 
	\draw    (263,127) -- (327,127) ;
	\draw [shift={(329,127)}, rotate = 180] [color={rgb, 255:red, 0; green, 0; blue, 0 }  ][line width=0.75]    (10.93,-3.29) .. controls (6.95,-1.4) and (3.31,-0.3) .. (0,0) .. controls (3.31,0.3) and (6.95,1.4) .. (10.93,3.29)   ;
	
	%Straight Lines [id:da9888198098011782] 
	\draw    (329,127) -- (382,127) ;
	
	
	%Flowchart: Connector [id:dp31773307782267945] 
	\draw  [fill={rgb, 255:red, 0; green, 0; blue, 0 }  ,fill opacity=1 ] (260.39,127) .. controls (260.39,125.56) and (261.56,124.39) .. (263,124.39) .. controls (264.44,124.39) and (265.61,125.56) .. (265.61,127) .. controls (265.61,128.44) and (264.44,129.61) .. (263,129.61) .. controls (261.56,129.61) and (260.39,128.44) .. (260.39,127) -- cycle ;
	%Flowchart: Connector [id:dp04076917753662146] 
	\draw  [fill={rgb, 255:red, 0; green, 0; blue, 0 }  ,fill opacity=1 ] (379.39,127) .. controls (379.39,125.56) and (380.56,124.39) .. (382,124.39) .. controls (383.44,124.39) and (384.61,125.56) .. (384.61,127) .. controls (384.61,128.44) and (383.44,129.61) .. (382,129.61) .. controls (380.56,129.61) and (379.39,128.44) .. (379.39,127) -- cycle ;
	%Straight Lines [id:da09323604628620097] 
	\draw    (382,127) -- (446,127) ;
	\draw [shift={(448,127)}, rotate = 180] [color={rgb, 255:red, 0; green, 0; blue, 0 }  ][line width=0.75]    (10.93,-3.29) .. controls (6.95,-1.4) and (3.31,-0.3) .. (0,0) .. controls (3.31,0.3) and (6.95,1.4) .. (10.93,3.29)   ;
	
	%Straight Lines [id:da08735784921599854] 
	\draw    (448,127) -- (501,127) ;
	
	
	%Flowchart: Connector [id:dp6288301261150899] 
	\draw  [fill={rgb, 255:red, 0; green, 0; blue, 0 }  ,fill opacity=1 ] (498.39,127) .. controls (498.39,125.56) and (499.56,124.39) .. (501,124.39) .. controls (502.44,124.39) and (503.61,125.56) .. (503.61,127) .. controls (503.61,128.44) and (502.44,129.61) .. (501,129.61) .. controls (499.56,129.61) and (498.39,128.44) .. (498.39,127) -- cycle ;
	
	% Text Node
	\draw (162.33,150.83) node   {$A_{ij}$};
	% Text Node
	\draw (100.5,109.5) node   {$x_{i}$};
	% Text Node
	\draw (222,109.5) node   {$x_{j}$};
	% Text Node
	\draw (325.33,150.83) node   {$A_{21}$};
	% Text Node
	\draw (263.5,109.5) node   {$x_{1}$};
	% Text Node
	\draw (382.92,109.5) node   {$x_{2}$};
	% Text Node
	\draw (502.33,109.5) node   {$x_{3}$};
	% Text Node
	\draw (440,150.83) node   {$A_{32}$};
	% Text Node
	\draw (387,172) node   {$A=A_{32} \cdotp A_{21}$};
	
	
	\end{tikzpicture}
\end{center}

$ x_j, x_i $ sono i nodi, $ A_ij $ è la trasmittanza %TODO è corretto il nome?
con il primo indice l'indice del nodo di arrivo e con secondo quello di partenza

\begin{definizione}
	Cammino: qualsiasi tipo di percorso che non passa più di una volta in un nodo (inizio e fine possono essere lo stesso nodo)
	
	\begin{center}
		
		
		\tikzset{every picture/.style={line width=0.75pt}} %set default line width to 0.75pt        
		
		\begin{tikzpicture}[x=0.75pt,y=0.75pt,yscale=-1,xscale=1]
		%uncomment if require: \path (0,300); %set diagram left start at 0, and has height of 300
		
		%Straight Lines [id:da8105771101337726] 
		\draw    (128,137) -- (192,137) ;
		\draw [shift={(194,137)}, rotate = 180] [color={rgb, 255:red, 0; green, 0; blue, 0 }  ][line width=0.75]    (10.93,-3.29) .. controls (6.95,-1.4) and (3.31,-0.3) .. (0,0) .. controls (3.31,0.3) and (6.95,1.4) .. (10.93,3.29)   ;
		
		%Straight Lines [id:da3382037954225505] 
		\draw    (194,137) -- (247,137) ;
		
		
		%Flowchart: Connector [id:dp4469951295676833] 
		\draw  [fill={rgb, 255:red, 0; green, 0; blue, 0 }  ,fill opacity=1 ] (125.39,137) .. controls (125.39,135.56) and (126.56,134.39) .. (128,134.39) .. controls (129.44,134.39) and (130.61,135.56) .. (130.61,137) .. controls (130.61,138.44) and (129.44,139.61) .. (128,139.61) .. controls (126.56,139.61) and (125.39,138.44) .. (125.39,137) -- cycle ;
		%Flowchart: Connector [id:dp15866470914000907] 
		\draw  [fill={rgb, 255:red, 0; green, 0; blue, 0 }  ,fill opacity=1 ] (244.39,137) .. controls (244.39,135.56) and (245.56,134.39) .. (247,134.39) .. controls (248.44,134.39) and (249.61,135.56) .. (249.61,137) .. controls (249.61,138.44) and (248.44,139.61) .. (247,139.61) .. controls (245.56,139.61) and (244.39,138.44) .. (244.39,137) -- cycle ;
		%Straight Lines [id:da3380953131659412] 
		\draw    (247,137) -- (311,137) ;
		\draw [shift={(313,137)}, rotate = 180] [color={rgb, 255:red, 0; green, 0; blue, 0 }  ][line width=0.75]    (10.93,-3.29) .. controls (6.95,-1.4) and (3.31,-0.3) .. (0,0) .. controls (3.31,0.3) and (6.95,1.4) .. (10.93,3.29)   ;
		
		%Straight Lines [id:da2778079775407136] 
		\draw    (313,137) -- (366,137) ;
		
		
		%Flowchart: Connector [id:dp24437453556375677] 
		\draw  [fill={rgb, 255:red, 0; green, 0; blue, 0 }  ,fill opacity=1 ] (363.39,137) .. controls (363.39,135.56) and (364.56,134.39) .. (366,134.39) .. controls (367.44,134.39) and (368.61,135.56) .. (368.61,137) .. controls (368.61,138.44) and (367.44,139.61) .. (366,139.61) .. controls (364.56,139.61) and (363.39,138.44) .. (363.39,137) -- cycle ;
		%Straight Lines [id:da8283013982720953] 
		\draw    (366,137) -- (430,137) ;
		\draw [shift={(432,137)}, rotate = 180] [color={rgb, 255:red, 0; green, 0; blue, 0 }  ][line width=0.75]    (10.93,-3.29) .. controls (6.95,-1.4) and (3.31,-0.3) .. (0,0) .. controls (3.31,0.3) and (6.95,1.4) .. (10.93,3.29)   ;
		
		%Straight Lines [id:da24499908355444422] 
		\draw    (432,137) -- (485,137) ;
		
		
		%Flowchart: Connector [id:dp5096754872900153] 
		\draw  [fill={rgb, 255:red, 0; green, 0; blue, 0 }  ,fill opacity=1 ] (482.39,137) .. controls (482.39,135.56) and (483.56,134.39) .. (485,134.39) .. controls (486.44,134.39) and (487.61,135.56) .. (487.61,137) .. controls (487.61,138.44) and (486.44,139.61) .. (485,139.61) .. controls (483.56,139.61) and (482.39,138.44) .. (482.39,137) -- cycle ;
		%Curve Lines [id:da05030671031007761] 
		\draw    (307.38,195.7) .. controls (345.83,195.06) and (366,178.91) .. (366,139.61) ;
		
		\draw [shift={(305,195.71)}, rotate = 0] [color={rgb, 255:red, 0; green, 0; blue, 0 }  ][line width=0.75]    (10.93,-3.29) .. controls (6.95,-1.4) and (3.31,-0.3) .. (0,0) .. controls (3.31,0.3) and (6.95,1.4) .. (10.93,3.29)   ;
		%Curve Lines [id:da9411729661125996] 
		\draw    (305,195.71) .. controls (269,193.71) and (247,179.71) .. (247,139.61) ;
		
		
		%Curve Lines [id:da8637716251821868] 
		\draw    (485,134.39) .. controls (476.39,95.11) and (446,45.71) .. (366,47.71) ;
		
		
		%Curve Lines [id:da1866156970628461] 
		\draw    (368.61,137) .. controls (399,128.71) and (402,89.71) .. (370,88.71) ;
		
		
		%Curve Lines [id:da1453137619761551] 
		\draw    (364,47.74) .. controls (319.62,48.92) and (242.07,88.41) .. (247,134.39) ;
		
		\draw [shift={(366,47.71)}, rotate = 180] [color={rgb, 255:red, 0; green, 0; blue, 0 }  ][line width=0.75]    (10.93,-3.29) .. controls (6.95,-1.4) and (3.31,-0.3) .. (0,0) .. controls (3.31,0.3) and (6.95,1.4) .. (10.93,3.29)   ;
		%Curve Lines [id:da7371522355149234] 
		\draw    (367.76,88.69) .. controls (319.87,88.74) and (340.78,127.68) .. (363.39,137) ;
		
		\draw [shift={(370,88.71)}, rotate = 181.32] [color={rgb, 255:red, 0; green, 0; blue, 0 }  ][line width=0.75]    (10.93,-3.29) .. controls (6.95,-1.4) and (3.31,-0.3) .. (0,0) .. controls (3.31,0.3) and (6.95,1.4) .. (10.93,3.29)   ;
		
		% Text Node
		\draw (190.33,160.83) node   {$A_{21}$};
		% Text Node
		\draw (128.5,120) node   {$x_{1}$};
		% Text Node
		\draw (233.92,120) node   {$x_{2}$};
		% Text Node
		\draw (367.33,120) node   {$x_{3}$};
		% Text Node
		\draw (305,160.83) node   {$A_{32}$};
		% Text Node
		\draw (496.33,120) node   {$x_{4}$};
		% Text Node
		\draw (426,160.83) node   {$A_{32}$};
		% Text Node
		\draw (306.6,217.63) node   {$A_{23}$};
		% Text Node
		\draw (357.33,28.83) node   {$A_{42}$};
		% Text Node
		\draw (372,73.5) node   {$A_{33}$};
		
		
		\end{tikzpicture}
	\end{center}
	
	Esempio di cammini:
	\begin{enumerate}
		\item $ A_{21},A_{32},A_{43} $ (cammino aperto)
		\item $ A_{21},A_{42} $
		\item $ A_{32},A_{23} $ (cammino chiuso/anello)
		\item $ A_{33} $ (autoanello)
	\end{enumerate}
	
	
	
	\tikzset{every picture/.style={line width=0.75pt}} %set default line width to 0.75pt        
	
	\begin{tikzpicture}[x=0.75pt,y=0.75pt,yscale=-1,xscale=1]
	%uncomment if require: \path (0,300); %set diagram left start at 0, and has height of 300
	
	%Shape: Rectangle [id:dp5584805448505255] 
	\draw   (133.5,55.07) -- (203.5,55.07) -- (203.5,95.07) -- (133.5,95.07) -- cycle ;
	%Shape: Rectangle [id:dp4639169675813819] 
	\draw   (133.5,125.07) -- (203.5,125.07) -- (203.5,165.07) -- (133.5,165.07) -- cycle ;
	%Straight Lines [id:da8973299402509956] 
	\draw    (35.5,75.07) -- (79.5,75.07) ;
	\draw [shift={(81.5,75.07)}, rotate = 180] [color={rgb, 255:red, 0; green, 0; blue, 0 }  ][line width=0.75]    (10.93,-3.29) .. controls (6.95,-1.4) and (3.31,-0.3) .. (0,0) .. controls (3.31,0.3) and (6.95,1.4) .. (10.93,3.29)   ;
	
	%Straight Lines [id:da21170135088038822] 
	\draw    (203.8,75.07) -- (268.5,75.07) ;
	\draw [shift={(270.5,75.07)}, rotate = 180] [color={rgb, 255:red, 0; green, 0; blue, 0 }  ][line width=0.75]    (10.93,-3.29) .. controls (6.95,-1.4) and (3.31,-0.3) .. (0,0) .. controls (3.31,0.3) and (6.95,1.4) .. (10.93,3.29)   ;
	
	%Straight Lines [id:da5723083322077416] 
	\draw    (113.8,75.07) -- (131.5,75.07) ;
	\draw [shift={(133.5,75.07)}, rotate = 180] [color={rgb, 255:red, 0; green, 0; blue, 0 }  ][line width=0.75]    (10.93,-3.29) .. controls (6.95,-1.4) and (3.31,-0.3) .. (0,0) .. controls (3.31,0.3) and (6.95,1.4) .. (10.93,3.29)   ;
	
	%Flowchart: Connector [id:dp7852522645677837] 
	\draw   (81.5,75.07) .. controls (81.5,66.15) and (88.73,58.92) .. (97.65,58.92) .. controls (106.57,58.92) and (113.8,66.15) .. (113.8,75.07) .. controls (113.8,83.99) and (106.57,91.22) .. (97.65,91.22) .. controls (88.73,91.22) and (81.5,83.99) .. (81.5,75.07) -- cycle ;
	%Straight Lines [id:da6466686716461845] 
	\draw    (97.65,145.07) -- (97.65,93.22) ;
	\draw [shift={(97.65,91.22)}, rotate = 450] [color={rgb, 255:red, 0; green, 0; blue, 0 }  ][line width=0.75]    (10.93,-3.29) .. controls (6.95,-1.4) and (3.31,-0.3) .. (0,0) .. controls (3.31,0.3) and (6.95,1.4) .. (10.93,3.29)   ;
	
	%Straight Lines [id:da05815677457067303] 
	\draw    (203.5,145.07) -- (238.5,145.07) ;
	
	
	%Straight Lines [id:da3233505457901875] 
	\draw    (97.65,145.07) -- (133.5,145.07) ;
	
	
	%Straight Lines [id:da01361016899977252] 
	\draw    (238.5,145.07) -- (238.5,75.07) ;
	
	
	
	%Straight Lines [id:da9489064314757574] 
	\draw    (335.35,79.58) -- (367.54,79.58) ;
	\draw [shift={(369.54,79.58)}, rotate = 180] [color={rgb, 255:red, 0; green, 0; blue, 0 }  ][line width=0.75]    (10.93,-3.29) .. controls (6.95,-1.4) and (3.31,-0.3) .. (0,0) .. controls (3.31,0.3) and (6.95,1.4) .. (10.93,3.29)   ;
	
	%Straight Lines [id:da023992851870582532] 
	\draw    (369.54,79.58) -- (397,79.58) ;
	
	
	%Flowchart: Connector [id:dp18813000644471334] 
	\draw  [fill={rgb, 255:red, 0; green, 0; blue, 0 }  ,fill opacity=1 ] (394.39,79.58) .. controls (394.39,78.14) and (395.56,76.98) .. (397,76.98) .. controls (398.44,76.98) and (399.61,78.14) .. (399.61,79.58) .. controls (399.61,81.02) and (398.44,82.19) .. (397,82.19) .. controls (395.56,82.19) and (394.39,81.02) .. (394.39,79.58) -- cycle ;
	%Straight Lines [id:da04318951705463547] 
	\draw    (397,79.58) -- (461,79.58) ;
	\draw [shift={(463,79.58)}, rotate = 180] [color={rgb, 255:red, 0; green, 0; blue, 0 }  ][line width=0.75]    (10.93,-3.29) .. controls (6.95,-1.4) and (3.31,-0.3) .. (0,0) .. controls (3.31,0.3) and (6.95,1.4) .. (10.93,3.29)   ;
	
	%Straight Lines [id:da1911166961256583] 
	\draw    (463,79.58) -- (516,79.58) ;
	
	
	%Flowchart: Connector [id:dp14709331014549343] 
	\draw  [fill={rgb, 255:red, 0; green, 0; blue, 0 }  ,fill opacity=1 ] (513.39,79.58) .. controls (513.39,78.14) and (514.56,76.98) .. (516,76.98) .. controls (517.44,76.98) and (518.61,78.14) .. (518.61,79.58) .. controls (518.61,81.02) and (517.44,82.19) .. (516,82.19) .. controls (514.56,82.19) and (513.39,81.02) .. (513.39,79.58) -- cycle ;
	%Straight Lines [id:da9901825078822903] 
	\draw    (516,79.58) -- (551.18,79.58) ;
	\draw [shift={(553.18,79.58)}, rotate = 180] [color={rgb, 255:red, 0; green, 0; blue, 0 }  ][line width=0.75]    (10.93,-3.29) .. controls (6.95,-1.4) and (3.31,-0.3) .. (0,0) .. controls (3.31,0.3) and (6.95,1.4) .. (10.93,3.29)   ;
	
	%Straight Lines [id:da9417777255935527] 
	\draw    (553.18,79.58) -- (583.03,79.58) ;
	
	
	%Curve Lines [id:da8849773075030309] 
	\draw    (457.38,138.28) .. controls (495.83,137.65) and (516,121.5) .. (516,82.19) ;
	
	\draw [shift={(455,138.3)}, rotate = 0] [color={rgb, 255:red, 0; green, 0; blue, 0 }  ][line width=0.75]    (10.93,-3.29) .. controls (6.95,-1.4) and (3.31,-0.3) .. (0,0) .. controls (3.31,0.3) and (6.95,1.4) .. (10.93,3.29)   ;
	%Curve Lines [id:da11116878216013837] 
	\draw    (455,138.3) .. controls (419,136.3) and (397,122.3) .. (397,82.19) ;
	
	
	
	
	% Text Node
	\draw (168.5,75.07) node   {$G_{1}$};
	% Text Node
	\draw (168.5,145.07) node   {$H$};
	% Text Node
	\draw (71.5,59.57) node   {$+$};
	% Text Node
	\draw (84.5,95.57) node   {$\pm $};
	% Text Node
	\draw (45.82,61.9) node   {$u$};
	% Text Node
	\draw (251.18,61.9) node   {$y$};
	% Text Node
	\draw (456,58.42) node   {$G$};
	% Text Node
	\draw (457.6,155.72) node   {$\pm H$};
	% Text Node
	\draw (304,98) node   {$\equiv $};
	
	
	\end{tikzpicture}
	
\end{definizione}
	
	


\documentclass[a4paper]{report}
\usepackage[T1]{fontenc}
\usepackage[utf8]{inputenc}
%\usepackage[]{babel}
\usepackage{mathrsfs}
\usepackage{amsthm}
\usepackage{amsmath}
\usepackage{amsfonts}
\begin{document}
	\section{Esercizi}
	
	\subsection{Esercizio 1}
	\begin{equation}
		\ddot{v}(t) + 3\dot{v}(t)+2v(t) = \frac{1}{3}u(t)
	\end{equation}
	
\end{document}
\include{./capitoli/AppB_Complessi}

\end{document}