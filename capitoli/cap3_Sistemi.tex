\chapter{Sistemi}
\section{Sistemi a tempo continuo}
%TODO immagine e descrizione?
\section{Proprietà dei sistemi a tempo continuo}
\subsection{Linearità}\label{sist_prop_Lin}  
	\[
	\alpha u_1+\beta u_1 \overset{\text{sistema}}{\longmapsto} \alpha v_1+\beta v_2
	\]
	dove $ u_1 \overset{\text{sistema}}{\longmapsto} v_1$ e $u_2\overset{\text{sistema}}{\longmapsto} v_2$
	
\subsection{Tempo invarianza} \label{sist_prop_TIn}
	\[
	u(t-\tau) \overset{\text{sistema}}{\longmapsto} v(t-\tau)
	\]
	\begin{definition}
		I sistemi che soddisfano \ref{sist_prop_Lin} e \ref{sist_prop_TIn} li chiameremo sistemi LTI
	\end{definition}

 \subsection{Causalità (Sistema non anticipatorio):} \label{sist_prop_cau}
	
	L'effetto non precede la causa (la causa anticipa l'effetto)
	
	\begin{oss}
		Ci sarà sempre un istante iniziale $t_0$ a partire da cui studieremo il sistema
		%TODO non sono sicuro
	\end{oss}
		\[
		\Leftrightarrow \exists \,t_0 \in \mathbb{R} \quad t.c. \quad u(t)=0 \quad t<t_0
		\Rightarrow v(t)=0 \quad t<t_0
		\quad t_0 =0 \quad \text{per Tempo Invarianza (TI)}
	\]
	%TODO aggiungere disegno?
\subsection{Stabilità BIBO (Bounded Input Bounded Output)}\label{sist_prop_BIBOstab}
	
	Per sistemi causali (da un entrata limitata c'è un' uscita limitata)
	\[
	\begin{split}
		\forall t \in [t_0, + \infty) \subseteq \mathbb{R} \qquad \exists \, M_u >0 \quad t.c. \quad \lvert u(t) \rvert < M_u \\
		\Rightarrow \exists \, M_v > 0 \quad t.c. \quad \forall t \in [t_0, + \infty) \quad t.c. \quad \lvert v(t) \rvert < M_v\\
	\end{split}		
	\]
	%TODO aggiungere disegno
	
\subsection{Stabilità asintotica}
	Se $\exists \, t_0 \quad t.c. \quad u(t)=0, \quad \forall t \ge t_0 \Rightarrow \lim_{t \to +\infty} v(t)=0$\\
	In assenza di sollecitazioni in ingresso l'uscita converge a zero asintoticamente: $\lim_{t \to \infty} v(t)=0$
	%TODO: aggiungere immagini?
	
%TODO esempi di modelli di equaz. differenziali

	
\section{Modelli}

In generale un modello SISO è descritto da un'equazione differenziale a coefficienti costanti di ordine $n$.
%TODO aggiungiamo di nuovo la formula 1?
\begin{equation}
\begin{split}
	a_n \frac{d^{(n)} v(t)}{dt^n} 
		+ a_{n-1} \frac{d^{(n-1)} v(t)}{dt^{n-1}} 
		+ \dots 
		+ a_1 \frac{dv(t)}{dt} 
		+ a_0\,v(t)\\
	= b_m \frac{d^{(m)} u(t)}{dt^m}
		+ b_{m-1} \frac{d^{(m-1)} u(t)}{dt^{m-1}} 
		+ \dots
		+ b_1 \frac{du(t)}{dt} 
		+ b_0\,u(t)\\
	a_n, b_m \ne 0	
\end{split}
	%TODO sarebbe carino dividerla decentemente	
\tag{2}\label{equation 2}
\end{equation}

\begin{oss}
	Nei casi pratici $n \ge m$	
\end{oss}

\begin{definition}
	I sistemi in cui $n \ge m$ si chiamano propri. Se $n > m$ il sistema si chiamerà strettamente proprio.
	%TODO enfatizzare proprio e strettametne proprio
\end{definition}

\begin{oss}
	In generale l'uscita $v(t)$ del sistema % TODO: aggiungere link all'equazione (1) %
	non è univoca. Se definiamo $n$ condizioni iniziali all'istante $t_0 = 0^-$ (per l'uscita $v$):
	\begin{equation}
		v(0^-) \,
		,\, \frac{dv(t)}{dt}\bigg\vert_{t=0^-} \,
		,\, \dots\,
		,\, \frac{d^{(n-1)}v(t)}{dt^{n-1}}\bigg\vert_{t=0^-}
		\tag{3}\label{equation 3}
	\end{equation}
	$\Rightarrow$ la soluzione di % TODO: aggiungere link all'equazione (1)
	è unica.
\end{oss}

La soluzione dell'equazione % TODO: aggiungere link all'equazione (1)
si ottiene come somma di una soluzione che dipende soltanto dalle condizioni iniiziali (\ref{equation 3}) chiamato evoluzione libera $v_l(t)$ e da una soluzione che dipende soltanto dall'ingresso $u(t)$ con condizioni iniziali tutte nulle, chiamato evoluzione forzata $v_f(t)$ %TODO sottolineare evoluzione forzata e libera
\[
v(t) = v_l(t) + v_f(t)
\]
L'evoluzione libera si ottiene dall'equazione omogenea associata
\begin{equation}
	a_n \frac{d^{(n)} v(t)}{dt^n} 
		+ a_{n-1} \frac{d^{(n-1)} v(t)}{dt^{n-1}} 
		+ \dots 
		+ a_1 \frac{dv(t)}{dt} 
		+ a_0\,v(t)
	= 0
	\tag{4}\label{equation 4}
\end{equation}
e le condizioni iniziali (\ref{equation 3}).

Associamo a (\ref{equation 4}) l'equazione caratteristica. Otteniamo il polinomio caratteristico: %TODO enfatizza polinomio caratteristico
\begin{equation}
	a_n s^n 
	+ a_{n-1} s^{n-1}
	+ \dots
	+ a_1 s^1
	+ a_0
	\tag{5}\label{equation 5}
\end{equation}

Per il toerema fondamentale dell'algebra, (\ref{equation 5}) ha $r$ radici  distinte $\lambda_1,\lambda_2, \dots, \lambda_r$ di molteplicità $\mu_1, \mu_2, \dots,\mu_r$ dove  $\mu_1 +\mu_2 + \dots + \mu_r = n$
%TODO controlla

%TODO




\section{Stabilità}
%TODO
\subsection{Stabilità asintotica}
Si studia la stabilità asintotica di sistemi descritti dall'equazione %TODO: aggiungere link all'equazione (1)

\begin{theorem}
	Il sistema % TODO: aggiungere link all'equazione (1) %
	è asintoticamente stabile se per ogni scelta delle consizioni iniziali:
	\begin{equation*}
		\lim_{t \to \infty}v_l(t)=0
		\Leftrightarrow\text{tutti i modi elementari } m(t)=e^{\lambda t}\frac{t^l}{t!}\text{ sono convergenti a $0$}
		\Leftrightarrow Re(\lambda_i)<0,\forall i = \overline{1,r}
	\end{equation*}
\end{theorem}
%TODO: Aggiungere esempio filtro RL
%TODO: Aggiungere esempio massa molla smorzatore

%TODO

\subsection{BIBO stabilità}
%TODO


\section{Risposta Libera}
%TODO

\section{Risposta Impulsiva}
\begin{definition}
	Dato un sistema descritto dall'equazione %TODO: aggiungere link all'equazione (1)
	, inizialmente a riposo, definiamo la risposta impulsiva e indichiamo con $h(t)$ l'uscita del sistema in corrispondenza dell'impulso unitario $\delta (t)$ in ingresso.
\end{definition}

\begin{definition}
	Date due funzioni $v_1(t)$ e $v_2(t) \quad t\in\mathbb{R} $ definiamo il prodotto (o integrale) di convoluzione come:
	\begin{equation*}
		(v_1 * v_2)(t) 
		= \int_{-\infty}^{\infty}v_1(\tau)v_2(\tau - t)\,d\tau 
		= \int_{-\infty}^{\infty}v_1(t - \tau)v_2(\tau )\,d\tau
	\end{equation*}
	Proprietà
	\begin{enumerate}
		\item $v_1 * v_2 = v_2 * v_1 $ (commutatività)
		\item $(v_1 * v_2) * v_3  = v_1 * (v_2 * v_3) $ (associatività)
		\item $v_1 * (v_2 + v_3)  = v_1 * v_2 + v_1 * v_3)$ (distributività)
	\end{enumerate}

\begin{oss}
	$(v *\delta)(t)\overset{def}{=} \int_{-\infty}^{\infty}v(\tau)\delta(t - \tau)\,d\tau = v(t)$ (proprietà di riproducibilità dell'impulso)
\end{oss}
\end{definition}
%TODO: Aggiugnere immagine con sistemi a blocchi? Mi pare una ripetizione
%TODO: non so come definire questa parte se è una dimostrazione o un teorema
Tenendo conto che $h(t)=0$ per $t<0$ (causalità) avremo:
\begin{equation*}
	\int_{-\infty}^{\infty}h(\tau)u(t - \tau)\,d\tau 
	= \int_{-\infty}^{t^+}h(t - \tau)u(\tau)\,d\tau
\end{equation*}
\begin{proof}
ponendo:  $t-\tau = s \Rightarrow d\tau = - ds 
\qquad \tau = 0^-  \Rightarrow s+t^+ 
\qquad \tau=\infty \Rightarrow s = -\infty$
\[
%TODO controllare gil estremi dell'integrale
\int_{t^+}^{\infty}h(t-s)u(s)\,-ds 
=-\int_{t^+}^{\infty}h(t-s)u(s)\,ds
=\int_{-\infty}^{t^+}h(t-s)u(s)\,ds
\qedhere
\]
\end{proof}

proprieta:
La risposta in uscita del sistema%TODO: aggiungere link all'equazione (1)
, inizialmente a riposo, di risposta impulsiva $h(t)\quad t\in\mathbb{R}$ in corrispondenza a un ingresso $u(t)\quad t\in\mathbb{R}$, se esiste, è data da:
\[
v(t)=(h*u)(t) = \int_{0^-}^{\infty}h(\tau)u(t - \tau)\,d\tau
= \int_{-\infty}^{t^+}h(t - \tau)u(\tau)\,d\tau
\]
in particolare la risposta forzata ad un ingresso $u(t)\quad t\in\mathbb{R} \quad u(t)=0 , t<0$ è:
\[
\int_{0^-}^{t^+}h(t - \tau)u(\tau)\,d\tau 
\quad\text{oppure}\quad
\int_{0^-}^{t^+}h(\tau)u(t - \tau)\,d\tau 
\]

%TODO anche qui non so cosa sia
sostituisco $\delta (t)$ in %TODO: aggiungere link all'equazione (1)
\begin{equation}
	a_n \frac{d^{(n)} h(t)}{dt^n}+\dots+a_0\,h(t) = b_m \frac{d^{(n+1)} \delta(t)}{dt^{n+1}}+\dots+b_0\,\delta(t) %TODO n+1 oppure m+1?
	\tag{7}\label{equation 7}
\end{equation}
Studiamo l'evoluzione forzata sull'orizzonte temporale $(0,+\infty)$ %TODO aggiungere disegno?
\[ %TODO controlla formula, soprattutto t = 0-
	h(0) = \frac{dh(t)}{dt}\bigg\vert_{t=0^-}
	=\frac{d^{(n-1)}h(t)}{dt^{n-1}}\bigg\vert_{t=0^-}
	= 0
	\delta(t) = 0 \qquad\text{per } t>0
\]
\begin{equation}
	\overset{\eqref{equation 7}}{\Rightarrow}a_n \frac{d^{(n)} h(t)}{dt^n}+\dots+a_0\,h(t) = 0 
	\tag{8}\label{equation 8}
\end{equation}
	
 %TODO controlla formula
\begin{oss}
	$\eqref{equation 8} \Rightarrow h (t)$ è una combinazione lineare di tutti i modi elementari ottenuti dalla soluzione dell'equazione caratteristica del sistema%TODO: aggiungere link all'equazione (1)
	in cui si aggiunge un termine impulsivo soltanto nei sistemi in cui $n=m$:
	\begin{equation}
		h(t)=0 \text{ per } t<0 \Rightarrow h(t) = h(t)\delta_{-1} (t)
		\Rightarrow h(t)= d_0 \delta(t) + \sum_{i=1}^{r}\sum_{l=0}^{\mu_i-1}d_{i,l} \,e^{\lambda_it}\frac{t^l}{t!}\delta_{-1} (t)
		\tag{9}\label{equation 9}
	\end{equation}
\end{oss}

%TODO

\section{Risposta Forzata}
%TODO
