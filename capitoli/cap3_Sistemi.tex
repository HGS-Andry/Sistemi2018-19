\chapter{Sistemi}
\section{introduzione}
%TODO



\section{Stabilità}
%TODO
\subsection{Stabilità asintotica}
Si studia la stabilità asintotica di sistemi descritti dall'equazione %TODO: aggiungere link all'equazione (1)

\begin{theorem}
	Il sistema % TODO: aggiungere link all'equazione (1) %
	è asintoticamente se per ogni scelta delle consizioni iniziali:
	\begin{equation*}
		\lim_{t \to \infty}v_l(t)=0
		\Leftrightarrow\text{tutti i modi elementari } m(t)=e^{\lambda t}\frac{t^l}{t!}\text{ sono convergenti a $0$}
		\Leftrightarrow Re(\lambda_i)<0,\forall i = \overline{1,r}
	\end{equation*}
\end{theorem}
%TODO: Aggiungere esempio filtro RL
%TODO: Aggiungere esempio massa molla smorzatore

%TODO

\subsection{BIBO stabilità}
%TODO

\section{Modelli}
%TODO

\section{Risposta Libera}
%TODO

\section{Risposta Impulsiva}
\begin{definition}
	Dato un sistema descritto dall'equazione %TODO: aggiungere link all'equazione (1)
	, inizialmente a riposo, definiamo la risposta impulsiva e indichiamo con $h(t)$ l'uscita del sistema in corrispondenza dell'impulso unitario $\delta (t)$ in ingresso.
\end{definition}

\begin{definition}
	Date due funzioni $v_1(t)$ e $v_2(t) \quad t\in\mathbb{R} $ definiamo il prodotto (o integrale) di convoluzione come:
	\begin{equation*}
		(v_1 * v_2)(t) 
		= \int_{-\infty}^{\infty}v_1(\tau)v_2(\tau - t)\,d\tau 
		= \int_{-\infty}^{\infty}v_1(t - \tau)v_2(\tau )\,d\tau
	\end{equation*}
	Proprietà
	\begin{enumerate}
		\item $v_1 * v_2 = v_2 * v_1 $ (commutatività)
		\item $(v_1 * v_2) * v_3  = v_1 * (v_2 * v_3) $ (associatività)
		\item $v_1 * (v_2 + v_3)  = v_1 * v_2 + v_1 * v_3)$ (distributività)
	\end{enumerate}

\begin{oss}
	$(v *\delta)(t)\overset{def}{=} \int_{-\infty}^{\infty}v(\tau)\delta(t - \tau)\,d\tau = v(t)$ (proprietà di riproducibilità dell'impulso)
\end{oss}
\end{definition}
%TODO: Aggiugnere immagine con sistemi a blocchi? Mi pare una ripetizione
%TODO: non so come definire questa parte se è una dimostrazione o un teorema
Tenendo conto che $h(t)=0$ per $t<0$ (causalità) avremo:
\begin{equation*}
	\int_{-\infty}^{\infty}h(\tau)u(t - \tau)\,d\tau 
	= \int_{-\infty}^{t^+}h(t - \tau)u(\tau)\,d\tau
\end{equation*}
\begin{proof}
ponendo:  $t-\tau = s \Rightarrow d\tau = - ds 
\qquad \tau = 0^-  \Rightarrow s+t^+ 
\qquad \tau=\infty \Rightarrow s = -\infty$
\[
%TODO controllare gil estremi dell'integrale
\int_{t^+}^{\infty}h(t-s)u(s)\,-ds 
=-\int_{t^+}^{\infty}h(t-s)u(s)\,ds
=\int_{-\infty}^{t^+}h(t-s)u(s)\,ds
\qedhere
\]
\end{proof}

proprieta:
La risposta in uscita del sistema%TODO: aggiungere link all'equazione (1)
, inizialmente a riposo, di risposta impulsiva $h(t)\quad t\in\mathbb{R}$ in corrispondenza a un ingresso $u(t)\quad t\in\mathbb{R}$, se esiste, è data da:
\[
v(t)=(h*u)(t) = \int_{0^-}^{\infty}h(\tau)u(t - \tau)\,d\tau
= \int_{-\infty}^{t^+}h(t - \tau)u(\tau)\,d\tau
\]
in particolare la risposta forzata ad un ingresso $u(t)\quad t\in\mathbb{R} \quad u(t)=0 , t<0$ è:
\[
\int_{0^-}^{t^+}h(t - \tau)u(\tau)\,d\tau 
\quad\text{oppure}\quad
\int_{0^-}^{t^+}h(\tau)u(t - \tau)\,d\tau 
\]

%TODO anche qui non so cosa sia
sostituisco $\delta (t)$ in %TODO: aggiungere link all'equazione (1)
\begin{equation}
	a_n \frac{d^{(n)} h(t)}{dt^n}+\dots+a_0\,h(t) = b_m \frac{d^{(n+1)} \delta(t)}{dt^{n+1}}+\dots+b_0\,\delta(t) %TODO n+1 oppure m+1?
	\tag{7}\label{equation 7}
\end{equation}
Studiamo l'evoluzione forzata sull'orizzonte temporale $(0,+\infty)$ %TODO aggiungere disegno?
\[
	h(0) = \frac{dh(t)}{dt}\bigg\vert_{t=0^-}
	=\frac{d^{(n-1)}h(t)}{dt^{n-1}}\bigg\vert_{h=0^-}
	= 0
	\delta(t) = 0 \qquad\text{per } t>0
\]
\begin{equation}
	\overset{\eqref{equation 7}}{\Rightarrow}a_n \frac{d^{(n)} h(t)}{dt^n}+\dots+a_0\,h(t) = 0 
	\tag{8}\label{equation 8}
\end{equation}
	
 %TODO controlla formula
\begin{oss}
	$\eqref{equation 8} \Rightarrow h (t)$ è una combinazione lineare di tutti i modi elementari ottenuti dalla soluzione dell'equazione caratteristica del sistema%TODO: aggiungere link all'equazione (1)
	in cui si aggiunge un termine impulsivo soltanto nei sistemi in cui $n=m$:
	\begin{equation}
		h(t)=0 \text{ per } t<0 \Rightarrow h(t) = h(t)\delta_{-1} (t)
		\Rightarrow h(t)= d_0 \delta(t) + \sum_{i=1}^{r}\sum_{l=0}^{\mu_i-1}d_{i,l} \,e^{\lambda_it}\frac{t^l}{t!}\delta_{-1} (t)
		\tag{9}\label{equation 9}
	\end{equation}
\end{oss}

%TODO

\section{Risposta Forzata}
%TODO
