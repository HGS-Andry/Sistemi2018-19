
%\begin{document}
	\section{Esercizi}
	
	\subsection{Esercizio 1}
	Calcolare la risposta libera
		\[
		\begin{split}
		&\ddot{v}(t) + 3\dot{v}(t)+2v(t) = \frac{1}{3}u(t)\\
		&\mathrm{condizioni \,\, iniziali:}\\
		&\dot{v}(0)=2\\
		&v(0)=1
		\end{split}		
		\]
		Equazione omogenea:
		\[s^2 + 3s + 2 = 0\]
		\[s_{1,2} = \frac{-3\pm\sqrt{9-4(1)(2)}}{2} =
		\begin{cases}
		-1 = \lambda_1\\
		-2 = \lambda_2
		\end{cases}
		\]
		Risposta libera, sostituire le radici trovate
		\[
		\begin{split}
		V_l(t)&=C_1e^{\lambda_1t} + C_2e^{\lambda_2t}\\
		&=C_1e^{-t}+C_2e^{-2t}
		\end{split}
		\]
		Ora per trovare $C_1$ e $C_2$, mettere a sistema la risposta libera con la propria derivata prima e porle uguali alle condizioni iniziali.\\
		Per $t=0 \quad e^t = e^0 = 1$
		\[
		%\begin{split}
		\begin{cases}
		C_1+C_2= 1\\
		-C_1-2C_2=2 
		\end{cases}
		\begin{cases}
		C_1 = 1-C_2\\
		-1+C_2-2C_2 = 2
		\end{cases}
		\begin{cases}
		C_1 = 1+3 = 4\\
		C_2 = -3
		\end{cases}
		\begin{cases}
		C_1=4\\
		C_2 =-3
		\end{cases}
		%\end{split}
		\]
		Sostituire $C_1$ e $C_2$ nella risposta libera
		\[V_l(t) = 4e^{-t}-3e^{-2t}\]
		Dalle due radici trovate notiamo che entrambe hanno parte reale negativa quindi abbiamo stabilità asintotica e BIBO stabilità.
		\newpage
	\subsection{Esercizio 2}
	Calcolare la risposta libera
		\[
		\begin{split}
		&\ddot{v}(t)-\dot{v}(t)-2v(t) = \ddot{u}(t)\\
		&\mathrm{condizioni \,\, iniziali:}\\
		&\dot{v}(0)=-1\\
		&v(0)=1
		\end{split}
		\]
		Equazione omogenea:
		\[s^2 - s - 2 = 0\]
		\[s_{1,2} = \frac{1\pm\sqrt{1-4(1)(-2)}}{2} =
		\begin{cases}
		2 = \lambda_1\\
		-1 = \lambda_2
		\end{cases}
		\]
		Risposta libera, sostituire le radici trovate
		\[
		\begin{split}
		V_l(t)&=C_1e^{\lambda_1t} + C_2e^{\lambda_2t}\\
		&=C_1e^{2t}+C_2e^{-t}
		\end{split}
		\]
		Ora per trovare $C_1$ e $C_2$, mettere a sistema la risposta libera con la propria derivata prima e porle uguali alle condizioni iniziali.\\
		Per $t=0 \quad e^t = e^0 = 1$
		\[
		%\begin{split}
		\begin{cases}
		C_1 + C_2 = 1\\
		2C_1 - C_2= -1 
		\end{cases}
		\begin{cases}
		C_1 = 1-C_2\\
		2-2C_2-C_2 = -1
		\end{cases}
		\begin{cases}
		C_1 = 1-1\\
		-3C_2 = -3
		\end{cases}
		\begin{cases}
		C_1=0\\
		C_2 =1
		\end{cases}
		%\end{split}
		\]
		Sostituire $C_1$ e $C_2$ nella risposta libera
		\[V_l(t) = 0e^{2t}+1e^{-t}\]
		Dalle due radici trovate notiamo che una è negativa e una positiva quindi il sistema NON è asintoticamente stabile, sulla BIBO stabilità per ora non possiamo dire nulla.
		\newpage
	\subsection{Esercizio 3}
	Calcolare risposta impulsiva
		\[
		\begin{split}
		&\dot{v}(t)+2v(t) = \dot{u}(t)+u(t)\\
		&\mathrm{condizioni \,\, iniziali:}\\
		&u(0)=\delta(t)
		\end{split}
		\]
		Equazione omogenea:
		\[s + 2 = 0\]
		\[s = -2\]
		Risposta impulsiva, sostituire le radici trovate\\
		Se n = m sommo $d_0 \delta(t)$
		\[
		\begin{split}
		h(t) = d_0 \delta(t) + d_1e^{-2t}\gradino(t)
		\end{split}
		\]
		Ora per trovare $d_0$ e $d_1$, fare la derivata della risposta impulsiva per ogni grado corrispondente all'equazione iniziale.\\
		Per $t=0 \quad e^t = e^0 = 1$
		\[
		\frac{dh(t)}{dt} = d_0\frac{d\delta(t)}{dt}-2d_1e^{-2t}\gradino(t)+d_1e^{-2t}\delta(t)
		\]
		Sostituire la derivata prima e la risposta impulsiva nell'equazione iniziale
		\[
		\begin{split}
		&\underbrace{d_0\frac{d\delta(t)}{dt}-2d_1e^{-2t}\gradino(t)+d_1e^{-2t}\delta(t)}_{\dot{v}(t)} + 
		2[\underbrace{d_0 \delta(t) + d_1e^{-2t}\gradino(t)}_{v(t)}] = \underbrace{\frac{d\delta(t)}{dt}}_{\dot{u}(t)} + \underbrace{\delta(t)}_{u(t)}\\
		&d_0\frac{d\delta(t)}{dt}\cancel{-2d_1e^{-2t}\gradino(t)}+d_1e^{-2t}\delta(t)+2d_0 \delta(t) + \cancel{2d_1e^{-2t}\gradino(t)} -\frac{d\delta(t)}{dt} - \delta(t) = 0\\
		&(d_0 - 1)\frac{d\delta(t)}{dt} + (d_1 + 2d_0 -1)\delta = 0
		\end{split}
		\]
		Per $t=0 \quad e^t = e^0 = 1$\\
		Metto a sistema le due soluzioni per trovare i due coefficienti della risposta impulsiva
		\[
		\begin{cases}
		d_0 -1 = 0\\
		d_1 + 2d_0 -1 = 0
		\end{cases}
		\begin{cases}
		d_0 = 1\\
		d_1 = -1
		\end{cases}
		\]
		La risposta impulsiva è:\\
		\[h(t) = \delta(t) - e^{-2t} \gradino(t)\]
		\newpage
	\subsection{Esercizio 4}
	Calcolare la risposta libera, la risposta impulsiva e la risposta forzata
		\[
		\begin{split}
		&\ddot{v}(t)+4v(t) = 2\dot{u}(t)\\
		&\mathrm{condizioni \,\, iniziali:}\\
		&\dot{v}(0)=0\\
		&v(0)=2\\
		&u(t)=\gradino(t)
		\end{split}
		\]
		Equazione omogenea:
		\[s^2 + 4 = 0\]
		\[s_{1,2} = \pm2j =
		\begin{cases}
		2j = \lambda_1\\
		-2j = \lambda_2
		\end{cases}
		\]
		
		Modi elementari:\\
		Rappresentazione esponenziale di un numero complesso:\\
		\[
		\begin{split}
		&1. e^{j2t}\\
		&2. e^{-j2t}\\
		\end{split}
		\]
		Rappresentazione polare:\\
		\[
		\begin{split}
		&1. e^{j\omega} = \cos(\omega)+j\sin(\omega)\\
		&2. e^{-j\omega} = -\cos(\omega)-j\sin(\omega)\\
		\end{split}
		\]
		Risposta libera, sostituire le radici trovate
		\[
		\begin{split}
		V_l(t)&=C_1\cos(2t)+C_2\sin(2t)
		\end{split}
		\]
		Ora per trovare $C_1$ e $C_2$, mettere a sistema la risposta libera con la propria derivata prima e porle uguali alle condizioni iniziali.\\
		Per $t=0 \quad \cos(2t) = \cos(0) = 1 \quad e \quad \sin(2t) = \sin(0) = 0$
		\[
		%\begin{split}
		\begin{cases}
		C_1\cos(0)+C_2\sin(0) = 2\\
		-2C_1\sin(0)+C_2\cos(0) = 0 
		\end{cases}
		\begin{cases}
		C_1 = 2\\
		C_2 = 0
		\end{cases}
		%\end{split}
		\]
		Sostituire $C_1$ e $C_2$ nella risposta libera
		\[V_l(t) = 2\cos(2t)+0\sin(2t)\]
		Dalle due radici trovate notiamo che la parte reale una è negativa e una positiva quindi il sistema NON è asintoticamente stabile, sulla BIBO stabilità per ora non possiamo dire nulla.\\
		Risposta impulsiva:\\
		\[
		h(t)= \underbrace{d_0\delta_0}_{\Leftrightarrow n=m}  + \gradino(t)[\sum_{i = 1}^{r}\sum_{l=0}^{\mu_0-1}d_{i,l}e^{\lambda_it}\frac{t^l}{t!}]
		\]
		\[ h(t)=[d_1\cos(2t) + d_2\sin(2t)] \gradino(t)\]
		Per trovare $d_1$ e $d_2$, i coefficienti, fare la derivata della risposta impulsiva per ogni grado corrispondente all'equazione iniziale.\\
		\[
		\begin{split}
		\dot{h}(t) =& [-2d_1\sin(2t)+2d_2\cos(2t)]\gradino(t)+[d_1\cos(2t)+d_2\sin(2t)]\delta(t)\\
		\ddot{h}(t) =& -4[d_1\cos(2t)+d_2\sin(2t)]\gradino(t)+[d_1\cos(2t)+d_2\sin(2t)]\frac{d\delta(t)}{dt}\\
		&+4[-d_1\sin(2t)+d_2\cos(2t)]\delta(t)\\
		\end{split}
		\]
		Sostituire nell'equazione iniziale tutte le derivate trovate e al posto dell'entrata sostituisco l'impulso
		\[
		\begin{split}
		&\cancel{-4[d_1\cos(2t)+d_2\sin(2t)]\gradino(t)}
		+[d_1\cos(2t)+d_2\sin(2t)]\frac{d\delta(t)}{dt}\\
		&+4[-d_1\sin(2t)+d_2\cos(2t)]\delta(t)\\&+\cancel{4[d_1\cos(2t) + d_2\sin(2t)] \gradino(t)}  = 2\frac{d\delta(t)}{dt}\\
		\\
		&4[-d_1\sin(2t)+d_2\cos(2t)]\delta(t) + [d_1\cos(2t)+d_2\sin(2t)-2]\frac{d\delta(t)}{dt} = 0
		\end{split}
		\]
		Mettere a sistema le due soluzioni per trovare $d_1$ e $d_2$ uguagliate a 0 e con t=0, quindi $\sin(0)=0$ e $\cos(0)=1$
		\[
		\begin{cases}
		-d_1\sin(0)+d_2cos(0) = 0\\
		d_1\cos(0)+d_2\sin(0)-2 = 0
 		\end{cases}
		\begin{cases}
		d_2 = 0\\
		d_1 = 2
		\end{cases}
		\]
		Sostituire nel h(t) $d_1$ e $d_2$\\
		Risposta Impulsiva:
		\[
		h(t)=2\cos(2t)\gradino(t)
		\]
		Per trovare la Risposta Forzata\\
		\[
		\begin{split}
		v_f(t)=&\int_{0}^{t}(h(\tau)u(t-\tau)d\tau)\\
		&\int_{0}^{t}(2\cos(2\tau)\underbrace{\gradino(\tau)\gradino(t-\tau)}_{per l'intervallo dato = 1}d\tau)\\
		&\int_{0}^{t}(2\cos(2\tau)d\tau) = \sin(2t)\\
		\end{split}
		\]
		Risposta Forzata:
		\[
		v_f(t) = \sin(2t)
		\]
	\newpage
	\subsection{Esercizio 5}
		Calcolare la risposta libera, risposta impulsiva e risposta forzata
		\[
		\begin{split}
		&\ddot{v}(t)+3\dot{v}(t)+2v(t) = \frac{1}{3}u(t)\\
		&\mathrm{condizioni \,\, iniziali:}\\
		&\dot{v}(0)=2\\
		&v(0)=1\\
		&u(t)=e^{-2t}\gradino(t)
		\end{split}
		\]
		Equazione omogenea:
		\[s^2 + 3s +2 = 0\]
		\[s_{1,2} = \frac{-3 \pm \sqrt{9-(4*2)}}{2}=
		\begin{cases}
		-1 = \lambda_1\\
		-2 = \lambda_2
		\end{cases}
		\]
		Risposta libera, sostituire le radici trovate
		\[
		\begin{split}
		V_l(t)&=C_1e^{-t}+C_2e^{-2t}
		\end{split}
		\]
		Ora per trovare $C_1$ e $C_2$, mettere a sistema la risposta libera con la propria derivata prima e porle uguali alle condizioni iniziali.\\
		Per $t=0 e^t = 0$
		\[
		%\begin{split}
		\begin{cases}
		C_1+C_2 = 1\\
		-C_1-2C_2 = 2 
		\end{cases}
		\begin{cases}
		C_1 = 1-C_2\\
		-1+C_2-2C_2 = 2
		\end{cases}
		\begin{cases}
		C_1 = 4\\
		C_2 = -3
		\end{cases}
		%\end{split}
		\]
		Sostituire $C_1$ e $C_2$ nella risposta libera
		\[V_l(t) = 4e^{-t}-3e^{-2t}\]
		Dalle due radici trovate notiamo che le parti reali sono entrambe negative quindi il sistema NON è asintoticamente stabile, sulla BIBO stabilità per ora non possiamo dire nulla.\\
		Risposta impulsiva:\\
		\[
		h(t)= \underbrace{d_0\delta_0}_{\Leftrightarrow n=m}  + \gradino(t)[\sum_{i = 1}^{r}\sum_{l=0}^{\mu_0-1}d_{i,l}e^{\lambda_it}\frac{t^l}{t!}]
		\]
		\[ h(t)=[d_1e^{-t} + d_2e^{-2t}] \gradino(t)\]
		Per trovare $d_1$ e $d_2$, i coefficienti, fare la derivata della risposta impulsiva per ogni grado corrispondente all'equazione iniziale.\\
		\[
		\begin{split}
		\dot{h}(t) =& [-d_1e^{-t}-2d_2e^{-2t}]\gradino(t)+[d_1e^{-t}+d_2e^{-2t}]\delta(t)\\
		\ddot{h}(t) =& [d_1e^{-t}+4d_2e^{-2t}]\gradino(t)-2[d_1e^{-t}+2d_2e^{-2t}]\delta(t)+[d_1e^{-t}+d_2e^{-2t}]\frac{d\delta(t)}{dt}\\
		\end{split}
		\]
		Sostituire nell'equazione iniziale tutte le derivate trovate e al posto dell'entrata sostituisco l'impulso
		\[
		\begin{split}
		&\cancel{[d_1e^{-t}+4d_2e^{-2t}]\gradino(t)}
		-2[d_1e^{-t}+2d_2e^{-2t}]\delta(t)+[d_1e^{-t}+d_2e^{-2t}]\frac{d\delta(t)}{dt}\\
		&\cancel{-3[d_1e^{-t}+2d_2e^{-2t}]\gradino(t)}+3[d_1e^{-t}+d_2e^{-2t}]\delta(t)+\cancel{2[d_1e^{-t} + d_2e^{-2t}]}  = \frac{1}{3}\delta(t)\\
		\\
		&[-2d_1e^{-t}-4d_2e^{-2t}+3d_1e^{-t}+3d_2e^{-2t}]\delta(t)+[d_1e^{-t}+d_2e^{-2t}]\frac{d\delta(t)}{dt} = \frac{1}{3}\delta(t)\\
		\\
		&[d_1e^{-t}-d_2e^{-2t}-\frac{1}{3}]\delta(t)+[d_1e^{-t}+d_2e^{-2t}]\frac{d\delta(t)}{dt}=0
		\end{split}
		\]
		Mettere a sistema le due soluzioni per trovare $d_1$ e $d_2$ uguagliate a 0 e con t=0, quindi $e^t = 1$
		\[
		\begin{cases}
		d_1-d_2-\frac{1}{3} = 0\\
		d_1+d_2 = 0
		\end{cases}
		\begin{cases}
		-d_2-d_2-\frac{1}{3} = 0\\
		d_1 = -d_2
		\end{cases}
		\begin{cases}
		d_2=-\frac{1}{3}*\frac{1}{2} = -\frac{1}{6}\\
		d_1 = \frac{1}{6}
		\end{cases}
		\]
		Sostituire nel h(t) $d_1$ e $d_2$\\
		Risposta Impulsiva:
		\[
		h(t)=[\frac{1}{6}e^{-t}-\frac{1}{6}e^{-2t}]\gradino(t)
		\]
		Per trovare la Risposta Forzata\\
		\[
		\begin{split}
		v_f(t)=&\int_{0}^{t}(h(\tau)u(t-\tau)d\tau)\\
		&\int_{0}^{t}((\frac{1}{6}e^{-\tau}-\frac{1}{6}e^{-2\tau})\underbrace{\gradino(\tau)\gradino(t-\tau)}_{per l'intervallo dato = 1}e^{-2(t-\tau)})d\tau\\
		&\int_{0}^{t}((\frac{1}{6}e^{-\tau}-\frac{1}{6}e^{-2\tau})e^{-2(t-\tau)}d\tau)\\
		&\frac{1}{6}e^{-2t}\int_{0}^{t}((e^{-\tau}-e^{-2\tau})e^{2\tau}d\tau)\\
		&\frac{1}{6}e^{-2t}\int_{0}^{t}(e^{\tau}-1)d\tau\\
		&\frac{1}{6}e^{-2t}\int_{0}^{t}(e^{-\tau})d\tau-\int_{0}^{t}(1)d\tau\\
		&\frac{1}{6}e^{-2t}[-e^{-\tau}]_0^t-[\tau]_0^t = \frac{1}{6}e^{-2t}(-e^{-t}-1-t)
		\\
		\end{split}
		\]
		\\
		Risposta Forzata:
		\[
		v_f(t) = \frac{1}{6}e^{-2t}(-e^{-t}-1-t)
		\]
		\newpage
	\subsection{Esercizio 6}
		Dire se il sistema è asintoticamente stabile, se è BIBO stabile, trovare la funzione di trasferimento, la risposta libera, forzata e totale
		\[
		\begin{split}
		&\ddot{v}(t)+3\dot{v}(t)+2v(t)=\dot{u}(t)+3u(t)\\
		&\mathrm{condizioni \,\, iniziali:}\\
		&v(0) = 1 \qquad u(0)=1\\
		&\dot{v}(0)=0 \qquad \dot{u}(0)=1\\
		&u(t)=e^{-4t}\delta(t)
		\end{split}
		\]
		Studiare la stabilità
		\[
		\begin{split}
		s^2+3s+2=0
		s_1,2=\frac{-3\pm\sqrt{9-4*2}}{2} = 
		\begin{cases}
		-1 = \lambda_1\\
		-2 = \lambda_2
		\end{cases} 
		\end{split}
		\]
		Re($\lambda_1$) e Re($\lambda_2$) < 0 , quindi il sistema è Asintoticamente stabile e implica la BIBO stabilità
		Faccio le Trasformate di Laplace e sostituisco le condizioni iniziali
		\[
		\begin{split}
		&\TdL[\ddot{v}(t)]=s^2V(s)-(v(0)s+\dot{v}(0)) = s^2V(s)-s\\
		&\TdL[3\dot{v}(t)]=3(sV(s)-v(0))=3sV(s)-3\\
		&\TdL[2v(t)]=2V(s)\\
		&\TdL[\dot{u}(t)]=sU(s)-u(0)=sU(s)-1\\
		&\TdL[3u(t)]=3U(s)
		\end{split}
		\]
		Riscrivo tutte le Trasformate di Laplace sommandole tra di loro
		\[
		\begin{split}
		&s^2V(s)-s+3sV(s)-3+2V(s)=sU(s)-1+3U(s)\\
		&V(s)(s^2+3s+2) = U(s)(s+3)+s+2\\
		&V(s)=\underbrace{\frac{(s+3)}{(s^2+3s+2)}U(s)}_{Risposta\,Forzata:\, V_f(s)=H(s)U(s)}+\underbrace{\frac{(s+2)}{(s^2+3s+2)}}_{Risposta\, Libera}
		\end{split}
		\]
		Se il sistema non fosse Asintoticamente stabile non avrei potuto dire nulla sulla BIBO stabilità, ora potrei studiarla grazie al denominatore della funzione di trasferimento trovando le radici chiamati poli.
		Per trovare la Risposta Forzata devo fare la Trasformata di Laplace dell'ingresso e moltiplicarlo per H(s)
		\[
		\begin{split}
		&H(s)=\frac{(s+3)}{(s^2+3s+2)}= \frac{(s+3)}{(s+2)(s+1)}\\
		&U(s)=\TdL[e^{-4t}\delta(t)]=\frac{1}{s+4}\\
		&V_f(s)=\frac{(s+3)}{(s+2)(s+1)}\frac{1}{s+4}=\frac{(s+3)}{(s+2)(s+1)(s+4)}
		\end{split}
		\]
		Per trovare la Risposta totale sommo la Risposta Forzata con la Risposta Libera
		\[
		\begin{split}
		V(s)=V_l(s)+V_f(s)
		&=\frac{(s+3)}{(s+2)(s+1)(s+4)} + \frac{(s+2)}{(s+2)(s+1)}\\
		&=\frac{s+3+(s+2)(s+4)}{(s+1)(s+2)(s+4)}
		=\frac{s^2+7s+11}{(s+1)(s+2)(s+4)}\\
		&=\frac{A}{s+1}+\frac{B}{s+2}+\frac{C}{s+4}
		\end{split}
		\]
		\[
		\begin{split}
		&A=\cancel{(s+1)}\frac{s^2+7s+11}{\cancel{(s+1)}(s+2)(s+4)}\Bigg|_{s=-1}=\frac{5}{3}\\
		&B=\cancel{(s+2)}\frac{s^2+7s+11}{(s+1)\cancel{(s+2)}(s+4)}\Bigg|_{s=-2}=-\frac{1}{2}\\
		&C=\cancel{(s+4)}\frac{s^2+7s+11}{(s+1)(s+2)\cancel{(s+4)}}\Bigg|_{s=-4}=\frac{1}{2}
		\end{split}
		\]
		Sostituendo troviamo la Risposta totale nel dominio delle frequenze, facendo l'anti-trasformata di Laplace torniamo nel dominio del tempo
		\[
		\begin{split}
		&V(s)=\frac{5}{3}\Bigg(\frac{1}{s+1}\Bigg)-\frac{1}{2}\Bigg(\frac{1}{s+2}\Bigg)+\frac{1}{2}\Bigg(\frac{1}{s+4}\Bigg)\\
		&\TdL^{-1}[V(s)]=\frac{5}{3}e^{-t}-\frac{1}{2}e^{-2t}+\frac{1}{2}e^{-4t}
		\end{split}
		\]
		Quindi la Risposta totale nel dominio del tempo è:
		\[
		v(t)=\Bigg(\frac{5}{3}e^{-t}-\frac{1}{2}e^{-2t}+\frac{1}{2}e^{-4t}\Bigg)\gradino(t)
		\]
		\newpage
%\end{document}