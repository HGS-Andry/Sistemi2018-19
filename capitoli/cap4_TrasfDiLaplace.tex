\chapter{Trasformata Di Laplace}
%\section{Ivan_Love_You}
%\documentclass[a4paper]{report}
%\usepackage[T1]{fontenc}
%\usepackage[utf8]{inputenc}
%\usepackage[italian]{babel}
%\usepackage{mathrsfs}
%\usepackage{amsthm}
%\usepackage{amsmath}
%\usepackage{amsfonts}
%\usepackage{cancel}
%ivan è bravo <3 LODI



%\begin{document}
%\chapter{La trasformata di Laplace [TdL]} % TODO: chapter

\begin{definizione}
   Se $v:\mathbb{R}\to\mathbb{C}$ è localmente sommabile in $[0, +\infty)$ oppure \\$\Bigl(\displaystyle\int_a^bv(t)\,dt < +\infty\quad\forall a,b \in[0, +\infty)\Bigr)$ si definisce la TdL unilatera del segnale $v(t)$\\
   $\displaystyle\LA[v(t)](s) := V(S) = \intZeroInfinity v(t)e^{-st}\,dt$ , dove $s = \sigma + jw$ è una variabile complessa
   $V$ è definita per quei valori di $s$ per cui l'integrale è ben definito\\ $\displaystyle\Leftrightarrow \intZeroInfinity v(t)e^{-st}\,dt < +\infty$\\
   Tale regione del piano complesso è chiamata \emph{regione di convergenza (RdC)}\\
   Si può dimostrare che la RdC è un semipiano aperto del tipo \\$\mathrm{RdC} = \{s\in\mathbb{C}/\Re(s)>\alpha\}$ dove $\alpha\in\mathbb{R}$ ascissa di convoluzione della TdL
   %TODO: Figura
   \begin{proof}[Dim]
      Dimostriamo per combinazioni lineari di funzioni esponenziali
      \[
         \begin{split}
            & v(t) = \sum_{i = 1}^nc_ie^{\lambda it} \qquad \lambda_i = \sigma_i + jw_i\in\mathbb{C}, i = \overline{1,n}\\
            & V(s) = \intZeroInfinity \sum_{i = 1}^nc_ie^{\lambda it}e^{-st}\,dt = \sum_{i = 1}^nc_i\intZeroInfinity e^{\lambda it}e^{-st}\,dt\\
            & \mathrm{Dimostriamo\, che\, } \Bigg\lvert\intZeroInfinity e^{\lambda_it}e^{-st}\,dt\Bigg\rvert < +\infty\quad\forall i \\
            & \intZeroInfinity e^{\sigma_it}\cdot \underbrace{e^{jw_it}}_{\lvert\,\,\rvert = 1}\cdot e^{-\sigma t}\cdot \underbrace{e^{-jwt}}_{\lvert\,\,\rvert = 1}\,dt = \intZeroInfinity e^{(\sigma_i - \sigma)t} = \frac{e^{(\sigma_i - \sigma)t}}{\sigma_i - \sigma}\Big|_{0^-}^{+\infty} =\\
            & = \underbrace{\lim_{t\to+\infty}\frac{e^{(\sigma_i - \sigma)t}}{\sigma_i - \sigma}}_{\mathrm{converge\, per\, \sigma_i - \sigma < 0 \,\Leftrightarrow\, \sigma > \sigma_i}} - \frac{1}{\sigma_i - \sigma}
         \end{split}
      \]
      Scegliendo $\alpha = sup\{\Re(\lambda_i) : i = \overline{1,n}\} \Leftarrow$ TdL converge $\displaystyle\forall v(t) = \sum_{i = 1}^nc_ie^{\lambda_it}$ \AdC
      \begin{osservazione}
         \begin{enumerate}
            \item I sistemi stabili hanno $\displaystyle \alpha < 0(\Leftrightarrow \Re(\lambda_i) < \alpha < 0\quad \forall i = \overline{1,n}) \Rightarrow jw\in\mathrm{\,RdC\,}\forall w\in\mathbb{R}$
            \item $v(t) (\mathrm{time}) \leftrightarrow V(s)(\mathrm{complex})$
         \end{enumerate}
      \end{osservazione}
   \end{proof}
\end{definizione}

\section{Proprietà della trasformata di Laplace}
\subsection{Linearità}
\[
   \begin{split}
      \LA[a_1v_1(t) + a_2v_2(t)] = a_1\LA[v_1(t)] + a_2\LA[v_2(t)]\\
      \mathrm{RDC} = {s\in\mathbb{C} / \Re(s)>\alpha}\mathrm{, dove}\,\alpha\ge\{\alpha_1, \alpha_2\}
   \end{split}
\]
%------------------------------------------------------------------------------------------------------------------------------------------------------------------------------
\subsection{Time shifting (Ritardo temporale)}
Se $v(t)$ ammette TdL allora, $v(t - \tau)$ ammette TdL e $\LA[v(t) - \tau] = e^{-s\tau}V(s)$\, $\tau>0$\\
L'\AdC di $v(t - \tau)$ è la stessa di $v(t)$
\begin{proof}[Dim]
   \[
      \begin{split}
         \LA[v(t) - \tau] & = \intZeroInfinity v(t - \tau)e^{-st}\,dt = \int_{0^-}^\tau v(t - \tau)e^{-st}\,dt\,+\,\int_\tau^{+\infty}v(t - \tau)e^{-st}\,dt =\\
         & = \int_\tau^{+\infty}v(t - \tau)e^{-st}\,dt = \intZeroInfinity v(e^{-s(x + \tau)}\,dx = \intZeroInfinity v(t)e^{-s\tau}e^{-st}\,dt =\\
         & x = t - \tau \Rightarrow dt = dx \\
         & poi \\ % TODO: inserire tabella
         & x = t \Rightarrow dx = dt\\
         & = e^{-s\tau}\overbrace{\int_\tau^{+\infty}v(t)e^{-st}\,dt}^{V(s)} = e^{-s\tau}V(s)
      \end{split}
   \]
\end{proof}
%------------------------------------------------------------------------------------------------------------------------------------------------------------------------------
\subsection{Moltiplicazione per una funzione esponenziale (Frequency shifting)}
   $\LA[e^{\lambda t}v(t)] = V(s - \tau)$\\
   $\alpha_2 = \alpha + \Re(\lambda)$, dove $\lambda$ è AdC di $v(t)$\\
   $\alpha_2 = $ \AdC di $e^{\lambda t}v(t)$
\begin{proof}[Dim]
   \[
      \intZeroInfinity e^{\lambda t}v(t)e^{-st}\,dt = \intZeroInfinity v(t)e^{-(s - \lambda)t}\,dt = V(s - \lambda)
   \]
\end{proof}
%------------------------------------------------------------------------------------------------------------------------------------------------------------------------------
\subsection{Cambiamento di scala}

$\LA[v(rt)] = \frac{1}{r}V\Bigl(\frac{s}{r}\Bigr)$\\
$\alpha_2 = r\alpha$ ($\alpha$ \AdC di $v(t)$)

\begin{proof}[Dim]
   \[
      \begin{split}
         &\intZeroInfinity v(rt)e^{-st}\,dt = \intZeroInfinity \frac{1}{r}v(x)e^{-\frac{s}{t}}\,dx = \frac{1}{r}V\Bigl(\frac{s}{r}\Bigr)\\
         & rt = x\\
         & t = \frac{x}{r}\,dx = \frac{1}{r}\,dt %TODO: tabella
      \end{split}
   \]
\end{proof}
%------------------------------------------------------------------------------------------------------------------------------------------------------------------------------
\subsection{Proprità della derivata}
Se $v(t)$ ammette TdL ed esiste ed è finito $v(0^-) = \lim_{t\to 0} \Rightarrow \frac{dv(t)}{dt}$ ammette TdL e
\[
   \begin{split}
      & \LA\Bigl[\frac{dv(t)}{dt}\Bigr] = s\LA[v(t)] - v(0^-)\\
      & \alpha_2 = \LA\Bigl[\frac{dv(t)}{dt}\Bigr]\\
      & \alpha = \LA[v(t)]
   \end{split}
\]
L'\AdC $(\alpha_2 \le \alpha)$

\begin{proof}[Dim]
   \[
      \begin{split}
         \LA\Bigl[\frac{dv(t)}{dt}\Bigr] & = \intZeroInfinity \frac{dv(t)}{dt}e^{-st}\,dt = v(t) - e^{-st}\Big|_{0^-}^{+\infty} - (-s)\underbrace{\intZeroInfinity v(t)e^{-st}\,dt}_{V(s)} =\\
         & = \underbrace{\lim_{t\to +\infty}[v(t)e^{-st}|_{0^-}^{t}]}_{-v(0^-)} + sV(s) = -v(0^-) + sV(s)\\
      \end{split}
   \]
\end{proof}
%------------------------------------------------------------------------------------------------------------------------------------------------------------------------------
\subsection*{5.1$\quad$ Proprità della derivata seconda ed n-esima}
\[
   \LA\Bigl[\frac{d^2v(t)}{dt^2}\Bigr] = s^2\LA[v(t)] - sv(0^-) - \frac{dv(0^-)}{dt}
\]
\begin{proof}[Dim]
   \[
      \begin{split}
         \LA\Bigl[\frac{d}{dt}\Bigl[\frac{dv(t)}{dt}\Bigr]\Bigr]\overset{(5)}{=} & s\overbrace{\LA\Bigl[\frac{dv(t)}{dt}\Bigr]}^{sV(s) - v(0^-)} - \frac{dv(0^-)}{dt} = s(s\LA[V(t)] - v(0^-)) - \frac{dv(0^-)}{dt} = \\
         = & s^2V(s) - sv(0^-) - \frac{dv(0^-)}{dt}
      \end{split}
   \]
   In generale,
   \[
      \LA\Bigl[\frac{d^iv(t)}{dt^i}\Bigr] = s^i\LA[v(t)] - \sum_{k = 0}^{i - 1}\frac{d^kv(t)}{dt^k}\Big|_{t = 0^-}s^{i - 1 - k}
   \]
\end{proof}
%------------------------------------------------------------------------------------------------------------------------------------------------------------------------------
\subsection{Moltiplicazione per una funzione polinomiale}
\begin{proof}[Dim]
\[
   \LA[tv(t)] = - \frac{dV(s)}{ds}
\]
Con la stessa \RdC
   \[
      \begin{split}
         \frac{dV(s)}{ds} & = \frac{d}{ds}\Bigl[\intZeroInfinity v(t)e^{-st}dt\Bigr] = \intZeroInfinity \frac{\delta e^{-st}}{\delta s}v(t)dt = \intZeroInfinity -te^{-st}v(t) dt =\\
         & = \intZeroInfinity[-tv(t)]e^{-st}dt = - \LA[tv(t)]
      \end{split}
   \]
   In generale,
   \[
      \LA[t^iv(t)] = (-1)^i\frac{d^iV(s)}{ds^i}
   \]
\end{proof}
%------------------------------------------------------------------------------------------------------------------------------------------------------------------------------
\subsection{Integrale nel dominio del tempo}
Se $v(t)$ ha la TdL $V(s)$ per $\Re(s) > \alpha \Longrightarrow \int_{0^-}^tv(\tau)d\tau$ ha TdL $\alpha_1 = max\{0, \alpha\}$ e $\LA[\int_{0^-}^tv(\tau)d\tau] = \frac{V(s)}{s}$
\begin{proof}[Dim]
   \[
      \begin{split}
         & v_1(t) = \int_{0^-}^tv(\tau)d\tau \Rightarrow v_1'(t) = v(t)\\
         & v_1(0^-) = 0\\
         & \LA[v(t)] = \LA\Bigl[\frac{dv_1(t)}{dt}\Bigr] = s\LA[v_1(t)] - \overbrace{v_1(0^-)}^{0} = \LA\Bigl[\int_{0^-}^tv(\tau)d\tau\Bigr] = \frac{V(s)}{s}
      \end{split}
   \]
\end{proof}
%------------------------------------------------------------------------------------------------------------------------------------------------------------------------------
\subsection{Integrale nel dominio complesso $\mathbb{C}$}
Se esiste $\lim_{t\to 0^-}\frac{v(t)}{t}$ allora,
\[
   \LA\Bigl[\frac{v(t)}{t}\Bigr] = \int_s^{+\infty}V(s)\,ds
\]
\begin{proof}[Dim]
   \[
      \begin{split}
         & V(s) = \intZeroInfinity v(t)e^{-st}\,dt= \int_s^{+\infty}V(s)\,ds = \int_s^{+\infty}\Bigl[\intZeroInfinity v(t)e^{-st}\,dt\Bigr]\,ds = \\
         & = \int_s^{+\infty}v(t)\Bigl[\intZeroInfinity e^{-st}\,dt\Bigr]\,ds = \intZeroInfinity v(t)\Bigl[-\frac{1}{t}e^{-st}\Big|_s^{+\infty}\Bigr]\,dt = \\ %TODO: controlla veridicità
         & = \intZeroInfinity v(t)\frac{e^{-st}}{t}\,dt = \LA\Bigl[\frac{v(t)}{t}\Bigr]
      \end{split}
   \]
\end{proof}
%------------------------------------------------------------------------------------------------------------------------------------------------------------------------------
\subsection{Teorema del valore iniziale}
Se $v(t)$ ha Tdl, se $\exists \lim_{s\to \infty}v(t)$, finito $\Rightarrow \lim_{t\to 0^-}v(t) = \lim_{s\to \infty}sV(s)$
\begin{proof}[Dim]
   \[
      \begin{split}
         & \LA\Bigl[\frac{dv(t)}{dt}\Bigr] = sV(s) - v(0^-)\qquad [5]\\
         & \lim_{s\to \infty}[sV(s) - v(0^-)] = \lim_{s\to \infty}\intZeroInfinity\frac{dv(t)}{dt}e^{-st}\,dt = \lim_{s\to \infty} \Biggl(\lim_{\substack{T\to \infty \\ \varepsilon\to 0^-}}\int_\varepsilon^T\frac{dv(t)}{dt}e^{-st}\,dt\Biggr) \\
         & = \lim_{\substack{T\to \infty \\ \varepsilon\to 0^-}}\Biggl[\int_\varepsilon^T\frac{dv(t)}{dt}\Bigl(\underbrace{\lim_{s\to \infty}e^{-st}\,dt}_0\Bigr)\Biggr] = 0 \Rightarrow \lim_{s\to \infty}[sV(s) - v(0^-)] = 0 \\
         & \lim_{s\to \infty}sV(s) = v(0^-)
      \end{split}
   \]
\end{proof}
%------------------------------------------------------------------------------------------------------------------------------------------------------------------------------
\subsection{Teorema del valore finale}
Se $v(t)$ ha Tdl, e $\lim_{s\to \infty}v(t)$ esiste ed è finito $\Rightarrow \lim_{t\to \infty}v(t) = \lim_{s\to 0^+}sV(s)$
\begin{proof}[Dim]
   \[
      \begin{split}
         & \LA\Bigl[\frac{dv(t)}{dt}\Bigr] = sV(s) - v(0^-)\qquad [5]\\
         & \lim_{s\to 0^+}[sV(s) - v(0^-)] = \lim_{s\to 0^+}\intZeroInfinity\frac{dv(t)}{dt}e^{-st}\,dt = \lim_{\substack{T\to \infty \\ \varepsilon\to 0^-}}\Biggl[\int_\varepsilon^T\frac{dv(t)}{dt}\Bigl(\underbrace{\lim_{s\to 0^+}e^{-st}\,dt}_1\Bigr)\Biggr]\\
         & = \lim_{\substack{T\to \infty \\ \varepsilon\to 0^-}} \Bigl[[v(t)|_\varepsilon^T\Bigr] = \lim_{\substack{T\to \infty \\ \varepsilon\to 0^-}}[v(t) - v(\varepsilon)] = \lim_{T\to\infty}v(T) - v(0^-)\\
         & \lim_{s\to 0^+}sV(s) - \cancel{v(0^-)} = \lim_{T\to\infty}v(T) - \cancel{v(0^-)}
      \end{split}
   \]
\end{proof}
%------------------------------------------------------------------------------------------------------------------------------------------------------------------------------
\subsection{Convoluzione nel dominio del tempo (Prodotto nelle frequenze)}
Se $v_1(t)$, $v_2(t)$ sono nulle per $t < 0$ e hanno TdL $V_1(s)$, $V_2(s)$ $\Rightarrow [v_1 * v_2](t)$ ha TdL:\\
\[
   \underbrace{\LA[(v_1 * v_2)(t)]}_{Convoluzione\, nel\, tempo} = \underbrace{V_1(s)\cdot V_2(s)}_{Moltiplicazione\, nelle\, frequenze}
\]
\begin{proof}[Dim]
   \[
      \begin{split}
         &\LA[(v_1 * v_2)(t)] =\\
         & = \LA\Biggl[\intZeroInfinity \underbrace{v_1(\tau)}_{=\, 0\, per\, t < 0}v_2(t - \tau)\,d\tau\Biggr] = \intZeroInfinity\Biggl[\intZeroInfinity v_1(\tau)v_2(t - \tau)\,d\tau\Biggr]e^{-st}\,dt\\
         & = \intZeroInfinity v_2(\tau)\Biggl[\intZeroInfinity v_2(t - \tau)e^{-st}\,dt\Biggr]\,d\tau \rightarrow \intZeroInfinity v_1(\tau)\Biggl[\intZeroInfinity v_2(\lambda)e^{-s(\lambda + \tau)}\,d\lambda\Biggr]\,d\tau \\ 
         & t - \tau = \lambda \\
         & t = \lambda + \tau \\
         & dt = d\lambda \\
         & = \intZeroInfinity v_1(\tau)e^{-st}\Biggl[\underbrace{\intZeroInfinity v_2(\lambda)e^{-s\lambda}\,d\lambda}_{\LA[v_2(t)]}\Biggr] = \LA[v_2(t)]\intZeroInfinity v(\tau)e^{-st}\,d\tau = V_2(s)\cdot V_1(s)
      \end{split}
   \]
\end{proof}

%===================================================================================================================================================================================

\section{Esempi di trasformazioni notevoli}
   \begin{enumerate}
      \item[a.] Impulso ideale unitario $\delta(t)$\\
         $\displaystyle\LA[\delta(t)] = \intZeroInfinity\delta(t)e^{-st}\,dt \overset{\mathrm{Campionamento\,per\,} v(t) = e^{-st}}{=} e^{s\cdot 0} = 1$\\
         $V(s) = 1$ RdC = $\mathbb{C}$
      \item[b.] Gradino unitario $\gradino$\\
         $\displaystyle\LA[\gradino(t)] = \intZeroInfinity\underbrace{\gradino(t)}_{1}e^{-st}\,dt = \intZeroInfinity e^{-st} = $\\
         $\displaystyle = -\frac{e^{-st}}{s}\Big|_{0^-}^{+\infty} = 0 - \Bigl(-\frac{1}{s}\Bigr) = \frac{1}{s}$
      \item[c.] Impulso unitario\\
         $\LA[\delta(t - t_0)] = e^{-st_0}\LA[\overbrace{\delta(t)}^{1}] = e^{-st_0}$
      \item[d.] Esponenziale causale $v(t) = e^{\lambda t}\gradino(t), \lambda\in\mathbb{C}$\\
         $\displaystyle\LA[e^{\lambda t}\gradino(t)] \overset{(3)}{=} V(s - \lambda) = \frac{1}{s -\lambda}$\\
         RdC = $\{s\in\mathbb{C} : \Re(s) > \Re(\lambda)\}$
      \item[e.] Esponenziale causale moltiplicata per una funzione polinomiale\\
         \begin{center}
            $\displaystyle v(t) = \frac{t^\ell}{\ell!}e^{\lambda t}\gradino(t)$
         \end{center}
         \[
            \begin{split}
               &\LA\Bigl[\frac{t^\ell}{\ell!}e^{\lambda t}\gradino(t)\Bigr] \overset{(1)}{=} \frac{1}{\ell}\LA[t^\ell \overbrace{e^{\lambda t}\gradino(t)}^{v(t)}] \overset{(6)}{=} \frac{(-1)^\ell}{\ell!}\cdot\frac{d^\ell}{ds^\ell}\LA[e^{\lambda t}\gradino(t)] =\\
               &\overset{(d)}{=} \frac{(-1)^\ell}{\ell!}\cdot\frac{d^\ell}{ds^\ell}\Bigl[\frac{1}{s - \lambda}\Bigr] = \frac{(-1)^\ell}{\cancel{\ell!}}\cdot\cancel{\ell!}\frac{1}{(s - \lambda)^{\ell + 1}} = \frac{1}{(s - \lambda)^{\ell + 1}}
            \end{split}
         \]
         Esempio
         \[
            \begin{split}
               \LA[te^{\lambda t}\gradino(t)] = \frac{1}{(s - \lambda)^{2}}\\
               \LA\Bigl[\frac{t^2}{2!}e^{\lambda t}\gradino(t)\Bigr] = \frac{1}{(s - \lambda)^{3}}
            \end{split}
         \]
         Per $\lambda = 0$
         \[
            \begin{split}
               \LA\Bigl[\frac{t^\ell}{\ell!}\gradino(t)\Bigr] = \frac{1}{s^{\ell + 1}}\\
               \LA[t^\ell\gradino(t)] = \frac{\ell!}{s^{\ell + 1}}
            \end{split}
         \]
   \end{enumerate}
   \section{Sistemi LTI causali analisi dominio complesso o nel dominio delle frequenze}
   \begin{center}
      $\displaystyle a_n\der{v(t)}{n} + \dots + a_0v(t) = b_m\der{u(t)}{m} + \dots + b_0u(t)\qquad(1)$
   \end{center}
      $a_n, b_m \neq 0 \quad n\ge m \quad u(t)$ ingresso nullo per $t < 0\quad (u(t) = u(t)\gradino)$\\
      Condizioni iniziali $\displaystyle= v(0^-), \der{v(0^-)}, \dots, \derN{v(0^-)}{n-1}$\\
      Se $u(t)$ ha TdL $v(t)$ ammette TdL $\bigr(v(t)$ ristretto per $t\ge 0\bigl)$\\
      $U(s) = \LA[u(t)]\qquad e\qquad V(s) = \LA[v(t)]$\\
      (Proprietà 5) $\displaystyle\qquad \LA\Bigr[\derN{v(t)}{i}\Bigl] = s^iV(s) - \sum_{k = 0}^{i - 1}\derN{v(t)}{k}\Big|_{t = 0^-}s^{i - 1 - k}, i = \overline{1,n}$\\
      Applichiamo la TdL a (1), ma solo all'uscita
      \[
         \begin{split}
            &\displaystyle a_n\Biggr[s^nV(s) - \sum_{k = 0}^{n - 1}\derN{v(t)}{k}\Big|_{t = 0^-}s^{n - 1 - k}\Biggl] + a_{n - 1}\Biggr[s^nV(s) - \sum_{k = 0}^{n - 2}\derN{v(t)}{k}\Big|_{t = 0^-}s^{n - 2 - k}\Biggl] + \dots + a_0V(s) =\\
            & = b_ms^mU(s) + b_{m - 1}s^{m - 1}U(s) + \dots + b_0U(s)\\\\
            & (\overbrace{a_ns^n + a_{n - 1}s^{n - 1} + \dots + a_0}^{d(s) \mathrm{\,\,polinomio\,\,di\,\,grado\,\,}n})V(s) -\\
            & \overbrace{a_nv(0^-)s^{n-1} - \Bigl[a_{n-1}v(0^-) + a_n\der{v(t)}\Big|_{t = 0^-}\Bigr]s^{n - 2} - \dots + \Biggl[\sum_{k = 0}^{n - 1}a_{k + 1}\derN{v(t)}{k}\Big|_{t = 0^-}\Biggr]}^{p(s)\mathrm{\,\,polinomio\,\,di\,\,grado\,\,}n-1} = \\
            & = (\overbrace{b_ms^m + b_{m - 1}s^{m - 1} + \dots + b_0}^{n(s) \mathrm{\,\,polinomio\,\,di\,\,grado\,\,}m})U(s)\\
            & \Rightarrow d(s)V(s) - p(s) = n(s)U(s)\qquad\Big|_{\mathrm{Divido\,\,per\,\,}d(s)}\\
            & \Rightarrow V(s) = \frac{p(s)}{d(s)}+ \frac{n(s)}{d(s)}U(s)
         \end{split}
      \]
      \begin{osservazione}
         \begin{enumerate}
            \item $\displaystyle\frac{p(s)}{d(s)}$ dipende soltanto dalle condizioni iniziali su $V$ e dal polinomio caratteristico $\Rightarrow$ rappresento la TdL della risposta libera
               \[V_\ell(s) = \frac{p(s)}{d(s)}\]
            \item $\displaystyle\frac{n(s)}{d(s)}U(s)$ dipende soltanto dal sistema e dall'ingresso $\Rightarrow$ TdL della risposta forzata
               \[V_f(s) = \frac{n(s)}{d(s)}U(s)\]
               \[V(s) = \frac{p(s)}{d(s)} + \frac{n(s)}{d(s)}U(s)\]
            \item Sappiamo che $v_f(t) = [h * u](t) \xRightarrow{(11)} V_f(s) = H(s)U(s) \Rightarrow H(s) = \frac{n(s)}{d(s)}$ è la TdL di $h(t)$ (risposta impulsiva)\\
            (Funzione di trasferimento del sistema)
            \[H(s) = \frac{b_ms^m + b_{m - 1}s^{m - 1} + \dots + b_0}{a_ns^n + a_{n - 1}s^{n - 1} + \dots + a_0}\]
            (Funzione razionale in $s \in \mathbb{C}$)
         \end{enumerate}
      \end{osservazione}
      \emph{Esempio} $\qquad\displaystyle\derN{v(t)}{3} + \derN{v(t)}{2} = \der{u(t)}$
      \[
         \begin{split}
            &\mathrm{TdL} \Rightarrow s^3V(s) - s^2v(0^-) - s\dot{v}(0^-) - \ddot{v}(0^-) + s^2V(s) - sv(0^-) - \dot{v}(0^-) = U(s) \rightarrow \mathrm{\,\,Niente\,\,condizioni\,\,iniziali}\\
            &d(s) = s^3 + s^2\\
            &p(s) = s^2v(0^-) + [\dot{v}(0^-) + v(0^-)]s + \ddot{v}(0^-) + \dot{v}(0^-)\\
            &n(s) = s\\
            &H(s) = \frac{s}{s^3 + s^2} = \frac{1}{s^2 + s}\\
            &H(s) = \LA[h(t)]\\
            &h(t) = d_0\delta(t) + \sum_{i = 0}^{r}\sum_{\ell = 0}^{\mu - 1}\cdot d_{i,\ell}\cdot e^{\lambda_it}\cdot\frac{t^\ell}{\ell}\cdot\gradino(t)\xRightarrow{TdL} H(s) = d_0\delta(t) + \sum_{i = 0}^{r}\sum_{\ell = 0}^{\mu - 1}d_{i, \ell}\cdot\frac{1}{(s - \lambda_i)^{\ell + 1}}
            \end{split}
      \]
      $d_0$ compare solo se $n = m$\\
      $\lambda_i$ sono le radici di $\displaystyle\sum_{i = 0}^{n}a_is^i = 0\qquad$ (equazione caratteristica)\\
      \[
         \displaystyle\xRightarrow[\mathrm{fondamentale}]{\mathrm{Teorema}}\sum_{i = 0}^{n}a_is^i = a_n(s - \lambda_1)^{\mu_1}\cdot...\cdot(s - \lambda_r)^{\mu_r} \Rightarrow H(s) = \frac{\overline{n}(s)}{(s - \lambda_1)^{\mu_1}\cdot...\cdot(s - \lambda_r)^{\mu_r}}, \mathrm{\,\,dove\,\,}\overline{n}(s) = \frac{n(s)}{a_n}
      \]
      Posso anche scrivere
      \[
         \frac{(d(s)) \rightarrow b_m(s - p_1)\cdot...\cdot(s - p_m)}{(n(s)) \rightarrow a_n(s - z_1)\cdot...\cdot(s - z_n)}
      \]
      $z_i, i = \overline{1, m}$ zeri della TdL\\
      $p_i, i = \overline{1, n}$ poli della TdL\\
      Possiamo ridefinire la molteplicità:\\
      $\alpha\in\mathbb{C}$ un polo di molteplicità $k\in\mathbb{N}$ se $\displaystyle\lim_{s\to\infty}(s - \alpha)^{k - 1}H(s) = \infty\quad\lim_{s\to\infty}(s - \alpha)^kH(s) < +\infty$\\
      $\beta\in\mathbb{C}$ uno zero di molteplicità $k\in\mathbb{N}$ se $\displaystyle\lim_{s\to\beta}\frac{1}{(s - \beta)^{k - 1}}H(s) = 0\quad\lim_{s\to\beta}\frac{1}{(s - \beta)^k}H(s) \neq 0$
      \begin{center}
         \begin{tabular}{lll}
            \toprule
            \_ & Poli & Zeri \\
            \midrule
            $\lambda_i$ & $\alpha$ & $\beta$ \\
            $\mu_i$ & k & k \\
            \bottomrule
         \end{tabular}
      \end{center}
      \begin{osservazione}
         $z_i\in\{p_1,\dots,p_n\} \Rightarrow \frac{n(s)}{d(s)}$ è riucibile\\
         $\Rightarrow \{\mathrm{zeri\,\,}H(s)\} \subseteq \{\mathrm{zeri\,\,di\,\,}n(s)\}$\\
         $\Rightarrow \{\mathrm{poli\,\,}H(s)\} \subseteq \{\mathrm{zeri\,\,di\,\,}d(s)\}$
      \end{osservazione}
      \emph{Proprietà}: Il sistema è BIBO stabile se e solo se tutti i poli hanno parte reale minore di 0 $\forall i, \Re(p_i) < 0$\\
      \NB: I poli di $H(s)$ sono zeri di $d(s)$
%\end{document}
